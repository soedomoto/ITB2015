%
% Halaman Abstrak
%
% @author  Andreas Febrian
% @version 1.00
%

\chapter*{Abstrak}

\vspace*{0.2cm}

\begin{center}
	{\large \textbf{PERANCANGAN SISTEM REKOMENDASI LOKASI PENCACAHAN SECARA \textit{REAL-TIME} BERBASIS KONTEKS}} \\
	\vspace*{0.2cm}
	Oleh \\
	{\large \textbf{Aris Prawisudatama}} \\
	{\large \textbf{NIM: 23215131}} \\
	{\large \textbf{(Program Studi Magister Teknik Elektro)}}
\end{center}


\vspace*{0.5cm}

\noindent Pengumpulan data lapangan merupakan salah satu tugas dan kewenangan yang dimiliki oleh institusi statistik sebuah negara, tidak terkecuali Badan Pusat Statistik (BPS). Pada praktiknya, BPS menggunakan Blok Sensus (BS) yang merupakan wilayah kerja dari seorang petugas pencacahan. Pengalokasian wilayah kerja seringkali dilakukan secara subyektif, sehingga menimbulkan ketimpangan waktu penyelesaian antar petugas pencacahan, yang pada akhirnya menyebabkan keterlambatan kegiatan pengumpulan data secara keseluruhan. Walaupun permasalahan alokasi petugas pengumpulan data memiliki kemiripan dengan permasalahan \textit{Multi Depot Vehicle Routing Problem} (VRP), algoritma penyelesaian MDVRP tidak serta merta dapat digunakan, karena informasi terkait lama pencacahan pada suatu wilayah kerja tidak tersedia. \\

\noindent Pada penelitian ini diusulkan sebuah sistem yang dapat digunakan untuk membuat rekomendasi yang lebih merata. Sistem usulan bekerja secara bertahap, dengan menggabungkan algoritma penyelesaian MDVRP dengan mekanisme \textit{Publish/Subscribe}. Agar rekomendasi yang dibuat oleh sistem akurat, digunakan konteks dari setiap petugas pada saat pencarian solusi. \\

\noindent Pengujian dilakukan dengan membandingkan sistem usulan dengan program pembanding, yaitu algoritma MDVRP tanpa menggunakan mekanisme Publish/Subscribe. Berdasarkan hasil pengujian, sistem usulan dapat memberikan rekomendasi dengan lebih efisien pada sebagaian besar kasus.  \\

\vspace*{0.2cm}

\noindent Kata Kunci: rekomendasi; \textit{location routing}; VRP; MDVRP; \textit{publish/subscribe} \\

\newpage