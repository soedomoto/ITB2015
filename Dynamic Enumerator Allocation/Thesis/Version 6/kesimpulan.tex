%---------------------------------------------------------------
\chapter{\kesimpulan}
%---------------------------------------------------------------
%\todo{Tambahkan kesimpulan dan saran terkait dengan perkerjaan 
%	yang dilakukan.}


%---------------------------------------------------------------
\section{Kesimpulan}
%---------------------------------------------------------------
Berdasarkan hasil pengujian, diperoleh kesimpulan sebagai berikut:

\begin{enumerate}
	\item Pada permasalahan MDVRP yang mempertimbangkan lama waktu pencacahan, dari pengujian dengan menggunakan data Cordeau dengan 7 (tujuh) buah \textit{instance}, diperoleh hasil bahwa sistem usulan, yaitu MDVRP berbasis CoEAs dengan mekanisme Publish/Subscribe, menghasilkan `rata-rata hari yang diperlukan tiap-tiap pencacah' yang sama dengan program pembanding, yaitu MDVRP berbasis CoEAs tanpa mekanisme Publish/Subscribe, pada seluruh instance. Akan tetapi dari segi `standar deviasi hari pencacahan yang diperlukan tiap-tiap pencacah', sistem usulan memberikan hasil yang lebih baik pada seluruh instance, atau sebesar 100 persen lebih baik. Selain itu, waktu tunggu sampai seluruh pencacah menyelesaikan pencacahan dengan menggunakan sistem usulan lebih rendah dibandingkan program usulan pada seluruh instance, atau 100 persen lebih baik. Hal ini menunjukkan sistem usulan lebih efisien digunakan pada pencacahan kondisi normal dengan menggunakan data Cordeau.
	
	\item Pada permasalahan MDVRP yang mempertimbangkan lama waktu pencacahan, dari pengujian dengan menggunakan data lapangan dengan 4 (empat) buah \textit{instance}, diperoleh hasil bahwa sistem usulan, yaitu MDVRP berbasis CoEAs dengan mekanisme Publish/Subscribe, memberikan hasil yang sama dibandingkan dengan program pembanding, yaitu MDVRP berbasis CoEAs tanpa mekanisme Publish/Subscribe dari segi `rata-rata hari yang diperlukan tiap-tiap pencacah'. Sementara dari segi `standar deviasi hari pencacahan yang diperlukan tiap-tiap pencacah', sistem usulan memberikan hasil yang lebih baik pada seluruh instance, atau sebesar 100 persen lebih baik. Selain itu, waktu tunggu sampai seluruh pencacah menyelesaikan pencacahan dengan menggunakan sistem usulan lebih rendah dibandingkan program usulan pada seluruh instance, atau 100 persen lebih baik. Hal ini menunjukkan sistem usulan juga lebih efisien digunakan pada pencacahan kondisi normal dengan menggunakan data administratif.
	
	\item Pada permasalahan MDVRP yang mempertimbangkan lama waktu pencacahan dan \textit{delay}, dari pengujian dengan menggunakan data lapangan dengan 7 (tujuh) buah \textit{instance}, diperoleh hasil bahwa sistem usulan menghasilkan hari pencacahan yang lebih lama dari segi `rata-rata hari yang diperlukan tiap-tiap pencacah' pada seluruh \textit{instance}, atau 100 persen lebih buruk. Akan tetapi, sistem usulan menghasilkan standar deviasi `standar deviasi hari pencacahan yang diperlukan tiap-tiap pencacah' yang lebih rendah pada seluruh instance, atau 100 persen lebih baik, yang artinya penyelesaian pencacahan lebih merata dengan sistem usulan. Sementara itu, dari segi waktu tunggu sampai seluruh pencacah menyelesaikan pencacahan, sistem usulan memberikan hasil yang lebih baik pada 3 (tiga) \textit{instance} yang memiliki delay relatif rendah, tetapi lebih buruk pada 4 (empat) \textit{instance} yang memiliki delay tinggi, atau hanya 43 persen lebih baik. Hal ini menunjukkan pada kondisi terdapat delay yang cukup lama, sistem usulan memberikan hasil yang tidak lebih baik jika dibandingkan dengan sistem pembanding.
	
	\item Dari seluruh pengujian dengan data dan skenario yang bervariasi, dapat disimpulkan bahwa algoritma MDVRP berbasis CoEAs dengan mekanisme Publish/Subscribe memberikan hasil yang lebih merata antar petugas dibandingkan dengan algoritma MDVRP berbasis CoEAs tanpa mekanisme Publish/Subscribe. Sementara dari segi waktu tunggu yang diperlukan sampai dengan seluruh pencacah menyelesaikan pencacahannya, algoritma MDVRP berbasis CoEAs dengan mekanisme Publish/Subscribe memberikan hasil lebih baik pada kondisi tidak terdapat delay atau terdapat delay yang rendah. Sebaliknya, pada kondisi terdapat delay yang tinggi, \textit{pre-calculated routes} yang dihasilkan oleh algoritma MDVRP berbasis CoEAs tanpa mekanisme Publish/Subscribe memberikan waktu tunggu yang lebih rendah.
\end{enumerate}


%---------------------------------------------------------------
\section{Saran}
%---------------------------------------------------------------
Dari penelitian ini, saran yang diberikan untuk penelitian selanjutnya sebagai berikut:

\begin{enumerate}
	\item Terdapat beberapa mekanisme komunikasi yang dapat digunakan sebagai alternatif dari mekanisme Publish/Subscribe, seperti Request/Reply dan Push/Pull. Untuk itu, diperlukan kajian untuk mendapatkan mekanisme komunikasi yang paling tepat.
	\item Pencarian solusi/rute oleh VRP Solver dijalankan hingga diperoleh rute yang konvergen. Parameter yang digunakan pada VRP solver bervariasi dalam hal total waktu eksekusi, total iterasi, waktu/iterasi yang digunakan sebagai syarat rute konvergen, dan penalti. Untuk itu diperlukan kajian untuk menentukan parameter yang paling optimal.
	\item Terdapat beberapa algoritma yang dapat digunakan untuk menyelesaikan permasalahan terkait MDVRP, seperti Tabu Search, Particle Swarm Optimization, dan Artificial Bee Colony. Untuk itu diperlukan kajian tentang penggunaan algoritma yang lain, sehingga diperoleh algoritma MDVRP yang paling sesuai.
	\item Penerapan sistem dalam skala global, misalnya untuk seluruh wilayah Indonesia, melibatkan sejumlah besar pencacah maupun lokasi pencacahan yang berakibat pada lamanya waktu yang diperlukan sampai dengan rute yang dihasilkan konvergen. Untuk itu, diperlukan kajian tentang penerapan sistem dalam skala global.
\end{enumerate}
