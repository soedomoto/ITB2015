%
% Template Laporan Skripsi/Thesis 
%
% @author  Andreas Febrian, Lia Sadita 
% @version 1.03
%
% Dokumen ini dibuat berdasarkan standar IEEE dalam membuat class untuk 
% LaTeX dan konfigurasi LaTeX yang digunakan Fahrurrozi Rahman ketika 
% membuat laporan skripsi. Konfigurasi yang lama telah disesuaikan dengan 
% aturan penulisan thesis yang dikeluarkan UI pada tahun 2008.
%

%
% Tipe dokumen adalah report dengan satu kolom. 
%
\documentclass[12pt, a4paper, onecolumn, oneside, final]{report}

% Load konfigurasi LaTeX untuk tipe laporan thesis
\usepackage{itb_style}


% Load konfigurasi khusus untuk laporan yang sedang dibuat
%-----------------------------------------------------------------------------%
% Informasi Mengenai Dokumen
%-----------------------------------------------------------------------------%
% 
% Judul laporan. 
\var{\judul}{Pengembangan Sistem Rekomendasi Lokasi Pencacahan Secara \textit{Real Time} Berbasis \textit{Context}}
% 
% Tulis kembali judul laporan, kali ini akan diubah menjadi huruf kapital
\Var{\Judul}{Pengembangan Sistem Rekomendasi Lokasi Pencacahan Secara \textit{Real Time} Berbasis \textit{Context}}
% 
% Tulis kembali judul laporan namun dengan bahasa Ingris
\var{\judulInggris}{Context-aware Real Time Location Recommendation System}

% 
% Tipe laporan, dapat berisi Skripsi, Tugas Akhir, Thesis, atau Disertasi
\var{\type}{Thesis}
% 
% Tulis kembali tipe laporan, kali ini akan diubah menjadi huruf kapital
\Var{\Type}{Thesis}
% 
% Tulis nama penulis 
\var{\penulis}{Aris Prawisudatama}
% 
% Tulis kembali nama penulis, kali ini akan diubah menjadi huruf kapital
\Var{\Penulis}{Aris Prawisudatama}
% 
% Tulis NPM penulis
\var{\npm}{23215131}
% 
% Tuliskan Fakultas dimana penulis berada
\Var{\Fakultas}{Sekolah Teknik Elektro dan Informatika}
\var{\fakultas}{Sekolah Teknik Elektro dan Informatika}
% 
% Tuliskan Program Studi yang diambil penulis
\Var{\Program}{Magister Teknik Elektro}
\var{\program}{Magister Teknik Elektro}
% 
% Tuliskan tahun publikasi laporan
\Var{\bulanTahun}{Januari 2016}
% 
% Tuliskan gelar yang akan diperoleh dengan menyerahkan laporan ini
\var{\gelar}{Master}
% 
% Tuliskan tanggal pengesahan laporan, waktu dimana laporan diserahkan ke 
% penguji/sekretariat
\var{\tanggalPengesahan}{XX April 2017} 
% 
% Tuliskan tanggal keputusan sidang dikeluarkan dan penulis dinyatakan 
% lulus/tidak lulus
\var{\tanggalLulus}{XX April 2017}
% 
% Tuliskan pembimbing 
\var{\pembimbing}{Dr. I Gusti Bagus Baskara Nugraha}
% 
% Alias untuk memudahkan alur penulisan paa saat menulis laporan
\var{\saya}{Penulis}

%-----------------------------------------------------------------------------%
% Judul Setiap Bab
%-----------------------------------------------------------------------------%
% 
% Berikut ada judul-judul setiap bab. 
% Silahkan diubah sesuai dengan kebutuhan. 
% 
\Var{\babSatu}{Pendahuluan}
\Var{\babDua}{Studi Literatur}
\Var{\babTiga}{Metodologi}
\Var{\babEmpat}{Analisis dan Perancangan}
\Var{\babLima}{Implementasi dan Pengujian}
\Var{\kesimpulan}{Kesimpulan dan Saran}

% Logo ITB
\var{\logoItb}{Resources/Images/itb}
% Bibliography
\var{\bibFile}{Resources/Bib/Reference}

% Daftar pemenggalan suku kata dan istilah dalam LaTeX
\include{hype.indonesia}
% Daftar istilah yang mungkin perlu ditandai 
\input{istilah}

% Awal bagian penulisan laporan
\begin{document}
	

%
% Sampul Laporan
%
% Sampul Laporan

%
% @author  unknown
% @version 1.01
% @edit by Andreas Febrian
%

\begin{titlepage}
	\begin{center}
		{\large\bfseries  %
			\expandafter \uppercase \expandafter{\Judul}%
			\\}
		% ----------------------------------------------------------------
		% \vspace{2cm}
		%	{Thesis  submitted to} \\[5pt]
		%	\emph{{Your University}}\\[2cm]
		\vspace{2cm}
		\expandafter\uppercase\expandafter{\Type}\\[1.5cm]
		Karya tulis sebagai salah satu syarat\\
		Untuk memperoleh gelar Magister dari\\
		Institut Teknologi Bandung\\[1.5cm]
		\textsc{\Large{{ }}} \\[5pt]
		{ } \vspace{0.4cm} 
		% {By}\\[5pt] {\Large \sc {Me}}
		\vfill
		% ----------------------------------------------------------------
		
		\vspace{6ex}
		{\small\bfseries{oleh}\\}
		{\normalsize\bfseries
			\expandafter\uppercase\expandafter{\Penulis}%
			\\
			23215131\\
		}
		\vspace{10ex}
		% ----------------------------------------------------------------
		\includegraphics[height=3.5cm]{\logoItb}\\[5pt]
		\vspace{2cm}
		{\bfseries \large \uppercase
			{Institut Teknologi Bandung}\\[5pt]
			{2017}
		}
		\vfill
		
	\end{center}
\end{titlepage}

%\begin{titlepage}
%    \begin{center}    
%        \begin{figure}
%            \begin{center}
%                \includegraphics[width=2.5cm]{\logoItb}
%            \end{center}
%        \end{figure}    
%        \vspace*{0cm}
%        \bo{
%        	INSTITUT TEKNOLOGI BANDUNG\\
%        }
%        
%        \vspace*{1.0cm}
%        % judul thesis harus dalam 14pt Times New Roman
%        \bo{\Judul} \\[1.0cm]
%
%        \vspace*{2.5 cm}    
%        % harus dalam 14pt Times New Roman
%        \bo{\Type}
%
%        \vspace*{3 cm}       
%        % penulis dan npm
%        \bo{\Penulis} \\
%        \bo{\npm} \\
%
%        \vspace*{5.0cm}
%
%        % informasi mengenai fakultas dan program studi
%        \bo{
%        	\Fakultas\\
%        	PROGRAM STUDI \Program \\
%        	BANDUNG \\
%        	\bulanTahun
%        }
%    \end{center}
%\end{titlepage}


%
% Gunakan penomeran romawi
\pagenumbering{roman}

%%
%% load halaman judul dalam
%\addChapter{HALAMAN JUDUL}
%\include{judul_dalam}

%
% setelah bagian ini, halaman dihitung sebagai halaman ke 2
\setcounter{page}{2}

%%
%% load halaman pengesahan
%\addChapter{LEMBAR PERSETUJUAN}
%\include{pengesahan}
%%
%% load halaman orisinalitas 
%\addChapter{LEMBAR PERNYATAAN ORISINALITAS}
%\include{orisinal}


%
% 
\addChapter{ABSTRAK}
%
% Halaman Abstrak
%
% @author  Andreas Febrian
% @version 1.00
%

\chapter*{Abstrak}

\vspace*{0.2cm}

\begin{center}
	{\large \textbf{PERANCANGAN SISTEM REKOMENDASI LOKASI PENCACAHAN SECARA \textit{REAL-TIME} BERBASIS KONTEKS}} \\
	\vspace*{0.2cm}
	Oleh \\
	{\large \textbf{Aris Prawisudatama}} \\
	{\large \textbf{NIM: 23215131}} \\
	{\large \textbf{(Program Studi Magister Teknik Elektro)}}
\end{center}


\vspace*{0.5cm}

\noindent Pengumpulan data lapangan merupakan salah satu tugas dan kewenangan yang dimiliki oleh institusi statistik sebuah negara, tidak terkecuali Badan Pusat Statistik (BPS). Pada praktiknya, BPS menggunakan Blok Sensus (BS) yang merupakan wilayah kerja dari seorang petugas pencacahan. Pengalokasian wilayah kerja seringkali dilakukan secara subyektif, sehingga menimbulkan ketimpangan waktu penyelesaian antar petugas pencacahan, yang pada akhirnya menyebabkan keterlambatan kegiatan pengumpulan data secara keseluruhan. Walaupun permasalahan alokasi petugas pengumpulan data memiliki kemiripan dengan permasalahan \textit{Multi Depot Vehicle Routing Problem} (VRP), algoritma penyelesaian MDVRP tidak serta merta dapat digunakan, karena informasi terkait lama pencacahan pada suatu wilayah kerja tidak tersedia. \\

\noindent Pada penelitian ini diusulkan sebuah sistem yang dapat digunakan untuk membuat rekomendasi yang lebih merata. Sistem usulan bekerja secara bertahap, dengan menggabungkan algoritma penyelesaian MDVRP dengan mekanisme \textit{Publish/Subscribe}. Agar rekomendasi yang dibuat oleh sistem akurat, digunakan konteks dari setiap petugas pada saat pencarian solusi. \\

\noindent Pengujian dilakukan dengan membandingkan sistem usulan dengan program pembanding, yaitu algoritma MDVRP tanpa menggunakan mekanisme Publish/Subscribe. Berdasarkan hasil pengujian, sistem usulan dapat memberikan rekomendasi dengan lebih efisien pada sebagaian besar kasus.  \\

\vspace*{0.2cm}

\noindent Kata Kunci: rekomendasi; \textit{location routing}; VRP; MDVRP; \textit{publish/subscribe} \\

\newpage

%
%
\addChapter{ABSTRACT}
%
% Halaman Abstract
%
% @author  Andreas Febrian
% @version 1.00
%

\chapter*{ABSTRACT}

\vspace*{0.2cm}

\begin{center}
	{\large \textbf{CONTEXT-AWARE REAL-TIME LOCATION RECOMMENDATION SYSTEM}} \\
	\vspace*{0.2cm}
	Oleh \\
	{\large \textbf{Aris Prawisudatama}} \\
	{\large \textbf{NIM: 23215131}} \\
	{\large \textbf{(Master of Electrical Engineering)}}
\end{center}

\vspace*{0.5cm}

\noindent Data collection is one of the main duties of a statistical agency in every country, including Badan Pusat Statistik (BPS). Practically, BPS use Census Block (BS) that is working area of every enumerator. Allocation of working area often done subjectively, causing imbalance completion time between enumerators. It will have an impact on the overall completion of the data collection. Despite the problem of enumerator allocation has similarities with MDVRP problem, MDVRP solver algorithm cannot directly solve this problem, because information of enumeration time is not available. \\

\noindent In this research, a system that can be used for generating recommendation of location is proposed. This system aims to provide a more equitable recommendation. Proposed system works gradually, by combining MDVRP solver algorithm with Publish/Subscribe mechanism. It will use context of each enumerator to make recommendation precisely.

\noindent Experiment was conducted by comparing proposed system and benchmark program, that is MDVRP algorithm without Publish/Subscribe mechanism. The result shows proposed system generate better recommendation in the most case.

\vspace*{0.2cm}

\noindent Keywords: recommendation; location routing; VRP; MDVRP; publish/subscribe \\

\newpage

%
%
\addChapter{LEMBAR PENGESAHAN}
%
% Halaman Pengesahan Sidang
%
% @author  Andreas Febrian, Andre Tampubolon 
% @version 1.02
%

\singlespacing
%\chapter*{HALAMAN PENGESAHAN}
%\medskip
\begin{center}
	{\large \textbf{\Judul}} \\
	\vspace*{5\baselineskip}
	Oleh \\
	{\large \textbf{Aris Prawisudatama}} \\
	{\large \textbf{NIM: 23215131}} \\
	{\large \textbf{(Program Studi Magister Teknik Elektro)}}\\
	\vspace*{\baselineskip}
	Institut Teknologi Bandung\\
	\vspace*{5\baselineskip}
	Menyetujui\\
	Pembimbing\\
	\vspace*{\baselineskip}
	27 April 2017\\
	\vspace*{7\baselineskip}
	\rule[0.5ex]{8.2cm}{1pt}\\
	(\pembimbing)
\end{center}


%\vspace*{0.4cm}
%\noindent 
%
%\noindent
%\begin{tabular}{ll p{9cm}}
%	\type~ini diajukan oleh&: & \\
%	Nama&: & \penulis \\
%	NPM&: & \npm \\
%	Program Studi&: & \program \\
%	Judul \type&: & \judul \\
%\end{tabular} \\
%
%\vspace*{1.0cm}
%
%\noindent \bo{Telah berhasil dipertahankan di hadapan Dewan Penguji 
%dan diterima sebagai bagian persyaratan yang diperlukan untuk 
%memperoleh gelar \gelar~pada Program Studi \program, Fakultas 
%\fakultas, Universitas Indonesia.}\\[0.2cm]
%
%\begin{center}
%	\bo{DEWAN PENGUJI}
%\end{center}
%
%\vspace*{0.3cm}
%
%\begin{tabular}{l l l l }
%	& & & \\
%	Pembimbing&: & \pembimbing & (\hspace*{3.0cm}) \\
%	& & & \\
%	Penguji&: & Prof. XXX & (\hspace*{3.0cm}) \\
%	& & & \\
%	Penguji&: & Prof. XXXX & (\hspace*{3.0cm}) \\
%	& & & \\
%	Penguji&: & Prof. XXXXXX & (\hspace*{3.0cm}) \\
%\end{tabular}\\
%
%\todo{Jangan lupa mengisi nama para penguji.}
%
%\vspace*{2.0cm}
%
%\begin{tabular}{ll l}
%	Ditetapkan di&: & Bandung\\
%	Tanggal&: & \tanggalLulus \\
%\end{tabular}

\onehalfspacing
\newpage

%
%
\addChapter{PEDOMAN PENGGUNAAN TESIS}
% 
% @author  Andre Tampubolon, Andreas Febrian
% @version 1.01
% 

\chapter*{PEDOMAN PENGGUNAAN TESIS}


\noindent Tesis S2 yang tidak dipublikasikan terdaftar dan tersedia di Perpustakaan Institut Teknologi Bandung, dan terbuka untuk umum dengan ketentuan bahwa hak cipta ada pada pengarang dengan mengikuti aturan HaKI yang berlaku di Institut Teknologi Bandung. Referensi kepustakaan diperkenankan dicatat, tetapi pengutipan atau peringkasan hanya dapat dilakukan seizin pengarang dan harus disertai dengan kaidah ilmiah untuk menyebutkan sumbernya.\\


\noindent Sitasi hasil penelitian Tesis ini dapat ditulis dalam bahasa Indonesia sebagai berikut:\\


\hangindent=2em
\hangafter=1
\noindent Prawisudatama, A. (2017): \textit{\judul}, Tesis Program Magister, Institut Teknologi Bandung.\\


\noindent dan dalam bahasa Inggris sebagai berikut:\\


\hangindent=2em
\hangafter=1
\noindent Prawisudatama, A. (2017): \textit{\judulInggris}, Master’s Program Thesis, Institut Teknologi Bandung.\\


\noindent Memperbanyak atau menerbitkan sebagian atau seluruh tesis haruslah seizin Dekan Sekolah Pascasarjana, Institut Teknologi Bandung.

\newpage



%
%
\addChapter{LEMBAR PERUNTUKAN}
%
% @author  Andreas Febrian
% @version 1.00 
% 
% Hanya sebuah pembatas bertuliskan LAMPIRAN ditengah halaman. 
%

\begin{center}
	\topskip0pt
	\vspace*{\fill}
	\noindent \textit{Dipersembahkan kepada orang tua, istriku, dan anakku}
	\vspace*{\fill}
\end{center}

\renewcommand{\thetable}{\arabic{table}}

%
%
\addChapter{KATA PENGANTAR}
%-----------------------------------------------------------------------------%
\chapter*{KATA PENGANTAR}
%-----------------------------------------------------------------------------%

\noindent Alhamdulillahi rabbil 'alamin, puji syukur kepada Allah Subhanahu wa Ta'ala, karena berkat rahmat dan kuasa-Nya, penulis dapat menyelesaikan tesis yang berjudul "\judul" pada Program Studi Magister Teknik Elektro, Sekolah Teknik Elektro dan Informatika, Institut Teknologi Bandung.\\
\\
\noindent Tesis ini dapat diselesaikan dengan baik berkat bimbingan, arahan, petunjuk, dan bantuan dari berbagai pihak. Oleh karena itu, penulis mengucapkan terima kasih dan penghargaan sebesar-besarnya kepada:

\begin{enumerate}
	\item Erika Mukhlisina Siregar, istri yang selalu memberikan semangat, dukungan, dan doa yang tulus beserta anakku tercinta, Muhammad Alif Arka Putra.
	\item Kedua orang tua yang selalu memberikan semangat dan dukungan kepada penulis dalam menyelesaikan pendidikan dan penyusuan tesis ini.
	\item Bapak Dr. I Gusti Bagus Baskara Nugraha, S.T., M.T. selaku pembimbing yang dengan sabar memberikan bimbingan, arahan, masukan, kritik, saran, dan koreksi selama penyelesaian tesis ini.
	\item Ibu Dr. Ir. Aciek Ida Wuryandari, M.T., Ibu Dra. Harlili, M.Sc, dan Bapak Fadhil Hidayat, S.Kom., MT. selaku penguji dari Institut Teknologi Bandung yang telah memberikan masukan dan arahan untuk penyempurnaan tesis ini.
	\item Bapak Dr. Ir. Albarda, MT. selaku dosen wali yang telah membimbing penulis selama menempuh pendidikan di Institut Teknologi Bandung.
	\item Seluruh dosen dan staf Sekolah Teknik Elektro dan Informatika Institut Teknologi Bandung yang telah meluangkan waktu untuk memberikan ilmu dan bantuannya.
	\item Pimpinan Badan Pusat Statistik yang telah memberikan kesempatan kepada penulis untuk melanjutkan pendidikan di Institut Teknologi Bandung.
	\item Teman-teman seperjuangan yang selalu memberikan dukungan dan membantu selama perkuliahan.
	\item Dan semua pihak yang tidak dapat penulis sebutkan satu persatu yang telah mendukung dan membantu penulis dalam penyelesaian tesis ini.
\end{enumerate}

\noindent Penulis menyadari bahwa tesis ini masih banyak kekurangan dan perlu pengembangan lebih lanjut. Oleh karena itu, penulis sangat mengharapkan kritik dan saran yang membangun agar tesis ini dapat menjadi lebih baik serta dapat menjadi masukan bagi penulis untuk penelitian di masa yang akan datang.

\vspace*{0.1cm}
\begin{flushright}
Bandung, April 2017\\[0.1cm]
\vspace*{1cm}
\penulis

\end{flushright}

%
% Daftar isi, gambar, dan tabel
%
\singlespacing
\tableofcontents
\clearpage
%\listofappendices
%\clearpage
{%
	\let\oldnumberline\numberline%
	\renewcommand{\numberline}{\figurename~\oldnumberline}%
	\listoffigures%
}
\clearpage
{%
	\let\oldnumberline\numberline%
	\renewcommand{\numberline}{\tablename~\oldnumberline}%
	\listoftables%
}
\clearpage
{%
	\let\oldnumberline\numberline%
	\renewcommand{\numberline}{\algorithmname~\oldnumberline}%
	\listofalgorithms%
}
\clearpage
{%
	\let\oldnumberline\numberline%
	\renewcommand{\numberline}{\listingsname~\oldnumberline}%
	\listoflistings%
}
\clearpage
\onehalfspacing

%
% Gunakan penomeran Arab (1, 2, 3, ...) setelah bagian ini.
%
\pagenumbering{arabic}

%
%
%
\chapter{PENDAHULUAN} \label{ch:chapter_1}

\section{Latar Belakang}
Badan Pusat Statistik (BPS) merupakan suatu lembaga pemerintah non-departemen yang bertanggung jawab dalam penyediaan statistik dasar \cite{_badan_????}. Dalam peranannya sebagai penyedia data, BPS melakukan pengumpulan data dengan 2 (dua) metode : primer dan sekunder. Pengumpulan data primer berarti BPS secara mandiri mengumpulkan data dengan menggunakan metode wawancara langsung dengan responden, baik responden individu, rumah tangga, maupun perusahaan. Sementara pengumpulan data sekunder berarti BPS memperoleh data dari pihak lain.

Dalam melakukan kegiatan perstatistikan, yang selanjutnya merujuk kepada pengumpulan data primer, BPS merujuk kepada \textit{General Statistical Business Process Model} (GSBPM) ~\cite{_gsbpm_????}. GSBPM merupakan suatu standard arsitektur bisnis kegiatan perstatistikan yang dirumuskan oleh \textit{United Nations Economic Commission for Europe} (UNECE). Dalam GSBPM, \textit{Business Process} Statistik dibagi menjadi 8 (tujuh) phase : \textit{Specify Needs}, \textit{Design}, \textit{Build}, \textit{Collect}, \textit{Process}, \textit{Analyze}, \textit{Disseminate}, dan \textit{Evaluate}, dimana masing-masing phase dapat dijabarkan lagi dalam beberapa sub-proses.

\begin{figure}
    \centering
    \includegraphics[width=13cm]{../../Resources/Images/gsbpm}
    \caption{\textit{Statistical Business Process Phases} dalam GSBPM}
    \label{fig:gsbpm}
\end{figure}

Pengumpulan dan pengolahan data dalam GSBPM tercakup dalam 3 (tiga) fase, yaitu : \textit{Collect Phase}, \textit{Process Phase}, dan \textit{Analyze Phase}. \textit{Collect phase} adalah fase dimana semua informasi (data dan metadata) dikumpulkan dengan menggunakan beberapa metode pengumpulan (termasuk ekstraksi dari register dan database statistik, administratif, maupun yang lain), dan memuatkannya ke dalam suatu \textit{environment} untuk pemrosesan lebih lanjut. \textit{Process phase} adalah fase dimana data dibersihkan dan dipersiapkan untuk tahap berikutnya, yaitu \textit{analysis phase}. \textit{Collect phase} dan \textit{process phase} dapat dilakukan secara berulang dan paralel. Fase terakhir sebelum data siap untuk didesiminasikan adalah \textit{analyze phase}. Pada tahap \textit{analyze phase}, data ditransformasikan kedalam bentuk \textit{statistical output} yang disesuaikan dengan kebutuhan (\textit{fit for purpose}).

Kondisi saat ini, \textit{process phase} dan \textit{analyze phase} merupakan tahapan yang memiliki ketergantungan akan Teknologi Informasi dan Komunikasi (TIK) yang sangat besar. \textit{Process phase} merupakan tahapan dimana dilakukan input data hasil pendataan lapangan dari format kuesioner ke dalam format digital, termasuk didalamnya pengkodean, imputasi, validasi, dan penghitungan penimbang. Sementara \textit{analyze phase} memerlukan keterlibatan software analisis yang membantu mentransformasikan data menjadi sebuah informasi. Adapun \textit{collect phase}, meskipun saat ini masih menggunakan pengumpulan data dengan mengadopsi \textit{paper questionaire}, tetapi kedepannya akan dilakukan transformasi dengan menggunakan metode \textit{Computer Assisted Personal Interviewing} (CAPI)\footnote{Keterangan Dr. Said Mirza Pahlevi, M.Eng., Kepala Subdirektorat Pengembangan Basis Data, 24 Februari 2016}, meskipun feasibility-nya belum pernah diujicobakan\footnote{Keterangan Dr. Muchammad Romzi, Kepala Subdirektorat Pengembangan Model Statistik, 4 Maret 2016}. 

Penggunaan metode CAPI dalam pengumpulan data yang dilakukan BPS, sedikit banyak akan mengubah paradigma pengumpulan dan pengolahan data yang selama ini telah berjalan. Pengumpulan dan pengolahan data selama ini merupakan dua buah tahapan yang terpisah. Dengan diterapkannya CAPI maka beberapa sub-proses dari \textit{process phase}, seperti pengkodean dan validasi, dapat dilakukan secara terintegrasi dengan pengumpulan data. Penggunaan CAPI (\textit{device}) juga dapat menghemat waktu dan biaya \cite{wright_researching_2005}. Metode CAPI sebenarnya bukanlah sebuah hal yang baru. Metode ini sudah ada sejak beberapa dekade terakhir ~\cite{_redesigning_????}. Bahkan sebuah penelitian yang dilakukan oleh Gary Klein dkk menyatakan pengumpulan data dengan menggunakan metode CAPI berpotensi terjadi bias, terutama dalam akurasi, \textit{completeness}, dan \textit{item omission} ~\cite{klein_bias_1996}. Akan tetapi, dengan semakin berkembangnya teknologi \textit{mobile computing} yang dipadukan dengan penggunaan \textit{Web service} ~\cite{tergujeff_mobile_2007}, maka potensi bias dapat dikurangi dengan merancang sejumlah \textit{service} yang berfungsi untuk menvalidasi hasil pendataan.

Implementasi \textit{Web service} pada pengumpulan data dengan menggunakan CAPI bukanlah tanpa kendala. Petugas pengumpulan data harus berpindah-pindah dari satu lokasi pendataan ke lokasi yang lain untuk mengunjungi responden. Dikarenakan keterbatasan infrastruktur seperti sinyal telekomunikasi dan daya tahan baterai dari \textit{device} yang digunakan, seringkali tidak mudah bagi \textit{device} untuk selalu terhubung dengan jaringan internet dan berkomunikasi dengan \textit{Web service}. \textit{Device} setiap saat dapat dengan mudah berubah dari \textit{connected node} menjadi \textit{disconnected node} dan sebaliknya. 

\begin{figure}[h]
    \centering
    \includegraphics[height=6cm]{../../Resources/Images/capi-ilustration}
    \caption{Ilustrasi CAPI}
    \label{fig:capi-ilustration}
\end{figure}

Tidak terhubungnya suatu \textit{mobile node} dalam sebuah jaringan dapat dikategorikan menjadi 3 (tiga) jenis (Gutwin dkk, 2010) \cite{gutwin_gone_2010}, yaitu:
\begin{itemize}
\item \textit{Delay-based Interruption}, merupakan sebuah gap singkat (\textit{short-term gap}) dalam pengiriman pesan. \textit{Delay} dapat disebabkan oleh berbagai faktor, tetapi porsi terbesar penyebab \textit{delay} adalah \textit{transmission delay}, \textit{contention delay}, dan \textit{queuing delay} ~\cite{zhang_interference-based_2015}.
\item \textit{Network Outage}, merupakan kondisi dimana sebuah \textit{mobile node} terputus dari jaringannya. Kondisi \textit{network outage} dapat disebabkan oleh berbagai faktor, seperti: bencana (kebakaran di \textit{Baltimore Howard Street Tunnel \cite{mcgrattan_numerical_2006}} atau terputusnya \textit{Mediterranean Cable} \cite{zmijewski_mediterranean_2008}), kesalahan konfigurasi (\textit{Pakistani Youtube routing} \cite{hunter_pakistan_2008}), terorisme (misalnya, serangan \textit{World Trade Center} \cite{partridge_internet_2003} atau serangan gelombang elektromagnetik \cite{foster_jr_report_2004}), atau \textit{censorship} (misalnya, respons terhadap kebangkitan masyarakat mesir 2011 \cite{dainotti_analysis_2011})
\item \textit{Explicit Departures}, merupakan kondisi \textit{mobile node} keluar dari jaringan atau keluar dari aplikasi secara eksplisit
\end{itemize}

Brian DeRenzi dkk telah melakukan penelitian tentang cara pengumpulan data berbasis \textit{mobile phone} pada lingkungan yang \textit{highly disconnected} ~\cite{derenzi_reliable_2007} dengan menggunakan \textit{CAM Framework}. \textit{CAM framework}~\cite{parikh_designing_2006} terbukti dapat digunakan dalam pengumpulan data dalam lingkungan yang \textit{disconnected}, dan setelah device kembali ke \textit{connected environment}, data yang terkumpul akan terupload ke server. Akan tetapi \textit{CAM framework} memiliki kelemahan, antara lain : 1) CAM berbasis \textit{fix-length text-based input}, yang membuatnya tidak cocok digunakan untuk pengumpulan data yang mengandung banyak pertanyaan terbuka; 2) Tidak terdapat \textit{conflict resolution}, sehingga masih memungkinkan dua device atau lebih mengeksekusi data yang sama, 3) Tidak terdapat mekanisme pembaharuan aplikasi.

Sementara itu, Takdir dkk telah melalukan penelitian tentang penggunaan pola terdistribusi berbasis SOA ~\cite{takdir_multi-layer_2014}. Pada implementasi pola terdistribusi berbasis SOA, digunakan metode \textit{proxy} baik pada \textit{workflow} (\textit{Web service}) maupun \textit{data-service}. Mekanisme yang digunakan dalam perancangan service pola terdistribusi mencakup 3 (tiga) hal : sinkronisasi, replikasi, dan \textit{routing}. \textit{Composite application} yang dijalankan pada sisi \textit{client} akan melakukan replikasi data maupun \textit{Web service}, kemudian data dan \textit{Web service} tersebut digunakan secara lokal. Sementara itu, untuk menjamin konsistensi data, Takdir dkk mengakomodir mekanisme \textit{sinkronisasi}.

\begin{figure}[h]
    \centering
    \includegraphics[width=13cm]{../../Resources/Images/takdir-deployment-3a}
    \caption{Skema Usulan, Takdir}
    \label{fig:takdir-soa}
\end{figure}

Berdasarkan hasil pengujian, pola distribusi berbasis SOA usulan Takdir dkk terbukti mampu memberikan hasil yang lebih baik, dengan penggunaan \textit{resource} CPU dan memory yang lebih rendah. Akan tetapi, lingkup perancangan dan pengujian sistem hanya terbatas pada perangkat komputer (\textit{desktop} maupun \textit{laptop}). Pada pola terdistribusi, usulan Takdir dkk, \textit{client} perlu untuk melakukan replikasi data maupun \textit{workflow} yang berupa \textit{Web service} agar dapat berjalan. 

Penelitian ini akan berfokus kepada perancangan metode pengumpulan data berbasis \textit{mobile} yang mendukung \textit{connected environment} maupun \textit{disconnected environment} dengan mengadaptasi mekanisme replikasi, sinkronisasi, dan routing.


\section{Rumusan Masalah}
Berdasarkan uraian latar belakang permasalahan diatas, maka dapat dirumuskan suatu permasalahan penelitian yaitu merancang metode pengumpulan data berbasis \textit{mobile} yang mendukung \textit{connected environment} maupun \textit{disconnected environment}.

\section{Tujuan Penelitian}
Tujuan utama dari penelitian ini adalah menghasilkan rancangan metode pengumpulan data berbasis \textit{mobile} yang mendukung \textit{connected environment} maupun \textit{disconnected environment}. Adapun tujuan khusus penelitian ini adalah :
\begin{itemize}
\item Menghasilkan rancangan mekanisme replikasi data dan \textit{rule} validasi,
\item Menghasilkan rancangan mekanisme sinkronisasi data dan \textit{rule} validasi,
\item Menghasilkan rancangan mekanisme \textit{routing}.
\end{itemize}

\section{Batasan Masalah}
Batasan masalah dalam penelitian ini adalah :

\begin{itemize}
\item Penelitian ini hanya berfokus pada desain dan implementasi sistem pada \textit{mobile device},
\item \textit{Mobile device} yang digunakan dalam ujicoba terbatas hanya \textit{mobile device} dengan sistem operasi \textit{Android}.
\end{itemize}


\section{Batasan Masalah}

Sistematika penulisan tesis ini terdiri atas enam bab dengan perincian sebagai berikut:

\begin{itemize}
\item Pendahuluan\\
Bab ini memuat latar belakang dan motivasi dalam penyusunan penelitian tesis ini serta permasalahan yang ada. Termasuk juga di dalamnya perumusan masalah, tujuan penelitian, manfaat penelitian, batasan masalah, dan keluaran penelitian.

\item Tinjauan Pustaka\\
Bab ini memuat kajian teori dalam penelitian ini. Mencakup berbagai dasar teori yang menjadi landasan dalam penelitian, serta pengetahuan dan informasi tentang penelitian yang pernah dilakukan terkait dengan topik penelitian ini.

\item Metodologi Riset\\
Bab ini memuat tahapan pelaksanaan penlitian mulai dari identifikasi problem sampai urutan penyelesaiannya.

\item Analisis dan Perancangan\\
Bab ini memuat analisis yang dilakukan terhadap masalah yang dihadapi, serta kriteria yang dibutuhkan pada desain yang dirancang. Selain itu, juga diuraikan mengenai rancangan/desain yang dibuat yang juga merupakan solusi yang ditawarkan.

\item Implementasi dan Pengujian\\
Bab ini memuat proses dan hasil implementasi desain yang dirancang. Proses pengujian yang telah dilakukan juga ditampilkan, mulai dari data dan variabel pengukuran yang digunakan, skenario pengujian, hingga hasil pengujian.

\item Kesimpulan dan Saran\\
Bab ini memuat kesimpulan yang diperoleh dari penelitian ini, serta saran untuk pihak terkait dan untuk penelitian selanjutnya yang berkaitan dengan topik penelitian ini.
\end{itemize}


%-----------------------------------------------------------------------------%
\chapter{\babDua}
%-----------------------------------------------------------------------------%
\todo{tambahkan kata-kata pengantar bab 2 disini}




%-----------------------------------------------------------------------------%
\chapter{\babTiga}
%-----------------------------------------------------------------------------%
%\todo{tambahkan kata-kata pengantar bab 1 disini}


Metodologi penelitian yang akan dilakukan di dalam penelitian ini adalah \textit{Design Science Research Methods and Patterns} \citep{vaishnavi_design_2007}. Metodologi penelitian terdiri dari 5 (lima) tahapan seperti digambarkan pada Gambar \ref{fig:design-science-research-methodology} yaitu: \textit{Awareness of Problem}, \textit{Suggestion}, \textit{Development}, \textit{Evaluation}, dan \textit{Conclusion}.


\begin{figure}[h]
	\centering
	\includegraphics[width=3cm]{Resources/Images/design-science-research-methodology}
	\caption{Tahapan \textit{Design Science Research Methods and Patterns}}
	\label{fig:design-science-research-methodology}
\end{figure}


%-----------------------------------------------------------------------------%
\section{\textit{Awareness of Problem}}
%-----------------------------------------------------------------------------%
Langkah pertama dalam \textit{Design Science Research Methods and Patterns} adalah \textit{awareness of problem}. Langkah ini merupakan proses identifikasi dan definisi masalah. Permasalahan yang diidentifikasi dapat tersusun dari permasalahan nyata (\textit{real problem}) maupun permasalahan penelitian (\textit{research problem}). \textit{Real problem} diidentifikasi melalui analisis dan wawancara dengan \textit{subject matter}. Sementara \textit{research problem} diidentifikasi dari riset-riset yang meneliti permasalahan yang terkait dengan \textit{real problem}.


%-----------------------------------------------------------------------------%
\section{\textit{Suggestion}}
%-----------------------------------------------------------------------------%
Setelah masalah terdefinisikan, maka tahapan berikutnya adalah menemukan solusi dari masalah tersebut. Serangkaian analisis dan \textit{preliminary research} dilakukan untuk menemukan kandidat solusi dari permasalahan. Pada tahapan ini, akan diperoleh keluaran berupa \textit{tentative design} atau \textit{design overview}.


%-----------------------------------------------------------------------------%
\section{\textit{Development}}
%-----------------------------------------------------------------------------%
\textit{Tentative design} yang diperoleh pada tahap sebelumnya kemudian dikembangkan dan diimplementasikan pada tahap ini. Elaborasi dari \textit{tentative design} memerlukan kreatifitas. Complete design diimplementasikan dalam sejumlah bahasa pemrograman yang bervariasi, kemudian dikombinasikan dengan sejumlah software untuk membentuk sebuah prototype sistem. Sebuah mekanisme komunikasi juga dipilih pada tahap ini untuk mendukung implementasi dari sistem.


%-----------------------------------------------------------------------------%
\section{\textit{Evaluation}}
%-----------------------------------------------------------------------------%
Prototype sistem yang diimplementasikan pada tahap sebelumnya kemudian diuji dengan menggunakan serangkaian skenario. Sejumlah data juga digunakan pada tahapan ini. Hasil pengujian kemudian direpresentasikan dalam tabel dan grafik untuk memudahkan dalam menarik kesimpulan.


%-----------------------------------------------------------------------------%
\section{\textit{Conclusion}}
%-----------------------------------------------------------------------------%
Kesimpulan merupakan tahapan akhir dari penelitian. Kesimpulan yang diambil merupakan harus dapat menjawab masalah yang didefinisikan.






 \chapter{Analisis dan Perancangan}


\section{Analisis}

Pada kondisi saat ini, alokasi petugas pengumpulan data di BPS dilakukan pada saat perancangan. Sebelum proses pengumpulan data dilakukan, dan lokasi pencacahan telah diketahui (berdasarkan metode sampling yang digunakan), petugas kemudian direkrut dan dialokasikan kepada lokasi pencacahan terdekat. Pengalokasian lokasi pencacahan terhadap masing-masing pencacah, seringkali dilakukan secara subyektif, sehingga menyebabkan variasi waktu penyelesaian tinggi. 


Metode MTSP dan berbagai variasinya dapat digunakan untuk mengatasi pengalokasian diatas. Variasi MTSP yang dapat digunakan dalam permasalahan ini antara lain: 1) Menggunakan lebih dari satu pencacah, dimana masing-masing pencacah memulai dari \textit{depot} yang berbeda-beda; 2) Penimbang dari setiap lokasi pencacahan dengan lokasi lainnya direpresentasikan dengan waktu tempuh. Waktu tempuh lebih representatif untuk digunakan sebagai penimbang dibanding jarak, karena jarak tidak memperhitungkan kesulitan akses. 


Untuk menggambarkan bagaimana MTSP dapat digunakan untuk mengatasi masalah ini, maka perlu dilakukan eksperimen. Eksperimen dilakukan dengan langkah-langkah seperti berikut :


\subsection{Dataset}

\subsubsection{Lokasi Pencacahan}

Lokasi pencacahan yang digunakan merupakan data \textit{real} (bukan dummy), yang merupakan lokasi nagari/kelurahan di Kab. Pesisir Selatan, Provinsi Sumatera Barat. Data lokasi pencacahan yang digunakan dalam eksperimen ini sebanyak 182 lokasi. Contoh data lokasi pencacahan dapat dilihat pada Tabel. \ref{tbl:enumeration_locations}, sementara data lokasi pencacahan secara lengkap dapat dilihat pada Tabel. \ref{tbl:enumeration_locations_full}.


\begin{table*}\centering
\ra{1.3}
\caption{Lokasi Pencacahan}
\label{tbl:enumeration_locations}
\begin{tabular}{lcc}
\toprule
& \multicolumn{2}{c}{Koordinat}\\
\cmidrule{2-3}
& Latitude & Longitude\\ 
\midrule
1302011001 & -2.3504 & 101.1434\\ 
1302011002 & -2.4233 & 101.0285\\ 
1302011003 & -2.3798 & 101.0427\\ 
1302011004 & -2.3884 & 101.049\\ 
1302011005 & -2.3936 & 101.0546\\
...\\
1302110019 & -1.2387 & 100.4853\\ 
1302110020 & -1.1408 & 100.4938\\ 
1302110021 & -1.0883 & 100.4652\\ 
1302110022 & -1.0886 & 100.489\\ 
1302110023 & -1.1523 & 100.4978\\
\bottomrule
\end{tabular}
\end{table*}


\subsubsection{Pencacah}

Pada konsep MTSP, pencacah berperan sebagai \textit{salesman} yang harus berpindah dari satu lokasi ke lokasi lain secara berurutan. Pencacah juga diidentifikasi dengan \textit{depot}, yaitu lokasi dimana pencacah harus memulai dan mengakhiri. Dalam eksperimen ini digunakan 15 pencacah dengan lokasi \textit{depot} yang bervariasi. Tabel \ref{tbl:enumerator} adalah contoh data pencacah beserta lokasi \textit{depot}-nya yang digunakan dalam eksperimen ini, sementara data lengkap dapat dilihat pada Tabel \ref{tbl:enumerator_full}.


\begin{table*}\centering
\ra{1.3}
\caption{Pencacah}
\label{tbl:enumerator}
\begin{tabular}{lcc}
\toprule
& \multicolumn{2}{c}{Koordinat Depot}\\
\cmidrule{2-3}
& Latitude & Longitude\\ 
\midrule
1302011008 & -2.3905 & 101.1214\\
1302012003 & -2.199 & 101.1188\\
1302020006 & -2.1225 & 101.0687\\
...\\
1302100002 & -1.23265 & 100.54314\\
1302101005 & -1.19831 & 100.58078\\
1302110003 & -1.2475 & 100.4745\\
\bottomrule
\end{tabular}
\end{table*}


\subsubsection{Jarak dan Waktu Tempuh}

Jarak dan waktu tempuh antar node digunakan sebagai penimbang dalam penentuan rekomendasi lokasi. Penghitungan jarak dan waktu tempuh, selain dapat dilakukan secara manual (berdasarkan hasil survei atau perkiraan \textit{subject matter}), dapat juga didekati dengan menggunakan Google Directions API \citep{google_google_2016}. Keuntungan dengan menggunakan Google Direction API adalah jarak dan waktu yang dihitung telah memperhitungkan rute tercepat, kondisi geografis, kemacetan lalu-lintas, dan moda yang digunakan. Adapun kekurangan dari Google Direction API adalah jumlah \textit{requests} yang dapat dikirimkah sangat terbatas, hanya 1.000 \textit{requests} per hari per akun, sementara jumlah waktu dan jarak yang harus dihitung sejumlah $ nx(n-1)/2 = 182x187/2 = 16.471 $. Listing \ref{lst:google_direction_api_request} memuat contoh \textit{requests} dengan Google Direction API, dan Gambar \ref{fig:google_direction_api_response} adalah contoh responnya.


\begin{listing}
    \caption{Google Direction API Request}
    \label{lst:google_direction_api_request}
%    \begin{minted}[showspaces=false, breaklines=true, escapeinside=||]{http}
%https://maps.googleapis.com/maps/api/directions/json?origin=originx,originy&destination=destx,desty&departure_time=timestamp&traffic_model=best_guess&key=APIKEY
%	\end{minted}
\end{listing}


\begin{figure}[h]
    \centering
    \includegraphics[width=\textwidth]{../../Resources/Images/google_direction_api_response}
    \caption{Google Direction API Response}
    \label{fig:google_direction_api_response}
\end{figure}


Contoh matrix jarak dan waktu tempuh dapat dilihat pada Tabel. \ref{tbl:distance_duration_matrix}, sementara data secara lengkap dapat dilihat pada Tabel. \ref{tbl:distance_duration_matrix_full}. Sebagai catatan, jika lokasi \textit{depot} dari pencacah bukan salah satu dari lokasi pencacahan, maka lokasi \textit{depot} juga harus disertakan dalam matriks jarak dan waktu tempuh.


\begin{table*}\centering
\ra{1.3}
\caption{Matriks Jarak dan Waktu Tempuh}
\label{tbl:distance_duration_matrix}
\begin{tabular}{llcc}
\toprule
Lokasi A & Lokasi B & Jarak & Waktu Tempuh\\
\midrule
1302021001 & 1302021003 & 11119 & 1055\\
1302021001 & 1302021002 & 9373 & 868\\
1302021001 & 1302021005 & 490 & 38\\
1302021001 & 1302021004 & 22760 & 2044\\
1302021001 & 1302021007 & 10228 & 950\\
...\\
1302040015 & 1302100015 & 99719 & 9682\\
1302040015 & 1302012010 & 77889 & 8305\\
1302040014 & 1302100015 & 103893 & 9984\\
1302040014 & 1302012010 & 73561 & 7546\\
1302100015 & 1302012010 & 171636 & 16801\\
\bottomrule
\end{tabular}
\end{table*}


\subsection{\textit{Library} dan Implementasi}

Pada eksperimen ini digunakan jsprit \citep{jsprit_jsprit_2014}, sebuah library berbasis java yang dapat digunakan untuk menyelesaikan permasalahan traveling salesman (TSP) dan vehicle routing problems (VRP). Jsprit mencakup berbagai skenario, antara lain : \textit{pickups and deliveries}, \textit{back hauls}, \textit{heterogeneous fleets}, \textit{finite and infinite fleets}, \textit{multiple depots}, \textit{time windows}, \textit{open routes}, \textit{different start and end locations}, \textit{multiple capacity dimensions}, \textit{initial loads}, \textit{skills}, dll. Cara kerja jsprit sangat terstruktur, mulai dari pendefinisan masalah, pemilihan algoritma, pencarian solusi, dan terakhir pemilihan solusi terbaik.


\subsubsection{Definisi Masalah}

\begin{listing}
    \caption{Definisi Pencacah dari File}
    \label{lst:jsprit_define_enumerators}
	\inputminted[showspaces=false, breaklines=true]{java}{../../Resources/Snippets/jsprit_enumerator.java}    
    
%    \begin{minted}[showspaces=false,breaklines=true]{java}
%VehicleTypeImpl.Builder vehicleTypeBuilder = VehicleTypeImpl.Builder.newInstance("enumerator");
%vehicleTypeBuilder.setCostPerDistance(0);
%vehicleTypeBuilder.setCostPerTransportTime(1);
%vehicleTypeBuilder.setCostPerServiceTime(1);
%VehicleType vehicleType = vehicleTypeBuilder.build();
%
%try {
%    CSVReader reader = new CSVReader(new FileReader(csvEnum));
%    reader.readNext();
%
%    String [] line;
%    while ((line = reader.readNext()) != null) {
%        VehicleImpl.Builder builder = VehicleImpl.Builder.newInstance(line[0]);
%
%        try {
%            Location loc = Location.Builder.newInstance()
%                    .setId(line[0])
%                    .setCoordinate(Coordinate.newInstance(Double.parseDouble(line[2]),
%                            Double.parseDouble(line[1])))
%                    .build();
%            builder.setStartLocation(loc);
%        } catch (Exception e) {}
%
%
%        builder.setType(vehicleType);
%        VehicleImpl vehicle = builder.build();
%        vrpBuilder.addVehicle(vehicle);
%    }
%
%} catch (FileNotFoundException e) {
%    e.printStackTrace();
%} catch (IOException e) {
%    e.printStackTrace();
%}
%	\end{minted}
\end{listing}


\section{Garis Besar Perancangan}

Alur kerja perancangan dimulai dengan dengan mengidentifikasi blok sensus yang akan dilakukan pendataan padanya, serta menentukan jumlah pencacah yang akan digunakan. Kedua permasalahan ini tidak akan dibahas terlalu mendalam dalam penelitian ini. Lokasi pencacahan telah ditentukan dalam fase perancangan sensus dan survei, mengikuti sebuah metodologi tertentu. Sementara jumlah pencacahan juga telah ditentukan, mengikuti jumlah sampel dan berbagai persyaratan tertentu, seperti waktu dan biaya.


Selanjutnya, setiap pencacah akan dialokasikan kepada blok sensus yang akan dicacah dengan menggunakan metode MTSP, sebagaimana diformulasikan oleh \citep{bektas_multiple_2006}, dengan ketentuan setiap pencacah dapat memulai dan mengakhiri pada \textit{depot} yang berbeda-beda. Setelah model diperoleh, setiap pencacah akan mengunjungi lokasi pertama dari rekomendasi. Setelah selesai kunjungan, lokasi akan disimpan dalam \textit{tabu list} dengan menggunakan metode pub-sub \citep{chen_efficient_2003}. Model baru akan digenerate setiap kali terdapat \textit{request} dari salah satu pencacah. Setiap kali model di-\textit{generate}, \textit{tabu search} \citep{glover_tabu_1989, glover_tabu_1990} yang memanfaatkan \textit{tabu list} akan digunakan untuk memastikan tidak terdapat \textit{conflict}. Setelah model baru selesai di-\textit{generate}, maka mudel akan di-\textit{publish} kepada setiap pencacah yang telah men-\textit{subscribe}.


\begin{figure}[h]
    \centering
    \includegraphics[width=\textwidth]{../../Resources/Images/design_overview}
    \caption{Garis Besar Sistem Usulan}
    \label{fig:design_overview}
\end{figure}


\section{Penyusunan Rekomendasi}

Pada tahap perumusan rekomendasi, input data yang terdiri dari data pencacah dan data blok sensus akan dioleh menjadi rekomendasi path yang harus dikunjungi. Proses penyusunan rekomendasi menggunakan metode \textit{Multiple Travelling Salesman Problem} (MTSP), yang merupakan pengembangan dari metode klasik \textit{Travelling Salesman Problem} (TSP).


Metode MTSP yang digunakan dalam masalah ini memiliki beberapa \textit{requirements}, antara lain:

\begin{itemize}
\item Jumlah \textit{depot} \\
MTSP dapat menggunakan lebih dari satu depot, dengan $ m_{j} $ \textit{salesman} untuk setiap depot $ j $. Pada permasalahan ini menggunakan \textit{non-fixed destination}, sehingga pencacahan tidak perlu kembali ke lokasi dimana pencacahan dimulai.
\item Jumlah \textit{salesman} \\
Jumlah \textit{salesman} yang digunakan dapat berupa \textit{fixed number} $ m $, atau dinamis dengan dibatasi jumlah maksimal $ max(m) $. Pada permasalahan ini digunakan \textit{fixed number} $ m $ pencacah.
\item \textit{Fixed charges} \\
Jika jumlah \textit{salesman} dinamis, maka bisa juga masing-masing \textit{salesman} dibatasi dengan sejumlah biaya tertentu. Pada permasalahan ini tidak digunakan \textit{fixed charges}.
\item Waktu kunjungan (\textit{time windows}) \\
\textit{Time windows} merepresentasikan waktu yang dihabiskan selama kunjugan dalam sebuah \textit{node}. Pada kasus ini \textit{time windows} tidak dapat ditentukan karena tidak tersedianya informasi, sehingga dianggap tidak menggunakan \textit{time windows}.
\end{itemize}


Requirements di atas, secara global dapat disederhanakan dalam tabel-tebel berikut. Tabel \ref{tbl:enumerators_overview} menunjukkan rancangan pencacah beserta koordinat \textit{depot}-nya, sementara Tabel \ref{tbl:census_blocks} menunjukkan rancangan blok sensus beserta koordinat dan \textit{time windows}-nya. Dalam fakta lapangan, jarak antara satu blok sensus dengan blok sensus yang lain tidaklah setara. Bisa jadi secara koordinat memiliki jarak yang berdekatan, tetapi secara akses tidaklah mudah. Untuk itu diperlukan sebuah tabel tambahan, yaitu tabel \textit{cost-matrix}, sebagaimana Tabel \ref{tbl:cost_matrix}. \textit{Cost} yang dimaksud disini adalah segala metrik yang dapat digunakan sebagai penimbang (\textit{weight}), misalnya: biaya, jarak, atau waktu tempuh.


\begin{table}[]
\centering
\caption{Table Pencacah}
\label{tbl:enumerators_overview}
\begin{tabular}{@{}lcc@{}}
\toprule
\multirow{2}{*}{Pencacah} & \multicolumn{2}{l}{\textit{Depot Coordinate}} \\ \cmidrule(l){2-3} 
                          & X                 & Y                \\ \midrule
Pencacah 1                & 20.0              & 20.0             \\
Pencacah 2                & 20.0              & 20.0             \\
Pencacah 3                & 30.0              & 40.0             \\
Pencacah 4                & 30.0              & 40.0             \\
...                       &                   &                  \\
Pencacah m                & x                 & y                \\ \bottomrule
\end{tabular}
\end{table}


\begin{table}[]
\centering
\caption{Tabel Blok Sensus}
\label{tbl:census_blocks}
\begin{tabular}{@{}lccc@{}}
\toprule
\multirow{2}{*}{Blok Sensus} & \multicolumn{2}{c}{Koordinat Lokasi} & \multirow{2}{*}{Time Windows} \\ \cmidrule(lr){2-3}
                             & X                 & Y                &                               \\ \midrule
001B                         & 62.0              & 63.0             & 0                             \\
002B                         & 63.0              & 69.0             & 0                             \\
003B                         & 46.0              & 10.0             & 0                             \\
004B                         & 61.0              & 33.0             & 0                             \\
...                          &                   &                  &                               \\
n                            & x                 & y                & 0                             \\ \bottomrule
\end{tabular}
\end{table}


\begin{table}[]
\centering
\caption{Table \textit{Cost-Matrix}}
\label{tbl:cost_matrix}
\begin{tabular}{@{}|c|c|c|c|c|c|c|@{}}
\toprule
        & 001B & 002B & 003B & 004B & ... & BS ke-n \\ \midrule
001B    & -    & 5    & 2    & 2    &     & ...     \\ \midrule
002B    &      & -    & 4    & 2    &     & ...     \\ \midrule
003B    &      &      & -    & 7    &     & ...     \\ \midrule
004B    &      &      &      & -    &     & ...     \\ \midrule
...     &      &      &      &      & -   & ...     \\ \midrule
BS ke-n &      &      &      &      &     & -       \\ \bottomrule
\end{tabular}
\end{table}


Tabel \textit{cost-matrix}, selain dapat didefinisikan secara manual (berdasarkan hasil survei atau perkiraan \textit{subject matter}), dapat juga didekati dengan menggunakan Google Directions API \citep{google_google_2016}. \textit{Request} yang digunakan menggunakan standar REST API, sementara \textit{response} yang ditampilkan dalam format JSON. Listing \ref{lst:google_direction_api_request} menunjukkan contoh \textit{request}, dan Gambar \ref{fig:google_direction_api_response} menunjukkan contoh \textit{response} dari Google Direction API.








%Gambar \ref{fig:mtsp_solution_example} berikut menunjukkan hasil rekomendasi dengan MTSP.
%
%
%\begin{figure}[h]
%    \centering
%    \includegraphics[width=\textwidth]{../../Resources/Images/mtsp_solution_example}
%    \caption{Contoh Hasil Rekomendasi}
%    \label{fig:mtsp_solution_example}
%\end{figure}


\section{Penyusunan \textit{Conflict Resolution}}


\section{\textit{Publish-Subscribe} Rekomendasi}

%-----------------------------------------------------------------------------%
\chapter{\babLima}
%-----------------------------------------------------------------------------%


%-----------------------------------------------------------------------------%
\section{Implementasi}
\label{sec:implementation}
%-----------------------------------------------------------------------------%
Algoritma yang disusun pada Bab \ref{sec:design} diimplementasikan dalam bahasa pemrograman. Hasil implementasi kemudian dikombinasikan dengan aplikasi dan \textit{library} pihak ketiga untuk membentuk sebuah \textit{prototype}.


%-----------------------------------------------------------------------------%
\subsection{\textit{Message Broker}}
%-----------------------------------------------------------------------------%
Pada \textit{prototype} yang disusun, \textit{message broker} diimplementasikan dengan menggunakan Redis. Redis merupakan \textit{in-memory data structure store} yang dapat digunakan sebagai \textit{database}, \textit{cache}, dan \textit{message broker} \citep{redis_introduction_2017}. Redis mempunyai \textit{feature} Redis Cluster, sehingga dapat dengan mudah disusun menjadi \textit{distributed system}. Selain itu, Redis Client tersedia dalam hampir semua bahasa pemrograman \textit{mainstream}, sehingga \textit{client} dari \textit{prototype} ini dapat dengan mudah dibuat dalam berbagai versi \citep{redis_clients_2017}.


%-----------------------------------------------------------------------------%
\subsection{\textit{TSP Solver}}
%-----------------------------------------------------------------------------%
TSPSolver, yang prototype ini menggunakan algoritma \textit{Cooperative Evolution Strategy} (CoES), diimplementasikan dalam bahasa pemrograman C++. Pemilihan bahasa C++ pada implementasi ini karena permasalahan kombinatorial pada TSP menggunakan resources (prosesor dan memory) yang intensive, sehingga penggunakan C++ dianggap lebih efisien dari bahasa pemrograman lainnya. Source code dari implementasi CoES dapat diunduh di \url{https://github.com/soedomoto/coes-mdvrp/tree/jni-coes-mdvrp}. Adapun lingkungan pengembangan Algoritma CoES adalah sebagai berikut:


\begin{itemize}
\item Sistem Operasi		: Elementary OS Loki (Berbasis Ubuntu 16.04)
\item C++ Compiler			: c++ (Ubuntu 5.4.0-6ubuntu1~16.04.4) 5.4.0 20160609
\item Hardware				: Asus TP300L, Quad-Core Intel® Core™ i3-4030U CPU @ 1.90GHz, 3,7 GiB DDRIII, 256GB SSD
\end{itemize}


%-----------------------------------------------------------------------------%
\subsection{\textit{Publisher}}
%-----------------------------------------------------------------------------%
Algoritma yang digunakan untuk publisher, sebagaimana Algoritma \ref{lst:proposed_solver_thread_algorithm} dan Algoritma \ref{lst:proposed_publisher_algorithm}, diimplementasikan dalam bahasa Java. Source code dari implementasi Publisher dapat diunduh di \url{https://github.com/soedomoto/coes-mdvrp/tree/master}. Adapun lingkungan pengembangan Algoritma Publisher adalah sebagai berikut:


\begin{itemize}
\item Sistem Operasi		: Elementary OS Loki (Berbasis Ubuntu 16.04)
\item Java version			: java version "1.8.0\_121"
\item Hardware				: Asus TP300L, Quad-Core Intel® Core™ i3-4030U CPU @ 1.90GHz, 3,7 GiB DDRIII, 256GB SSD
\end{itemize}


%-----------------------------------------------------------------------------%
\section{Pengujian}
\label{sec:testing}
%-----------------------------------------------------------------------------%
Pengujian program dilakukan untuk mengetahui tingkat tercapainya tujuan penelitian ini, yaitu bagaimana membuat rekomendasi lokasi pencacahan pada kondisi data \textit{time windows} tidak tersedia, dan bagaimana menyusun mekanisme \textit{conflict resolution}, agar system tidak merekomendasikan lokasi yang sama pada dua atau lebih pencacah. Ukuran yang akan digunakan dalam pengujian adalah \textit{total cost}, \textit{mean route cost}, dan \textit{route cost mean square error}. Untuk mengukur keakuratan algoritma, maka hasil pengujian akan dibandingkan dengan algoritma MTSP tanpa mekanisme publish/subscribe.


%-----------------------------------------------------------------------------%
\subsection{Lingkungan Pengujian}
%-----------------------------------------------------------------------------%
Pegujian akan dilakukan pada \textit{environment} dengan spesifikasi seperti berikut: 
\begin{itemize}
\item Sistem Operasi		: Elementary OS Loki (Berbasis Ubuntu 16.04)
\item Redis Environment		: Redis 3.2.6, Debian Jessie (Docker version)
\item JRE version			: Java(TM) SE Runtime Environment (build 1.8.0\_121-b13)
\item Database				: H2 Database 1.4.193 (Server mode)
\item Hardware				: Asus TP300L, Quad-Core Intel® Core™ i3-4030U CPU @ 1.90GHz, 3,7 GiB DDRIII, 256GB SSD
\end{itemize}


%-----------------------------------------------------------------------------%
\subsection{\textit{Dataset} dan \textit{Metric}}
%-----------------------------------------------------------------------------%
%-----------------------------------------------------------------------------%
\subsubsection{\textit{Dataset}}
%-----------------------------------------------------------------------------%
Untuk memastikan program dapat bekerja dengan baik, maka program harus diujicobakan dengan berbagai variasi data. Dataset yang paling banyak digunakan dalam pengujian kasus terkait \textit{Vehicle Routing Problem} adalah data Breedam, Cordeau, Solomon, Homberger, dan Russell. Dalam pengujian kali ini akan digunakan dataset Cordeau tipe 2 (Multi-Depot VRP). Dataset Cordeau sendiri tersedia dalam 7 (tujuh) tipe VRP \textit{problem}, antara lain: 

\begin{enumerate}
\item Tipe 0 untuk kasus VRP
\item Tipe 1 untuk kasus Periodic VRP
\item Tipe 2 untuk kasus Multi-Depot VRP
\item Tipe 3 untuk kasus Split Delivery VRP
\item Tipe 4 untuk kasus VRP dengan Time Windows
\item Tipe 5 untuk kasus Periodic VRP dengan Time Windows
\item Tipe 6 untuk kasus Multi-Depot VRP dengan Time Windows
\item Tipe 7 untuk kasus Split Delivery VRP dengan Time Windows
\end{enumerate}


Adapun penjelasan format dari dataset Cordeau tipe 2 adalah sebagai berikut:
\begin{enumerate}
\item Baris pertama berformat \textbf{TYPE M N T}, dimana: \\
M = Jumlah \textit{vehicle} \\
N = Jumlah \textit{customer} \\
T = Jumlah \textit{depot}

\item Baris kedua sampai T baris berikutnya berformat \textbf{D Q}, dimana: \\
D = Durasi maksimum dari setiap rute \\
Q = Kapasitas maksumum dari setiap \textit{vehicle}

\item Baris selanjutnya sampai M baris berikutnya berformat \textbf{i x y d q f a list e l}, dimana: \\
i	= nomer \textit{customer} \\
x	= koordinat x \\
y	= koordinat y \\
d	= durasi pelayanan (\textit{service time}) \\
q	= \textit{demand} \\
f	= frekuensi kunjungan \\
a	= jumlah kombinasi kunjungan \\
list	= list dari semua kombinasi kunjungan \\
e	= jika ada, waktu dimulainya kunjungan \\
l	= jika ada, waktu selesainya kunjungan

\item Baris selanjutnya sampai M baris berikutnya berformat \textbf{i x y}, dimana: \\
i	= nomer \textit{vehicle} \\
x	= koordinat depot x \\
y	= koordinat depot y \\
\end{enumerate}


Selain menggunakan data Cordeau tipe 2, program juga diujicobakan dengan menggunakan data lapangan. Data lapangan yang digunakan meliputi 182 \textit{customers} ($N$) beserta koordinatnya, 15 \textit{vehicle} ($M$) beserta koordinat depot masing-masing \textit{vehicle}. Dari seluruh \textit{customer} dan \textit{vehicle}, dihitung waktu tempuh dari seluruh kombinasi dengan memanfaatkan \textit{Google Direction API}, sebagaimana langkah pada Poin. \ref{ss:distance-duration-matrix}.


%-----------------------------------------------------------------------------%
\subsubsection{\textit{Metric}}
%-----------------------------------------------------------------------------%
Data yang akan digunakan, Cordeau type 2 dan data lapangan, akan diujicobakan dengan menggunakan program usulan, berbasis Pub/Sub dan CoES. Data kemudian diperbandingkan dengan program pembanding yang mengimplementasi algoritma CoES tanpa mekanisme Pub/Sub. Dari kedua program tersebut akan diperoleh output rute untuk masing-masing \textit{vehicle} dengan ilustrasi sebagi berikut:

\begin{itemize}
\item \textit{Vehicle} A = Loc1 -> Loc5 -> Loc15 -> Loc6
\item \textit{Vehicle} B = Loc 6 -> Loc2 -> Loc16 -> Loc3
\item \textit{Vehicle} C = Loc4 -> Loc8 -> Loc14 -> Loc 7
\item \textit{Vehicle} D = Loc9 -> Loc10 -> Loc11 -> Loc12
\end{itemize}


Masing-masing rute kemudian dihitung \textit{total cost}-nya, yang merupakan penjumlahan dari seluruh \textit{service time} pada masing-masing lokasi dan seluruh waktu tempuh dari masing-masing perpindahan. Metric yang akan dijadikan pembanding adalah \textit{mean square error} (MSE) seluruh rute pada masing-masing program dan program pembanding. Program yang lebih baik akan menghasilkan MSE yang lebih kecil. MSE dipilih sebagai \textit{metric} karena merepresentasikan kondisi sebenarnya dilapangan, dimana jika variasi waktu dari seluruh pencacah lebih kecil, maka penyelesaian pencacahan akan lebih merata.


%-----------------------------------------------------------------------------%
\subsection{Skenario dan Hasil Pengujian}
%-----------------------------------------------------------------------------%
Untuk memastikan program dapat bekerja dengan baik, selain menggunakan variasi data, juga akan digunakan berbagai skenario, yaitu:

%-----------------------------------------------------------------------------%
\subsubsection{Pengujian Kondisi Normal}
%-----------------------------------------------------------------------------%
Skenario pengujian kondisi normal dimaksudkan untuk membandingkan program yang dijalankan pada kondisi normal. Pengujian akan dilakukan satu kali untuk masing-masing data set. Pada pengujian ini sebuah program \textit{client} dibuat sebagai simulator.


\begin{figure}[h]
    \centering
    \includegraphics[width=\textwidth]{../../Resources/Images/client-algorithm-nornal-cordeau}
    \caption{\textit{Flowchart} pada \textit{Client Simulator} Kondisi Normal (Cordeau)}
    \label{fig:client-algorithm-nornal-cordeau}
\end{figure}


\begin{figure}[h]
    \centering
    \includegraphics[width=\textwidth]{../../Resources/Images/client-algorithm-nornal-field}
    \caption{\textit{Flowchart} pada \textit{Client Simulator} Kondisi Normal (Lapangan)}
    \label{fig:client-algorithm-nornal-field}
\end{figure}


Langkah-langkah yang diimplementasikan pada program \textit{client}, sebagaimana ilustrasi Gambar \ref{fig:client-algorithm-nornal-cordeau} dan Gambar \ref{fig:client-algorithm-nornal-field}, adalah sebagai berikut:

\begin{enumerate}
\item Buat koneksi dengan \textit{message broker}, 
\item Lakukan \textit{subscription} dengan topik \textit{current location}, 
\item Setelah sebuah \textit{message} diterima, dimana \textit{message} dari \textit{broker} merupakan \textit{next location}, kalkulasi waktu tempuh antara \textit{current location} dengan \textit{next location},
\item Simulasikan perjalanan ke lokasi dengan \textit{sleep},
\item Buat notifikasi terhadap broker atas perubahan \textit{current location}, 
\item Generate \textit{service time}, mengikuti distribusi normal, 
\item Simulasikan pencacahan dengan \textit{sleep}, 
\item Ulangi lagi dari langkah pertama.
\end{enumerate}


Setelah simulasi dilakukan dengan menggunakan dataset Cordeau, diperoleh rute final beserta total durasinya baik untuk algoritma CoES tanpa Pub/Sub maupun untuk algoritma CoES dengan mekanisme Pub/Sub, sebagaimana Tabel \ref{tbl:test_result_normal_cordeau_coes} dan \ref{tbl:test_result_normal_cordeau_pubsub_coes}. Dari hasil pengujian diperoleh perbandingan, sebagaimana Tabel \ref{tbl:test_result_normal_cordeau_comparison}, bahwasannya \textbf{total waktu} yang diperoleh dengan menggunakan CoES lebih kecil, akan tetapi dari sisi \textbf{standar deviasi} dari keseluruhan rute diperoleh hasil metode CoES yang dikombinasikan dengan Pub/Sub menghasilkan angka yang lebih kecil, yang artinya total waktu dari seluruh rute lebih merata.


\begin{longtable}[h]{lp{8cm}r}
	\caption{Hasil Pengujian Kondisi Normal (Cordeau), CoES}
	\label{tbl:test_result_normal_cordeau_coes}\\
	\toprule
		\textit{Vehicle} & Rute & Waktu Total\\ 
	\midrule
	\endfirsthead
	\toprule
		\textit{Vehicle} & Rute & Waktu Total\\ 
	\midrule
	\endhead
	\bottomrule
	\endfoot
		54 & 54-20-3-36-35-29-21 & 135578.58 \\
		51 & 51-42-44-15-37-4-18-17-14-25-13-41-40-19 & 293122.68 \\
		53 & 53-9-50-34-30-10-39-33-45-5-38-11-2-16-49 & 333166.34 \\
		52 & 52-46-12-47-48-24-43-23-7-26-8-31-28-22-1-32-6-27 & 403645.69 \\
\end{longtable}


\begin{longtable}[h]{lp{8cm}r}
	\caption{Hasil Pengujian Kondisi Normal (Cordeau), Pub/Sub CoES}
	\label{tbl:test_result_normal_cordeau_pubsub_coes}\\
	\toprule
		\textit{Vehicle} & Rute & Waktu Total\\ 
	\midrule
	\endfirsthead
	\toprule
		\textit{Vehicle} & Rute & Waktu Total\\ 
	\midrule
	\endhead
	\bottomrule
	\endfoot
		51 & 51-42-19-18-25-14-41-40-13-47-4-12-17-16 & 295641.36 \\
		52 & 52-46-27-24-6-7-23-48-43-8-26-1-31-32 & 310764.56 \\
		53 & 53-49-5-44-37-15-38-11-29-21-34-50-22 & 277819.28 \\
		54 & 54-20-3-28-36-2-35-9-30-10-39-33-45 & 281481.64 \\
\end{longtable}


\begin{longtable}[h]{lrr}
	\caption{Perbandingan Kondisi Normal (Cordeau)}
	\label{tbl:test_result_normal_cordeau_comparison}\\
	\toprule
		\textit{Vehicle} & Rute & Waktu Total\\ 
	\midrule
	\endfirsthead
	\toprule
		\textit{Vehicle} & Rute & Waktu Total\\ 
	\midrule
	\endhead
	\bottomrule
	\endfoot
		Total Waktu & 1165513.28 & 1165706.84\\
		Rata-rata & 291378.32 & 291426.71\\
		Standar Deviasi & 98268.51 & 12997.91\\
\end{longtable}


Sementara itu, pengujian dengan menggunakan data lapangan, rute yang diperoleh sebagaimana Tabel \ref{tbl:test_result_normal_field_coes} untuk algoritma CoES, dan Tabel \ref{tbl:test_result_normal_field_pubsub_coes} untuk algoritma CoES dengan mekanisme Pub/Sub. Dari hasil pengujian, sebagaimana Tabel \ref{tbl:test_result_normal_field_comparison}, menghasilkan kesimpulan yang tidak jauh berbeda dengan Tabel \ref{tbl:test_result_normal_cordeau_comparison}, dimana algoritma CoES menghasilkan \textbf{waktu total} yang lebih kecil dibandingkan algoritma CoES yang dikombinasikan dengan Pub/Sub, tetapi dari sisi \textbf{standar deviasi} lebih besar.


\begin{longtable}[h]{lp{8cm}r}
	\caption{Hasil Pengujian Kondisi Normal (Lapangan), CoES}
	\label{tbl:test_result_normal_field_coes}\\
	\toprule
		\textit{Vehicle} & Rute & Waktu Total\\ 
	\midrule
	\endfirsthead
	\toprule
		\textit{Vehicle} & Rute & Waktu Total\\ 
	\midrule
	\endhead
	\bottomrule
	\endfoot
		1302021005 & 1302021005-1302021005-1302020010-1302020001-1302020009-1302020011-1302021001 & 143455.33 \\
		1302090004 & 1302090004-1302090008-1302090002-1302090007-1302090005-1302090001-1302090015-1302090009-1302090014-1302090003-1302090018-1302090016-1302090017-1302090013-1302090012-1302090006-1302090004-1302090019-1302090020 & 429950.94 \\
		1302011008 & 1302011008-1302011008-1302011010-1302011001-1302012005-1302011007-1302011009-1302011004-1302011003-1302011002-1302011006-1302011005 & 264181.29 \\
		1302031005 & 1302031005-1302031001-1302031006-1302031008-1302040011-1302031009-1302031010-1302031007-1302030001-1302031002-1302031005 & 246533.56 \\
		1302050007 & 1302050007-1302050007-1302050008-1302050009-1302050010-1302050002-1302050006 & 146835.04 \\
		1302100002 & 1302100002-1302100002-1302100009-1302100008-1302100007-1302100006-1302100001-1302100017-1302100016-1302090011-1302090010-1302100004-1302100015-1302100003-1302100013-1302100012 & 377098.71 \\
		1302030005 & 1302030005-1302030005-1302030014-1302030006-1302030009-1302030004-1302030012-1302030003-1302031004-1302031003-1302030002-1302030010 & 261871.21 \\
		1302012003 & 1302012003-1302012008-1302012003-1302012001-1302012006-1302012002-1302012010-1302012009-1302012004-1302012007 & 218170.80 \\
		1302080006 & 1302080006-1302080006-1302080004-1302080009-1302080003-1302080005-1302080001-1302080008-1302080002-1302080007 & 226083.58 \\
		1302070006 & 1302070006-1302070006-1302070002-1302070011-1302070003-1302070012-1302070005-1302070007-1302070008-1302070009-1302070010-1302070004-1302070001 & 299255.93 \\
		1302060005 & 1302060005-1302060005-1302060009-1302060006-1302060008-1302060002-1302060007-1302060001-1302060003-1302060004 & 215375.02 \\
		1302020006 & 1302020006-1302020006-1302020017-1302020016-1302020015-1302021002-1302021008-1302021003-1302021009-1302021010-1302021004-1302021007-1302021006-1302020005-1302020003 & 347975.49 \\
		1302110003 & 1302110003-1302110019-1302110018-1302110012-1302110023-1302110006-1302110021-1302110008-1302110007-1302110022-1302110009-1302110020-1302110005-1302110004-1302110010-1302100005-1302110001-1302110013-1302110002-1302110014-1302110015-1302110017-1302110016-1302110011-1302110003 & 600827.54 \\
		1302040002 & 1302040002-1302040004-1302040003-1302040015-1302040001-1302040008-1302040009-1302040010-1302040012-1302040014-1302040013-1302040002-1302040016-1302050003-1302050005-1302050004-1302050001-1302040005-1302040006-1302040007 & 445481.39 \\
		1302101005 & 1302101005-1302101005-1302100010-1302101002-1302101003-1302101006-1302101004-1302100011-1302100014-1302101001 & 225893.85 \\
\end{longtable}


\begin{longtable}[h]{lp{8cm}r}
	\caption{Hasil Pengujian Kondisi Normal (Lapangan), Pub/Sub CoES}
	\label{tbl:test_result_normal_field_pubsub_coes}\\
	\toprule
		\textit{Vehicle} & Rute & Waktu Total\\ 
	\midrule
	\endfirsthead
	\toprule
		\textit{Vehicle} & Rute & Waktu Total\\ 
	\midrule
	\endhead
	\bottomrule
	\endfoot
		1302011008 & 1302011008-1302011008-1302011010-1302011001-1302012005-1302011007-1302011009-1302011005-1302011004-1302011006-1302011002-1302110008 & 288136.01 \\
		1302040002 & 1302040002-1302040004-1302040002-1302040003-1302040015-1302040001-1302040008-1302040009-1302040010-1302040016-1302040013-1302040012 & 260428.46 \\
		1302012003 & 1302012003-1302012008-1302012001-1302012006-1302012002-1302012004-1302012010-1302012009-1302020001-1302011003-1302040014-1302110022 & 289894.14 \\
		1302060005 & 1302060005-1302060005-1302060009-1302060006-1302060008-1302060002-1302060007-1302060001-1302060004-1302060003-1302040007-1302040005 & 263996.44 \\
		1302020006 & 1302020006-1302020006-1302020016-1302020017-1302020015-1302021002-1302021006-1302020005-1302020003-1302021007-1302021003-1302021009 & 271963.26 \\
		1302080006 & 1302080006-1302080006-1302080009-1302080005-1302080002-1302080008-1302080007-1302080001-1302080003-1302070008-1302070004-1302100005-1302110014-1302110013-1302110009 & 357607.07 \\
		1302021005 & 1302021005-1302021005-1302021001-1302012003-1302012007-1302020010-1302020011-1302020009-1302021010-1302021004-1302021008-1302110017-1302110002 & 309232.92 \\
		1302100002 & 1302100002-1302100002-1302100003-1302100009-1302100006-1302100008-1302100013-1302101001-1302100012-1302100004-1302100015-1302100016-1302090010-1302090011-1302100001-1302100017 & 382894.06 \\
		1302101005 & 1302101005-1302101005-1302100010-1302100014-1302100011-1302101004-1302101006-1302101003-1302101002-1302100007-1302090005-1302110015-1302110011 & 301312.52 \\
		1302110003 & 1302110003-1302110019-1302110004-1302110018-1302110001-1302110010-1302110020-1302110006-1302110023-1302110012-1302110005-1302110003-1302110016 & 294736.00 \\
		1302090004 & 1302090004-1302090008-1302090001-1302090015-1302090013-1302090012-1302090006-1302090002-1302090007-1302090009-1302090003-1302090018-1302090016-1302090014-1302090017 & 334625.82 \\
		1302030005 & 1302030005-1302030005-1302030014-1302030006-1302030009-1302030012-1302030004-1302030010-1302031004-1302030003-1302030002-1302031003-1302110021-1302110007 & 330329.51 \\
		1302070006 & 1302070006-1302070006-1302070002-1302070011-1302070005-1302080004-1302070003-1302070012-1302070007-1302070001-1302070009-1302070010 & 277740.88 \\
		1302050007 & 1302050007-1302050007-1302050006-1302050004-1302050003-1302050005-1302050009-1302050010-1302050002-1302050008-1302050001-1302090004-1302090020-1302090019 & 323349.34 \\
		1302031005 & 1302031005-1302031001-1302031006-1302031008-1302040011-1302031009-1302031010-1302031002-1302031007-1302030001-1302031005-1302040006 & 273412.25 \\
\end{longtable}


\begin{longtable}[h]{lrr}
	\caption{Komparasi CoES dan Pub/Sub + CoES pada Data Lapangan}
	\label{tbl:test_result_normal_field_comparison}\\
	\toprule
		Ukuran & CoES & CoES + Pub/Sub\\ 
	\midrule
	\endfirsthead
	\toprule
		Ukuran & CoES & CoES + Pub/Sub\\ 
	\midrule
	\endhead
	\bottomrule
	\endfoot
		Total Waktu & 4448989.67 & 4559658.67\\
		Rata-rata & 296599.31 & 303977.24\\
		Standar Deviasi & 119720.84 & 34472.12\\
\end{longtable}


%-----------------------------------------------------------------------------%
\subsubsection{Pengujian Kondisi \textit{Unstable Connection}}
%-----------------------------------------------------------------------------%
Skenario pengujian kondisi \textit{unstable connection} dimaksudkan untuk membandingkan program yang dijalankan pada kondisi dimana terjadi \textit{delay} secara random. Pengujian ini digunakan sebagai cerminan kodisi lapangan dimana pada beberapa lokasi tidak terdapat koneksi yang stabil. Pengujian akan dilakukan satu kali untuk data lapangan. Pada pengujian ini sebuah program \textit{client} juga dirancang sebagai simulator.


\begin{figure}[h]
    \centering
    \includegraphics[width=\textwidth]{../../Resources/Images/client-algorithm-unstable-connection-field}
    \caption{\textit{Flowchart} pada \textit{Client Simulator} Kondisi \textit{Unstable Connection}}
    \label{fig:client-algorithm-unstable-connection-field}
\end{figure}


Langkah-langkah yang diimplementasikan pada program \textit{client}, sebagaimana ilustrasi Gambar \ref{fig:client-algorithm-unstable-connection-field}, adalah sebagai berikut:

\begin{enumerate}
\item Buat sebuah \textit{forward proxy server} yang mengimplementasikan \textit{random delay}, dengan destinasi adalah IP dari \textit{message broker}, 
\item Setiap komunikasi dengan \textit{message broker} harus melalui \textit{forward proxy server}, 
\item Buat koneksi dengan \textit{message broker}, 
\item Lakukan \textit{subscription} dengan topik \textit{current location}, 
\item Setelah sebuah \textit{message} diterima, dimana \textit{message} dari \textit{broker} merupakan \textit{next location}, kalkulasi waktu tempuh antara \textit{current location} dengan \textit{next location},
\item Simulasikan perjalanan ke lokasi dengan \textit{sleep},
\item Buat notifikasi terhadap broker atas perubahan \textit{current location}, 
\item Generate \textit{service time}, mengikuti distribusi normal, 
\item Simulasikan pencacahan dengan \textit{sleep}, 
\item Ulangi lagi dari langkah pertama.
\end{enumerate}


%-----------------------------------------------------------------------------%
\subsubsection{Pengujian Kondisi Pencacah Berhenti}
%-----------------------------------------------------------------------------%
Skenario pengujian kondisi normal dimaksudkan untuk membandingkan program yang dijalankan pada kondisi dimana satu atau lebih pencacah berhenti ditengah jalan. Pengujian akan dilakukan satu kali untuk data lapangan. Pada pengujian ini sebuah program \textit{client} dirancang sebagai simulator.


\begin{figure}[h]
    \centering
    \includegraphics[width=\textwidth]{../../Resources/Images/client-algorithm-enumerator-quit-field}
    \caption{\textit{Flowchart} pada \textit{Client Simulator} Kondisi Pencacah Berhenti}
    \label{fig:client-algorithm-enumerator-quit-field}
\end{figure}


Langkah-langkah yang diimplementasikan pada program \textit{client}, sebagaimana ilustrasi Gambar \ref{fig:client-algorithm-enumerator-quit-field}, adalah sebagai berikut:

\begin{enumerate}
\item Buat koneksi dengan \textit{message broker}, 
\item Lakukan \textit{subscription} dengan topik \textit{current location}, 
\item Setelah sebuah \textit{message} diterima, dimana \textit{message} dari \textit{broker} merupakan \textit{next location}, kalkulasi waktu tempuh antara \textit{current location} dengan \textit{next location},
\item Simulasikan perjalanan ke lokasi dengan \textit{sleep},
\item Buat notifikasi terhadap broker atas perubahan \textit{current location}, 
\item Generate \textit{service time}, mengikuti distribusi normal, 
\item Simulasikan pencacahan dengan \textit{sleep}, 
\item Hentikan beberapa proses, 
\item Ulangi lagi dari langkah pertama.
\end{enumerate}


Pada pengujian ini, disimulasikan bahwasannya pencacah 1302110003 berhenti setelah 150000 detik, 1302101005 setelah 360000 detik, 1302012003 setelah 180000 detik, dan 1302080006 setelah 300000 detik. Tabel \ref{tbl:test_result_enumerator_quit_field_pubsub_coes} menunjukkan hasil simulasi dan waktu total untuk masing-masing pencacah. Ketika rute dari keempat pencacah tersebut dikeluarkan dari penghitungan standar deviasi, maka sebagaimana Tabel \ref{tbl:test_result_enumerator_quit_field} diperoleh .


\begin{longtable}[h]{lp{8cm}r}
	\caption{Hasil Pengujian Kondisi Pencacah Berhenti}
	\label{tbl:test_result_enumerator_quit_field_pubsub_coes}\\
	\toprule
		\textit{Vehicle} & Rute & Waktu Total\\ 
	\midrule
	\endfirsthead
	\toprule
		\textit{Vehicle} & Rute & Waktu Total\\ 
	\midrule
	\endhead
	\bottomrule
	\endfoot
		1302011008 & 1302011008-1302011008-1302011010-1302011001-1302012005-1302011007-1302011009-1302011005-1302011006-1302011002-1302011004-1302011003-1302090007-1302090015-1302110002-1302110013-1302110006 & 407401.10 \\
		1302040002 & 1302040002-1302040002-1302040016-1302040004-1302040005-1302040006-1302040015-1302040007-1302040003-1302040009-1302040013-1302040008-1302090013 & 306233.49 \\
		1302012003 & 1302012003-1302012008-1302012003-1302012001-1302012006-1302012002-1302012004-1302012010 & 169070.15 \\
		1302060005 & 1302060005-1302060005-1302060009-1302060006-1302060008-1302060007-1302060001-1302060004-1302050010-1302050009-1302060003-1302012009-1302100016-1302090010 & 345013.89 \\
		1302020006 & 1302020006-1302020006-1302020016-1302020015-1302021002-1302021006-1302020017-1302031005 & 171648.83 \\
		1302080006 & 1302080006-1302080006-1302080007-1302080008-1302080002-1302080001-1302080004-1302080009-1302070012-1302070003-1302080003-1302080005 & 269836.91 \\
		1302021005 & 1302021005-1302021005-1302021001-1302020011-1302020009-1302020001-1302020010-1302021003-1302021008-1302021010-1302021007-1302021009-1302021004-1302110012 & 335580.21 \\
		1302100002 & 1302100002-1302100002-1302100003-1302100013-1302100009-1302100001-1302100005-1302110023-1302110021-1302110022-1302110008-1302110007-1302110020-1302110018-1302110009-1302110003-1302110010-1302110005 & 428783.15 \\
		1302101005 & 1302101005-1302101005-1302100010-1302101002-1302100014-1302100011-1302101003-1302101001-1302100006-1302100004-1302100012-1302100007-1302100017-1302101004 & 327886.72 \\
		1302110003 & 1302110003-1302110019-1302110004-1302110001-1302110011-1302110017-1302110016 & 147061.67 \\
		1302090004 & 1302090004-1302090008-1302090002-1302090005-1302090004-1302090012-1302090014-1302090020-1302090019-1302090001-1302090017-1302090018-1302090016-1302110014-1302101006-1302090011 & 368625.31 \\
		1302030005 & 1302030005-1302030005-1302030014-1302030006-1302030009-1302030004-1302030012-1302030003-1302031004-1302030010-1302030002-1302031003-1302020005-1302020003-1302100015-1302100008 & 382133.70 \\
		1302070006 & 1302070006-1302070006-1302070002-1302070011-1302070005-1302070008-1302070009-1302070010-1302070007-1302070001-1302070004-1302060002-1302110015 & 299606.48 \\
		1302050007 & 1302050007-1302050007-1302050008-1302050002-1302050006-1302050004-1302050001-1302050003-1302050005-1302040001-1302040012-1302012007-1302090006-1302090009-1302090003 & 369932.48 \\
		1302031005 & 1302031005-1302031001-1302031006-1302031008-1302040011-1302031009-1302031010-1302040010-1302031007-1302030001-1302031002-1302040014 & 278953.18 \\
\end{longtable}


\begin{longtable}[h]{lrr}
	\caption{Komparasi CoES dan Pub/Sub + CoES pada Data Lapangan}
	\label{tbl:test_result_enumerator_quit_field}\\
	\toprule
		Ukuran & CoES & CoES + Pub/Sub\\ 
	\midrule
	\endfirsthead
	\toprule
		Ukuran & CoES & CoES + Pub/Sub\\ 
	\midrule
	\endhead
	\bottomrule
	\endfoot
		Total Waktu & 4448989.67 & 3693911.82\\
		Rata-rata & 296599.31 & 335810.17\\
		Standar Deviasi & 119720.84 & 67828.78\\
\end{longtable}


%---------------------------------------------------------------
\chapter{\kesimpulan}
%---------------------------------------------------------------
%\todo{Tambahkan kesimpulan dan saran terkait dengan perkerjaan 
%	yang dilakukan.}


%---------------------------------------------------------------
\section{Kesimpulan}
%---------------------------------------------------------------
Berdasarkan hasil pengujian, diperoleh kesimpulan sebagai berikut:

\begin{enumerate}
	\item Pada permasalahan MDVRP yang tidak mempertimbangkan \textit{service time}, diperoleh kesimpulan bahwasannya sistem usulan yang menggunakan algoritma MDVRP berbasis CoEAs yang dikombinasikan dengan mekanisme Publish/Subscribe menghasilkan total waktu 80 persen lebih buruk dari program pembanding yang menggunakan algoritma MDVRP berbasis CoEAs tanpa mekanisme Publish/Subscribe. Sementara dari sisi \textit{standar deviasi}, sistem usulan menghasilkan 80 persen lebih baik dari aplikasi pembanding.
	\item Pada permasalahan MDVRP yang mempertimbangkan \textit{service time}, dengan menggunakan data Cordeau PO1 diperoleh hasil total waktu yang dihasilkan dari sistem usulan 6,25 persen lebih buruk dari aplikasi pembanding, tetapi 68,83 persen lebih baik dari sisi standar deviasi. Sementara pada pengujian dengan menggunakan data lapangan, sistem usulan lebih buruk 8,25 persen dibandingkan program pembanding, tetapi 59,11 persen dari sisi standar deviasi.
	\item Pada permasalahan MDVRP yang mempertimbangkan \textit{service time} dan terdapat \textit{delay} dalam jaringan, pada pengujian dengan menggunakan data lapangan diperoleh hasil bahwasannya sistem usulan 4,83 persen lebih buruk dari sisi total waktu, tetapi 3,62 persen lebih baik dari sisi standar deviasi.
	\item Pada permasalahan MDVRP yang mempertimbangkan \textit{service time} dan terdapat \textit{packet loss} dalam jaringan, pada pengujian dengan menggunakan data lapangan diperoleh hasil bahwasannya sistem usulan 12,79 persen lebih buruk dari sisi total waktu, tetapi 36,14 persen lebih baik dari sisi standar deviasi.
\end{enumerate}


%---------------------------------------------------------------
\section{Saran}
%---------------------------------------------------------------
Dari penelitian ini, saran yang diberikan untuk penelitian selanjutnya sebagai berikut:

\begin{enumerate}
	\item Mengkaji mekanisme komunikasi yang digunakan, misalnya dengan membandingkan dengan mekanisme Request/Reply dan Push/Pull untuk mendapatkan mekanisme komunikasi yang paling tepat.
	\item Mengkaji parameter yang dapat digunakan dalam VRP Solver yaitu lama waktu eksekusi dan lama waktu dimana tidak ada perubahan solusi terbaik, sehingga diperoleh waktu eksekusi yang paling optimal.
	\item Mengkasi penggunaan algoritma MDVRP yang lain, misalnya Tabu Search, Particle Swarm Optimization, dan lain lain, sehingga diperoleh algoritma MDVRP yang paling sesuai.
\end{enumerate}


%
% Daftar Pustaka
%
% Daftar Pustaka 
% 

% 
% Tambahkan pustaka yang digunakan setelah perintah berikut. 
% 
\singlespacing
\addcontentsline{}{chapter}{\bibname}
\bibliography{\bibFile}
\onehalfspacing



%
% Lampiran 
%

%\appendix
%
%\include{markLampiran}
%%-----------------------------------------------------------------------------%
\addChapter{LAMPIRAN X}
\chapter*{Lampiran X}
\label{ch:lampiran_hasil_pengujian_lapangan_kondisi_normal}
%-----------------------------------------------------------------------------%


\begin{longtable}[!]{c|ccc}
	\caption{Hasil pengujian kondisi normal pada data lapangan \textit{instance} tw01 dengan menggunakan sistem usulan (algoritma MDVRP berbasis CoEAs dan mekanisme Publish/Subscribe)}
	\label{tbl:test_result_field_tw01}\\
	\toprule
	Pengujian & \MyHead{3.1cm}{Total waktu pencacahan dari seluruh pencacah (hari)} & \MyHead{3.1cm}{Rata-rata waktu pencacahan dari setiap pencacah (hari)} & \MyHead{3.1cm}{Standar deviasi waktu pencacahan dari seluruh pencacah (hari)} \\ 
	\midrule
	\endfirsthead
	\toprule
	\textit{Pengujian} & \MyHead{3.1cm}{Total waktu pencacahan dari seluruh pencacah (hari)} & \MyHead{3.1cm}{Rata-rata waktu pencacahan dari setiap pencacah (hari)} & \MyHead{3.1cm}{Standar deviasi waktu pencacahan dari seluruh pencacah (hari)} \\ 
	\midrule
	\endhead
	\bottomrule
	\endfoot
	1	& 197	& 13	& 1	\\
	2	& 197	& 13	& 2	\\
	3	& 197	& 13	& 1	\\
	4	& 197	& 13	& 1	\\
	5	& 197	& 13	& 2	\\
	6	& 197	& 13	& 2	\\
	7	& 197	& 13	& 1	\\
	8	& 197	& 13	& 1	\\
	9	& 197	& 13	& 1	\\
	10	& 197	& 13	& 2	\\
	11	& 197	& 13	& 2	\\
	12	& 197	& 13	& 1	\\
	13	& 197	& 13	& 1	\\
	14	& 197	& 13	& 1	\\
	15	& 197	& 13	& 2	\\
	16	& 197	& 13	& 2	\\
	17	& 197	& 13	& 2	\\
	18	& 197	& 13	& 1	\\
	19	& 197	& 13	& 2	\\
	20	& 197	& 13	& 2	\\
	21	& 197	& 13	& 2	\\
	22	& 197	& 13	& 2	\\
	23	& 197	& 13	& 2	\\
	24	& 197	& 13	& 1	\\
	25	& 197	& 13	& 1	\\
	26	& 197	& 13	& 2	\\
	27	& 197	& 13	& 1	\\
	28	& 197	& 13	& 2	\\
	29	& 197	& 13	& 2	\\
	30	& 197	& 13	& 1	\\
	31	& 197	& 13	& 2	\\
	32	& 197	& 13	& 2	\\
	33	& 197	& 13	& 3	\\
	34	& 197	& 13	& 2	\\
	35	& 197	& 13	& 1	\\
	36	& 197	& 13	& 2	\\
	37	& 197	& 13	& 2	\\
	38	& 197	& 13	& 1	\\
	39	& 197	& 13	& 2	\\
	40	& 197	& 13	& 1	\\
	41	& 197	& 13	& 2	\\
	42	& 197	& 13	& 1	\\
	43	& 197	& 13	& 2	\\
	44	& 197	& 13	& 2	\\
	45	& 197	& 13	& 1	\\
	46	& 197	& 13	& 1	\\
	47	& 197	& 13	& 2	\\
	48	& 197	& 13	& 1	\\
	49	& 197	& 13	& 1	\\
	50	& 197	& 13	& 2	\\
	51	& 197	& 13	& 1	\\
	52	& 197	& 13	& 1	\\
	53	& 197	& 13	& 1	\\
	54	& 197	& 13	& 2	\\
	55	& 197	& 13	& 1	\\
	56	& 197	& 13	& 1	\\
	57	& 197	& 13	& 2	\\
	58	& 197	& 13	& 2	\\
	59	& 197	& 13	& 1	\\
	60	& 197	& 13	& 2	\\
	61	& 197	& 13	& 1	\\
	62	& 197	& 13	& 1	\\
	63	& 197	& 13	& 2	\\
	64	& 197	& 13	& 2	\\
	65	& 197	& 13	& 1	\\
	66	& 197	& 13	& 1	\\
	67	& 197	& 13	& 1	\\
	68	& 197	& 13	& 2	\\
	69	& 197	& 13	& 1	\\
	70	& 197	& 13	& 2	\\
	71	& 197	& 13	& 2	\\
	72	& 197	& 13	& 3	\\
	73	& 197	& 13	& 1	\\
	74	& 197	& 13	& 1	\\
	75	& 197	& 13	& 2	\\
	76	& 197	& 13	& 2	\\
	77	& 197	& 13	& 1	\\
	78	& 197	& 13	& 2	\\
	79	& 197	& 13	& 1	\\
	80	& 197	& 13	& 2	\\
	81	& 197	& 13	& 1	\\
	82	& 197	& 13	& 3	\\
	83	& 197	& 13	& 2	\\
	84	& 197	& 13	& 2	\\
	85	& 197	& 13	& 1	\\
	86	& 197	& 13	& 2	\\
	87	& 197	& 13	& 1	\\
	88	& 197	& 13	& 2	\\
	89	& 197	& 13	& 2	\\
	90	& 197	& 13	& 2	\\
	91	& 197	& 13	& 1	\\
	92	& 197	& 13	& 2	\\
	93	& 197	& 13	& 2	\\
	94	& 197	& 13	& 2	\\
	95	& 197	& 13	& 1	\\
	96	& 197	& 13	& 1	\\
	97	& 197	& 13	& 1	\\
	98	& 197	& 13	& 1	\\
	99	& 197	& 13	& 1	\\
	100	& 197	& 13	& 2	\\
\end{longtable}


\begin{longtable}[!]{c|ccc}
	\caption{Hasil pengujian kondisi normal pada data lapangan \textit{instance} tw02 dengan menggunakan sistem usulan (algoritma MDVRP berbasis CoEAs dan mekanisme Publish/Subscribe)}
	\label{tbl:test_result_field_tw02}\\
	\toprule
	Pengujian & \MyHead{3.1cm}{Total waktu pencacahan dari seluruh pencacah (hari)} & \MyHead{3.1cm}{Rata-rata waktu pencacahan dari setiap pencacah (hari)} & \MyHead{3.1cm}{Standar deviasi waktu pencacahan dari seluruh pencacah (hari)} \\ 
	\midrule
	\endfirsthead
	\toprule
	\textit{Pengujian} & \MyHead{3.1cm}{Total waktu pencacahan dari seluruh pencacah (hari)} & \MyHead{3.1cm}{Rata-rata waktu pencacahan dari setiap pencacah (hari)} & \MyHead{3.1cm}{Standar deviasi waktu pencacahan dari seluruh pencacah (hari)} \\ 
	\midrule
	\endhead
	\bottomrule
	\endfoot
	1	& 197	& 13	& 2	\\
	2	& 197	& 13	& 1	\\
	3	& 197	& 13	& 1	\\
	4	& 197	& 13	& 0	\\
	5	& 197	& 13	& 0	\\
	6	& 197	& 13	& 0	\\
	7	& 197	& 13	& 0	\\
	8	& 197	& 13	& 0	\\
	9	& 197	& 13	& 0	\\
	10	& 197	& 13	& 0	\\
	11	& 197	& 13	& 0	\\
	12	& 197	& 13	& 0	\\
	13	& 197	& 13	& 0	\\
	14	& 197	& 13	& 0	\\
	15	& 197	& 13	& 0	\\
	16	& 197	& 13	& 0	\\
	17	& 197	& 13	& 0	\\
	18	& 197	& 13	& 0	\\
	19	& 197	& 13	& 0	\\
	20	& 197	& 13	& 0	\\
	21	& 197	& 13	& 0	\\
	22	& 197	& 13	& 0	\\
	23	& 197	& 13	& 1	\\
	24	& 197	& 13	& 0	\\
	25	& 197	& 13	& 0	\\
	26	& 197	& 13	& 1	\\
	27	& 197	& 13	& 0	\\
	28	& 197	& 13	& 0	\\
	29	& 197	& 13	& 0	\\
	30	& 197	& 13	& 0	\\
	31	& 197	& 13	& 0	\\
	32	& 197	& 13	& 0	\\
	33	& 197	& 13	& 0	\\
	34	& 197	& 13	& 0	\\
	35	& 197	& 13	& 1	\\
	36	& 197	& 13	& 0	\\
	37	& 197	& 13	& 0	\\
	38	& 197	& 13	& 0	\\
	39	& 197	& 13	& 0	\\
	40	& 197	& 13	& 0	\\
	41	& 197	& 13	& 0	\\
	42	& 197	& 13	& 0	\\
	43	& 197	& 13	& 0	\\
	44	& 197	& 13	& 0	\\
	45	& 197	& 13	& 0	\\
	46	& 197	& 13	& 0	\\
	47	& 197	& 13	& 0	\\
	48	& 197	& 13	& 0	\\
	49	& 197	& 13	& 0	\\
	50	& 197	& 13	& 0	\\
	51	& 197	& 13	& 0	\\
	52	& 197	& 13	& 0	\\
	53	& 197	& 13	& 0	\\
	54	& 197	& 13	& 0	\\
	55	& 197	& 13	& 0	\\
	56	& 197	& 13	& 0	\\
	57	& 197	& 13	& 0	\\
	58	& 197	& 13	& 0	\\
	59	& 197	& 13	& 0	\\
	60	& 197	& 13	& 0	\\
	61	& 197	& 13	& 0	\\
	62	& 197	& 13	& 0	\\
	63	& 197	& 13	& 1	\\
	64	& 197	& 13	& 0	\\
	65	& 197	& 13	& 0	\\
	66	& 197	& 13	& 0	\\
	67	& 197	& 13	& 0	\\
	68	& 197	& 13	& 0	\\
	69	& 197	& 13	& 0	\\
	70	& 197	& 13	& 0	\\
	71	& 197	& 13	& 0	\\
	72	& 197	& 13	& 0	\\
	73	& 197	& 13	& 0	\\
	74	& 197	& 13	& 0	\\
	75	& 197	& 13	& 0	\\
	76	& 197	& 13	& 0	\\
	77	& 197	& 13	& 0	\\
	78	& 197	& 13	& 0	\\
	79	& 197	& 13	& 0	\\
	80	& 197	& 13	& 0	\\
	81	& 197	& 13	& 0	\\
	82	& 197	& 13	& 0	\\
	83	& 197	& 13	& 0	\\
	84	& 197	& 13	& 0	\\
	85	& 197	& 13	& 0	\\
	86	& 197	& 13	& 0	\\
	87	& 197	& 13	& 0	\\
	88	& 197	& 13	& 0	\\
	89	& 197	& 13	& 0	\\
	90	& 197	& 13	& 0	\\
	91	& 197	& 13	& 0	\\
	92	& 197	& 13	& 0	\\
	93	& 197	& 13	& 0	\\
	94	& 197	& 13	& 0	\\
	95	& 197	& 13	& 0	\\
	96	& 197	& 13	& 0	\\
	97	& 197	& 13	& 0	\\
	98	& 197	& 13	& 0	\\
	99	& 197	& 13	& 0	\\
	100	& 197	& 13	& 0	\\
\end{longtable}


\begin{longtable}[!]{c|ccc}
	\caption{Hasil pengujian kondisi normal pada data lapangan \textit{instance} tw03 dengan menggunakan sistem usulan (algoritma MDVRP berbasis CoEAs dan mekanisme Publish/Subscribe)}
	\label{tbl:test_result_field_tw03}\\
	\toprule
	Pengujian & \MyHead{3.1cm}{Total waktu pencacahan dari seluruh pencacah (hari)} & \MyHead{3.1cm}{Rata-rata waktu pencacahan dari setiap pencacah (hari)} & \MyHead{3.1cm}{Standar deviasi waktu pencacahan dari seluruh pencacah (hari)} \\ 
	\midrule
	\endfirsthead
	\toprule
	\textit{Pengujian} & \MyHead{3.1cm}{Total waktu pencacahan dari seluruh pencacah (hari)} & \MyHead{3.1cm}{Rata-rata waktu pencacahan dari setiap pencacah (hari)} & \MyHead{3.1cm}{Standar deviasi waktu pencacahan dari seluruh pencacah (hari)} \\ 
	\midrule
	\endhead
	\bottomrule
	\endfoot
	1	& 197	& 13	& 1	\\
	2	& 197	& 13	& 1	\\
	3	& 197	& 13	& 1	\\
	4	& 197	& 13	& 1	\\
	5	& 197	& 13	& 0	\\
	6	& 197	& 13	& 0	\\
	7	& 197	& 13	& 1	\\
	8	& 197	& 13	& 0	\\
	9	& 197	& 13	& 1	\\
	10	& 197	& 13	& 1	\\
	11	& 197	& 13	& 1	\\
	12	& 197	& 13	& 0	\\
	13	& 197	& 13	& 1	\\
	14	& 197	& 13	& 0	\\
	15	& 197	& 13	& 1	\\
	16	& 197	& 13	& 1	\\
	17	& 197	& 13	& 1	\\
	18	& 197	& 13	& 1	\\
	19	& 197	& 13	& 1	\\
	20	& 197	& 13	& 1	\\
	21	& 197	& 13	& 1	\\
	22	& 197	& 13	& 1	\\
	23	& 197	& 13	& 1	\\
	24	& 197	& 13	& 1	\\
	25	& 197	& 13	& 1	\\
	26	& 197	& 13	& 0	\\
	27	& 197	& 13	& 1	\\
	28	& 197	& 13	& 0	\\
	29	& 197	& 13	& 1	\\
	30	& 197	& 13	& 1	\\
	31	& 197	& 13	& 1	\\
	32	& 197	& 13	& 1	\\
	33	& 197	& 13	& 1	\\
	34	& 197	& 13	& 1	\\
	35	& 197	& 13	& 1	\\
	36	& 197	& 13	& 1	\\
	37	& 197	& 13	& 1	\\
	38	& 197	& 13	& 1	\\
	39	& 197	& 13	& 1	\\
	40	& 197	& 13	& 1	\\
	41	& 197	& 13	& 1	\\
	42	& 197	& 13	& 1	\\
	43	& 197	& 13	& 1	\\
	44	& 197	& 13	& 1	\\
	45	& 197	& 13	& 1	\\
	46	& 197	& 13	& 1	\\
	47	& 197	& 13	& 0	\\
	48	& 197	& 13	& 1	\\
	49	& 197	& 13	& 1	\\
	50	& 197	& 13	& 1	\\
	51	& 197	& 13	& 1	\\
	52	& 197	& 13	& 1	\\
	53	& 197	& 13	& 0	\\
	54	& 197	& 13	& 1	\\
	55	& 197	& 13	& 0	\\
	56	& 197	& 13	& 1	\\
	57	& 197	& 13	& 1	\\
	58	& 197	& 13	& 1	\\
	59	& 197	& 13	& 2	\\
	60	& 197	& 13	& 0	\\
	61	& 197	& 13	& 1	\\
	62	& 197	& 13	& 0	\\
	63	& 197	& 13	& 1	\\
	64	& 197	& 13	& 1	\\
	65	& 197	& 13	& 1	\\
	66	& 197	& 13	& 1	\\
	67	& 197	& 13	& 0	\\
	68	& 197	& 13	& 1	\\
	69	& 197	& 13	& 1	\\
	70	& 197	& 13	& 0	\\
	71	& 197	& 13	& 1	\\
	72	& 197	& 13	& 1	\\
	73	& 197	& 13	& 1	\\
	74	& 197	& 13	& 0	\\
	75	& 197	& 13	& 1	\\
	76	& 197	& 13	& 1	\\
	77	& 197	& 13	& 1	\\
	78	& 197	& 13	& 0	\\
	79	& 197	& 13	& 1	\\
	80	& 197	& 13	& 1	\\
	81	& 197	& 13	& 1	\\
	82	& 197	& 13	& 1	\\
	83	& 197	& 13	& 1	\\
	84	& 197	& 13	& 1	\\
	85	& 197	& 13	& 0	\\
	86	& 197	& 13	& 1	\\
	87	& 197	& 13	& 1	\\
	88	& 197	& 13	& 1	\\
	89	& 197	& 13	& 1	\\
	90	& 197	& 13	& 1	\\
	91	& 197	& 13	& 1	\\
	92	& 197	& 13	& 1	\\
	93	& 197	& 13	& 1	\\
	94	& 197	& 13	& 1	\\
	95	& 197	& 13	& 1	\\
	96	& 197	& 13	& 1	\\
	97	& 197	& 13	& 1	\\
	98	& 197	& 13	& 0	\\
	99	& 197	& 13	& 1	\\
	100	& 197	& 13	& 1	\\
\end{longtable}


\begin{longtable}[!]{c|ccc}
	\caption{Hasil pengujian kondisi normal pada data lapangan \textit{instance} tw04 dengan menggunakan sistem usulan (algoritma MDVRP berbasis CoEAs dan mekanisme Publish/Subscribe)}
	\label{tbl:test_result_field_tw04}\\
	\toprule
	Pengujian & \MyHead{3.1cm}{Total waktu pencacahan dari seluruh pencacah (hari)} & \MyHead{3.1cm}{Rata-rata waktu pencacahan dari setiap pencacah (hari)} & \MyHead{3.1cm}{Standar deviasi waktu pencacahan dari seluruh pencacah (hari)} \\ 
	\midrule
	\endfirsthead
	\toprule
	\textit{Pengujian} & \MyHead{3.1cm}{Total waktu pencacahan dari seluruh pencacah (hari)} & \MyHead{3.1cm}{Rata-rata waktu pencacahan dari setiap pencacah (hari)} & \MyHead{3.1cm}{Standar deviasi waktu pencacahan dari seluruh pencacah (hari)} \\ 
	\midrule
	\endhead
	\bottomrule
	\endfoot
	1	& 197	& 13	& 0	\\
	2	& 197	& 13	& 0	\\
	3	& 197	& 13	& 1	\\
	4	& 197	& 13	& 0	\\
	5	& 197	& 13	& 0	\\
	6	& 197	& 13	& 0	\\
	7	& 197	& 13	& 0	\\
	8	& 197	& 13	& 0	\\
	9	& 197	& 13	& 1	\\
	10	& 197	& 13	& 0	\\
	11	& 197	& 13	& 0	\\
	12	& 197	& 13	& 0	\\
	13	& 197	& 13	& 0	\\
	14	& 197	& 13	& 0	\\
	15	& 197	& 13	& 0	\\
	16	& 197	& 13	& 0	\\
	17	& 197	& 13	& 0	\\
	18	& 197	& 13	& 0	\\
	19	& 197	& 13	& 0	\\
	20	& 197	& 13	& 0	\\
	21	& 197	& 13	& 0	\\
	22	& 197	& 13	& 1	\\
	23	& 197	& 13	& 0	\\
	24	& 197	& 13	& 0	\\
	25	& 197	& 13	& 1	\\
	26	& 197	& 13	& 0	\\
	27	& 197	& 13	& 0	\\
	28	& 197	& 13	& 0	\\
	29	& 197	& 13	& 0	\\
	30	& 197	& 13	& 0	\\
	31	& 197	& 13	& 0	\\
	32	& 197	& 13	& 0	\\
	33	& 197	& 13	& 0	\\
	34	& 197	& 13	& 0	\\
	35	& 197	& 13	& 0	\\
	36	& 197	& 13	& 0	\\
	37	& 197	& 13	& 0	\\
	38	& 197	& 13	& 0	\\
	39	& 197	& 13	& 0	\\
	40	& 197	& 13	& 0	\\
	41	& 197	& 13	& 0	\\
	42	& 197	& 13	& 0	\\
	43	& 197	& 13	& 0	\\
	44	& 197	& 13	& 0	\\
	45	& 197	& 13	& 0	\\
	46	& 197	& 13	& 0	\\
	47	& 197	& 13	& 0	\\
	48	& 197	& 13	& 0	\\
	49	& 197	& 13	& 0	\\
	50	& 197	& 13	& 0	\\
	51	& 197	& 13	& 0	\\
	52	& 197	& 13	& 0	\\
	53	& 197	& 13	& 0	\\
	54	& 197	& 13	& 0	\\
	55	& 197	& 13	& 0	\\
	56	& 197	& 13	& 0	\\
	57	& 197	& 13	& 0	\\
	58	& 197	& 13	& 0	\\
	59	& 197	& 13	& 0	\\
	60	& 197	& 13	& 0	\\
	61	& 197	& 13	& 0	\\
	62	& 197	& 13	& 0	\\
	63	& 197	& 13	& 0	\\
	64	& 197	& 13	& 0	\\
	65	& 197	& 13	& 0	\\
	66	& 197	& 13	& 0	\\
	67	& 197	& 13	& 0	\\
	68	& 197	& 13	& 0	\\
	69	& 197	& 13	& 0	\\
	70	& 197	& 13	& 0	\\
	71	& 197	& 13	& 0	\\
	72	& 197	& 13	& 0	\\
	73	& 197	& 13	& 0	\\
	74	& 197	& 13	& 0	\\
	75	& 197	& 13	& 0	\\
	76	& 197	& 13	& 0	\\
	77	& 197	& 13	& 0	\\
	78	& 197	& 13	& 0	\\
	79	& 197	& 13	& 0	\\
	80	& 197	& 13	& 0	\\
	81	& 197	& 13	& 0	\\
	82	& 197	& 13	& 0	\\
	83	& 197	& 13	& 0	\\
	84	& 197	& 13	& 0	\\
	85	& 197	& 13	& 0	\\
	86	& 197	& 13	& 0	\\
	87	& 197	& 13	& 0	\\
	88	& 197	& 13	& 0	\\
	89	& 197	& 13	& 0	\\
	90	& 197	& 13	& 0	\\
	91	& 197	& 13	& 0	\\
	92	& 197	& 13	& 0	\\
	93	& 197	& 13	& 0	\\
	94	& 197	& 13	& 0	\\
	95	& 197	& 13	& 0	\\
	96	& 197	& 13	& 0	\\
	97	& 197	& 13	& 0	\\
	98	& 197	& 13	& 0	\\
	99	& 197	& 13	& 0	\\
	100	& 197	& 13	& 0	\\
\end{longtable}
%%-----------------------------------------------------------------------------%
\addChapter{LAMPIRAN X}
\chapter*{Lampiran X}
\label{ch:lampiran_hasil_pengujian_kondisi_delay}
%-----------------------------------------------------------------------------%


\begin{longtable}[!]{c|ccc}
	\caption{Hasil pengujian kondisi \textit{delay} pada data lapangan \textit{instance} d01 dengan menggunakan sistem usulan (algoritma MDVRP berbasis CoEAs dan mekanisme Publish/Subscribe)}
	\label{tbl:test_result_d01_tw}\\
	\toprule
	Pengujian & \MyHead{3.1cm}{Total waktu pencacahan dari seluruh pencacah (hari)} & \MyHead{3.1cm}{Rata-rata waktu pencacahan dari setiap pencacah (hari)} & \MyHead{3.1cm}{Standar deviasi waktu pencacahan dari seluruh pencacah (hari)} \\ 
	\midrule
	\endfirsthead
	\toprule
	\textit{Pengujian} & \MyHead{3.1cm}{Total waktu pencacahan dari seluruh pencacah (hari)} & \MyHead{3.1cm}{Rata-rata waktu pencacahan dari setiap pencacah (hari)} & \MyHead{3.1cm}{Standar deviasi waktu pencacahan dari seluruh pencacah (hari)} \\ 
	\midrule
	\endhead
	\bottomrule
	\endfoot
	1	& 223	& 14	& 3	\\
	2	& 221	& 14	& 2	\\
	3	& 219	& 14	& 3	\\
	4	& 223	& 14	& 1	\\
	5	& 230	& 15	& 1	\\
	6	& 223	& 14	& 1	\\
	7	& 222	& 14	& 1	\\
	8	& 227	& 15	& 1	\\
	9	& 229	& 15	& 1	\\
	10	& 221	& 14	& 1	\\
	11	& 224	& 14	& 2	\\
	12	& 215	& 14	& 1	\\
	13	& 234	& 15	& 1	\\
	14	& 220	& 14	& 1	\\
	15	& 217	& 14	& 1	\\
	16	& 229	& 15	& 1	\\
	17	& 219	& 14	& 1	\\
	18	& 226	& 15	& 1	\\
	19	& 233	& 15	& 1	\\
	20	& 219	& 14	& 1	\\
	21	& 218	& 14	& 1	\\
	22	& 220	& 14	& 1	\\
	23	& 215	& 14	& 1	\\
	24	& 220	& 14	& 1	\\
	25	& 220	& 14	& 1	\\
	26	& 218	& 14	& 1	\\
	27	& 221	& 14	& 1	\\
	28	& 224	& 14	& 1	\\
	29	& 217	& 14	& 1	\\
	30	& 227	& 15	& 1	\\
	31	& 222	& 14	& 1	\\
	32	& 226	& 15	& 1	\\
	33	& 234	& 15	& 2	\\
	34	& 219	& 14	& 1	\\
	35	& 224	& 14	& 1	\\
	36	& 229	& 15	& 1	\\
	37	& 224	& 14	& 1	\\
	38	& 222	& 14	& 1	\\
	39	& 224	& 14	& 2	\\
	40	& 218	& 14	& 1	\\
	41	& 224	& 14	& 1	\\
	42	& 221	& 14	& 1	\\
	43	& 235	& 15	& 1	\\
	44	& 223	& 14	& 1	\\
	45	& 218	& 14	& 1	\\
	46	& 234	& 15	& 1	\\
	47	& 217	& 14	& 1	\\
	48	& 224	& 14	& 1	\\
	49	& 224	& 14	& 1	\\
	50	& 219	& 14	& 1	\\
	51	& 223	& 14	& 1	\\
	52	& 221	& 14	& 1	\\
	53	& 227	& 15	& 1	\\
	54	& 221	& 14	& 1	\\
	55	& 222	& 14	& 1	\\
	56	& 226	& 15	& 1	\\
	57	& 224	& 14	& 1	\\
	58	& 223	& 14	& 1	\\
	59	& 231	& 15	& 1	\\
	60	& 220	& 14	& 1	\\
	61	& 221	& 14	& 1	\\
	62	& 228	& 15	& 1	\\
	63	& 217	& 14	& 1	\\
	64	& 226	& 15	& 1	\\
	65	& 227	& 15	& 1	\\
	66	& 225	& 15	& 1	\\
	67	& 224	& 14	& 1	\\
	68	& 225	& 15	& 2	\\
	69	& 227	& 15	& 1	\\
	70	& 223	& 14	& 1	\\
	71	& 234	& 15	& 1	\\
	72	& 227	& 15	& 1	\\
	73	& 227	& 15	& 1	\\
	74	& 221	& 14	& 2	\\
	75	& 226	& 15	& 1	\\
	76	& 225	& 15	& 1	\\
	77	& 221	& 14	& 1	\\
	78	& 229	& 15	& 1	\\
	79	& 226	& 15	& 1	\\
	80	& 231	& 15	& 1	\\
	81	& 233	& 15	& 1	\\
	82	& 222	& 14	& 1	\\
	83	& 231	& 15	& 1	\\
	84	& 225	& 15	& 1	\\
	85	& 221	& 14	& 1	\\
	86	& 227	& 15	& 1	\\
	87	& 221	& 14	& 1	\\
	88	& 224	& 14	& 2	\\
	89	& 218	& 14	& 1	\\
	90	& 225	& 15	& 1	\\
	91	& 222	& 14	& 1	\\
	92	& 228	& 15	& 1	\\
	93	& 220	& 14	& 1	\\
	94	& 223	& 14	& 1	\\
	95	& 213	& 14	& 0	\\
	96	& 219	& 14	& 1	\\
	97	& 218	& 14	& 2	\\
	98	& 230	& 15	& 1	\\
	99	& 227	& 15	& 1	\\
	100	& 225	& 15	& 1	\\
\end{longtable}


\begin{longtable}[!]{c|ccc}
	\caption{Hasil pengujian kondisi \textit{delay} pada data lapangan \textit{instance} d02 dengan menggunakan sistem usulan (algoritma MDVRP berbasis CoEAs dan mekanisme Publish/Subscribe)}
	\label{tbl:test_result_d02_tw}\\
	\toprule
	Pengujian & \MyHead{3.1cm}{Total waktu pencacahan dari seluruh pencacah (hari)} & \MyHead{3.1cm}{Rata-rata waktu pencacahan dari setiap pencacah (hari)} & \MyHead{3.1cm}{Standar deviasi waktu pencacahan dari seluruh pencacah (hari)} \\ 
	\midrule
	\endfirsthead
	\toprule
	\textit{Pengujian} & \MyHead{3.1cm}{Total waktu pencacahan dari seluruh pencacah (hari)} & \MyHead{3.1cm}{Rata-rata waktu pencacahan dari setiap pencacah (hari)} & \MyHead{3.1cm}{Standar deviasi waktu pencacahan dari seluruh pencacah (hari)} \\ 
	\midrule
	\endhead
	\bottomrule
	\endfoot
	1	& 219	& 14	& 3	\\
	2	& 226	& 15	& 3	\\
	3	& 220	& 14	& 4	\\
	4	& 222	& 14	& 1	\\
	5	& 216	& 14	& 1	\\
	6	& 219	& 14	& 1	\\
	7	& 223	& 14	& 1	\\
	8	& 225	& 15	& 1	\\
	9	& 218	& 14	& 1	\\
	10	& 217	& 14	& 1	\\
	11	& 223	& 14	& 1	\\
	12	& 218	& 14	& 1	\\
	13	& 221	& 14	& 1	\\
	14	& 219	& 14	& 1	\\
	15	& 227	& 15	& 1	\\
	16	& 226	& 15	& 1	\\
	17	& 225	& 15	& 1	\\
	18	& 226	& 15	& 1	\\
	19	& 228	& 15	& 1	\\
	20	& 220	& 14	& 2	\\
	21	& 225	& 15	& 1	\\
	22	& 220	& 14	& 1	\\
	23	& 221	& 14	& 1	\\
	24	& 223	& 14	& 1	\\
	25	& 228	& 15	& 1	\\
	26	& 229	& 15	& 1	\\
	27	& 230	& 15	& 1	\\
	28	& 228	& 15	& 2	\\
	29	& 224	& 14	& 1	\\
	30	& 225	& 15	& 1	\\
	31	& 223	& 14	& 1	\\
	32	& 218	& 14	& 1	\\
	33	& 221	& 14	& 1	\\
	34	& 220	& 14	& 2	\\
	35	& 224	& 14	& 1	\\
	36	& 220	& 14	& 1	\\
	37	& 225	& 15	& 1	\\
	38	& 223	& 14	& 1	\\
	39	& 226	& 15	& 1	\\
	40	& 226	& 15	& 1	\\
	41	& 226	& 15	& 1	\\
	42	& 218	& 14	& 1	\\
	43	& 223	& 14	& 1	\\
	44	& 232	& 15	& 2	\\
	45	& 222	& 14	& 1	\\
	46	& 224	& 14	& 1	\\
	47	& 217	& 14	& 1	\\
	48	& 226	& 15	& 1	\\
	49	& 228	& 15	& 1	\\
	50	& 220	& 14	& 1	\\
	51	& 234	& 15	& 1	\\
	52	& 218	& 14	& 1	\\
	53	& 225	& 15	& 2	\\
	54	& 227	& 15	& 2	\\
	55	& 225	& 15	& 2	\\
	56	& 219	& 14	& 3	\\
	57	& 221	& 14	& 3	\\
	58	& 228	& 15	& 3	\\
	59	& 229	& 15	& 3	\\
	60	& 221	& 14	& 3	\\
	61	& 219	& 14	& 1	\\
	62	& 223	& 14	& 1	\\
	63	& 222	& 14	& 1	\\
	64	& 223	& 14	& 1	\\
	65	& 214	& 14	& 1	\\
	66	& 217	& 14	& 1	\\
	67	& 222	& 14	& 1	\\
	68	& 225	& 15	& 1	\\
	69	& 226	& 15	& 0	\\
	70	& 228	& 15	& 1	\\
	71	& 220	& 14	& 1	\\
	72	& 223	& 14	& 1	\\
	73	& 224	& 14	& 1	\\
	74	& 226	& 15	& 1	\\
	75	& 216	& 14	& 1	\\
	76	& 217	& 14	& 1	\\
	77	& 231	& 15	& 1	\\
	78	& 224	& 14	& 1	\\
	79	& 226	& 15	& 1	\\
	80	& 219	& 14	& 1	\\
	81	& 227	& 15	& 1	\\
	82	& 217	& 14	& 0	\\
	83	& 213	& 14	& 1	\\
	84	& 226	& 15	& 1	\\
	85	& 223	& 14	& 2	\\
	86	& 224	& 14	& 2	\\
	87	& 224	& 14	& 1	\\
	88	& 222	& 14	& 1	\\
	89	& 219	& 14	& 1	\\
	90	& 227	& 15	& 1	\\
	91	& 232	& 15	& 1	\\
	92	& 220	& 14	& 1	\\
	93	& 214	& 14	& 1	\\
	94	& 225	& 15	& 1	\\
	95	& 227	& 15	& 1	\\
	96	& 225	& 15	& 1	\\
	97	& 219	& 14	& 1	\\
	98	& 229	& 15	& 2	\\
	99	& 220	& 14	& 1	\\
	100	& 225	& 15	& 1	\\
\end{longtable}


\begin{longtable}[!]{c|ccc}
	\caption{Hasil pengujian kondisi \textit{delay} pada data lapangan \textit{instance} d03 dengan menggunakan sistem usulan (algoritma MDVRP berbasis CoEAs dan mekanisme Publish/Subscribe)}
	\label{tbl:test_result_d03_tw}\\
	\toprule
	Pengujian & \MyHead{3.1cm}{Total waktu pencacahan dari seluruh pencacah (hari)} & \MyHead{3.1cm}{Rata-rata waktu pencacahan dari setiap pencacah (hari)} & \MyHead{3.1cm}{Standar deviasi waktu pencacahan dari seluruh pencacah (hari)} \\ 
	\midrule
	\endfirsthead
	\toprule
	\textit{Pengujian} & \MyHead{3.1cm}{Total waktu pencacahan dari seluruh pencacah (hari)} & \MyHead{3.1cm}{Rata-rata waktu pencacahan dari setiap pencacah (hari)} & \MyHead{3.1cm}{Standar deviasi waktu pencacahan dari seluruh pencacah (hari)} \\ 
	\midrule
	\endhead
	\bottomrule
	\endfoot
	1	& 364	& 24	& 4	\\
	2	& 364	& 24	& 3	\\
	3	& 364	& 24	& 5	\\
	4	& 364	& 24	& 1	\\
	5	& 364	& 24	& 1	\\
	6	& 364	& 24	& 1	\\
	7	& 364	& 24	& 1	\\
	8	& 364	& 24	& 1	\\
	9	& 364	& 24	& 1	\\
	10	& 364	& 24	& 1	\\
	11	& 364	& 24	& 2	\\
	12	& 364	& 24	& 1	\\
	13	& 364	& 24	& 1	\\
	14	& 364	& 24	& 1	\\
	15	& 364	& 24	& 1	\\
	16	& 364	& 24	& 1	\\
	17	& 364	& 24	& 1	\\
	18	& 364	& 24	& 1	\\
	19	& 364	& 24	& 1	\\
	20	& 364	& 24	& 1	\\
	21	& 364	& 24	& 1	\\
	22	& 364	& 24	& 1	\\
	23	& 364	& 24	& 1	\\
	24	& 364	& 24	& 2	\\
	25	& 364	& 24	& 1	\\
	26	& 364	& 24	& 1	\\
	27	& 364	& 24	& 1	\\
	28	& 364	& 24	& 1	\\
	29	& 364	& 24	& 1	\\
	30	& 364	& 24	& 1	\\
	31	& 364	& 24	& 1	\\
	32	& 364	& 24	& 1	\\
	33	& 364	& 24	& 1	\\
	34	& 364	& 24	& 1	\\
	35	& 364	& 24	& 1	\\
	36	& 364	& 24	& 1	\\
	37	& 364	& 24	& 1	\\
	38	& 364	& 24	& 1	\\
	39	& 364	& 24	& 1	\\
	40	& 364	& 24	& 1	\\
	41	& 364	& 24	& 1	\\
	42	& 364	& 24	& 2	\\
	43	& 364	& 24	& 1	\\
	44	& 364	& 24	& 1	\\
	45	& 364	& 24	& 1	\\
	46	& 364	& 24	& 1	\\
	47	& 364	& 24	& 2	\\
	48	& 364	& 24	& 1	\\
	49	& 364	& 24	& 1	\\
	50	& 364	& 24	& 2	\\
	51	& 364	& 24	& 1	\\
	52	& 364	& 24	& 1	\\
	53	& 364	& 24	& 1	\\
	54	& 364	& 24	& 1	\\
	55	& 364	& 24	& 1	\\
	56	& 364	& 24	& 1	\\
	57	& 364	& 24	& 1	\\
	58	& 364	& 24	& 1	\\
	59	& 364	& 24	& 2	\\
	60	& 364	& 24	& 1	\\
	61	& 364	& 24	& 2	\\
	62	& 364	& 24	& 1	\\
	63	& 364	& 24	& 1	\\
	64	& 364	& 24	& 1	\\
	65	& 364	& 24	& 1	\\
	66	& 364	& 24	& 1	\\
	67	& 364	& 24	& 1	\\
	68	& 364	& 24	& 1	\\
	69	& 364	& 24	& 1	\\
	70	& 364	& 24	& 1	\\
	71	& 364	& 24	& 1	\\
	72	& 364	& 24	& 1	\\
	73	& 364	& 24	& 1	\\
	74	& 364	& 24	& 1	\\
	75	& 364	& 24	& 1	\\
	76	& 364	& 24	& 1	\\
	77	& 364	& 24	& 1	\\
	78	& 364	& 24	& 1	\\
	79	& 364	& 24	& 1	\\
	80	& 364	& 24	& 1	\\
	81	& 364	& 24	& 1	\\
	82	& 364	& 24	& 1	\\
	83	& 364	& 24	& 1	\\
	84	& 364	& 24	& 0	\\
	85	& 364	& 24	& 1	\\
	86	& 364	& 24	& 1	\\
	87	& 364	& 24	& 1	\\
	88	& 364	& 24	& 1	\\
	89	& 364	& 24	& 1	\\
	90	& 364	& 24	& 1	\\
	91	& 364	& 24	& 1	\\
	92	& 364	& 24	& 1	\\
	93	& 364	& 24	& 1	\\
	94	& 364	& 24	& 1	\\
	95	& 364	& 24	& 1	\\
	96	& 364	& 24	& 1	\\
	97	& 364	& 24	& 1	\\
	98	& 364	& 24	& 1	\\
	99	& 364	& 24	& 1	\\
	100	& 364	& 24	& 1	\\
\end{longtable}


\begin{longtable}[!]{c|ccc}
	\caption{Hasil pengujian kondisi \textit{delay} pada data lapangan \textit{instance} d04 dengan menggunakan sistem usulan (algoritma MDVRP berbasis CoEAs dan mekanisme Publish/Subscribe)}
	\label{tbl:test_result_d04_tw}\\
	\toprule
	Pengujian & \MyHead{3.1cm}{Total waktu pencacahan dari seluruh pencacah (hari)} & \MyHead{3.1cm}{Rata-rata waktu pencacahan dari setiap pencacah (hari)} & \MyHead{3.1cm}{Standar deviasi waktu pencacahan dari seluruh pencacah (hari)} \\ 
	\midrule
	\endfirsthead
	\toprule
	\textit{Pengujian} & \MyHead{3.1cm}{Total waktu pencacahan dari seluruh pencacah (hari)} & \MyHead{3.1cm}{Rata-rata waktu pencacahan dari setiap pencacah (hari)} & \MyHead{3.1cm}{Standar deviasi waktu pencacahan dari seluruh pencacah (hari)} \\ 
	\midrule
	\endhead
	\bottomrule
	\endfoot
	1	& 224	& 14	& 2	\\
	2	& 227	& 15	& 2	\\
	3	& 220	& 14	& 2	\\
	4	& 228	& 15	& 1	\\
	5	& 230	& 15	& 3	\\
	6	& 224	& 14	& 1	\\
	7	& 224	& 14	& 1	\\
	8	& 220	& 14	& 1	\\
	9	& 226	& 15	& 1	\\
	10	& 217	& 14	& 1	\\
	11	& 230	& 15	& 1	\\
	12	& 214	& 14	& 1	\\
	13	& 224	& 14	& 1	\\
	14	& 227	& 15	& 1	\\
	15	& 216	& 14	& 1	\\
	16	& 223	& 14	& 2	\\
	17	& 226	& 15	& 1	\\
	18	& 217	& 14	& 1	\\
	19	& 229	& 15	& 2	\\
	20	& 218	& 14	& 1	\\
	21	& 213	& 14	& 1	\\
	22	& 221	& 14	& 1	\\
	23	& 213	& 14	& 1	\\
	24	& 222	& 14	& 1	\\
	25	& 229	& 15	& 1	\\
	26	& 232	& 15	& 1	\\
	27	& 225	& 15	& 1	\\
	28	& 221	& 14	& 1	\\
	29	& 226	& 15	& 1	\\
	30	& 217	& 14	& 1	\\
	31	& 221	& 14	& 1	\\
	32	& 223	& 14	& 1	\\
	33	& 234	& 15	& 1	\\
	34	& 221	& 14	& 1	\\
	35	& 230	& 15	& 1	\\
	36	& 222	& 14	& 1	\\
	37	& 223	& 14	& 1	\\
	38	& 226	& 15	& 1	\\
	39	& 218	& 14	& 1	\\
	40	& 220	& 14	& 2	\\
	41	& 229	& 15	& 1	\\
	42	& 224	& 14	& 1	\\
	43	& 221	& 14	& 1	\\
	44	& 228	& 15	& 1	\\
	45	& 222	& 14	& 1	\\
	46	& 219	& 14	& 1	\\
	47	& 227	& 15	& 1	\\
	48	& 226	& 15	& 1	\\
	49	& 216	& 14	& 1	\\
	50	& 220	& 14	& 2	\\
	51	& 220	& 14	& 1	\\
	52	& 223	& 14	& 1	\\
	53	& 229	& 15	& 1	\\
	54	& 238	& 15	& 1	\\
	55	& 220	& 14	& 1	\\
	56	& 229	& 15	& 1	\\
	57	& 220	& 14	& 1	\\
	58	& 230	& 15	& 1	\\
	59	& 224	& 14	& 1	\\
	60	& 222	& 14	& 1	\\
	61	& 227	& 15	& 1	\\
	62	& 224	& 14	& 1	\\
	63	& 229	& 15	& 2	\\
	64	& 225	& 15	& 1	\\
	65	& 218	& 14	& 1	\\
	66	& 218	& 14	& 1	\\
	67	& 222	& 14	& 1	\\
	68	& 224	& 14	& 1	\\
	69	& 231	& 15	& 1	\\
	70	& 225	& 15	& 1	\\
	71	& 222	& 14	& 1	\\
	72	& 216	& 14	& 1	\\
	73	& 226	& 15	& 1	\\
	74	& 227	& 15	& 2	\\
	75	& 232	& 15	& 2	\\
	76	& 232	& 15	& 1	\\
	77	& 221	& 14	& 1	\\
	78	& 222	& 14	& 1	\\
	79	& 233	& 15	& 1	\\
	80	& 221	& 14	& 1	\\
	81	& 222	& 14	& 1	\\
	82	& 216	& 14	& 1	\\
	83	& 218	& 14	& 1	\\
	84	& 218	& 14	& 1	\\
	85	& 211	& 14	& 1	\\
	86	& 221	& 14	& 1	\\
	87	& 219	& 14	& 1	\\
	88	& 225	& 15	& 1	\\
	89	& 231	& 15	& 1	\\
	90	& 225	& 15	& 1	\\
	91	& 219	& 14	& 1	\\
	92	& 225	& 15	& 1	\\
	93	& 215	& 14	& 1	\\
	94	& 221	& 14	& 1	\\
	95	& 222	& 14	& 2	\\
	96	& 222	& 14	& 1	\\
	97	& 236	& 15	& 1	\\
	98	& 231	& 15	& 2	\\
	99	& 220	& 14	& 1	\\
	100	& 209	& 13	& 1	\\
\end{longtable}


\begin{longtable}[!]{c|ccc}
	\caption{Hasil pengujian kondisi \textit{delay} pada data lapangan \textit{instance} d05 dengan menggunakan sistem usulan (algoritma MDVRP berbasis CoEAs dan mekanisme Publish/Subscribe)}
	\label{tbl:test_result_d05_tw}\\
	\toprule
	Pengujian & \MyHead{3.1cm}{Total waktu pencacahan dari seluruh pencacah (hari)} & \MyHead{3.1cm}{Rata-rata waktu pencacahan dari setiap pencacah (hari)} & \MyHead{3.1cm}{Standar deviasi waktu pencacahan dari seluruh pencacah (hari)} \\ 
	\midrule
	\endfirsthead
	\toprule
	\textit{Pengujian} & \MyHead{3.1cm}{Total waktu pencacahan dari seluruh pencacah (hari)} & \MyHead{3.1cm}{Rata-rata waktu pencacahan dari setiap pencacah (hari)} & \MyHead{3.1cm}{Standar deviasi waktu pencacahan dari seluruh pencacah (hari)} \\ 
	\midrule
	\endhead
	\bottomrule
	\endfoot
	1	& 364	& 24	& 4	\\
	2	& 364	& 24	& 3	\\
	3	& 364	& 24	& 1	\\
	4	& 364	& 24	& 1	\\
	5	& 364	& 24	& 1	\\
	6	& 364	& 24	& 1	\\
	7	& 364	& 24	& 1	\\
	8	& 364	& 24	& 1	\\
	9	& 364	& 24	& 1	\\
	10	& 364	& 24	& 1	\\
	11	& 364	& 24	& 2	\\
	12	& 364	& 24	& 1	\\
	13	& 364	& 24	& 1	\\
	14	& 364	& 24	& 1	\\
	15	& 364	& 24	& 1	\\
	16	& 364	& 24	& 1	\\
	17	& 364	& 24	& 1	\\
	18	& 364	& 24	& 1	\\
	19	& 364	& 24	& 1	\\
	20	& 364	& 24	& 1	\\
	21	& 364	& 24	& 1	\\
	22	& 364	& 24	& 2	\\
	23	& 364	& 24	& 3	\\
	24	& 364	& 24	& 1	\\
	25	& 364	& 24	& 2	\\
	26	& 364	& 24	& 1	\\
	27	& 364	& 24	& 1	\\
	28	& 364	& 24	& 1	\\
	29	& 364	& 24	& 0	\\
	30	& 364	& 24	& 1	\\
	31	& 364	& 24	& 1	\\
	32	& 364	& 24	& 1	\\
	33	& 364	& 24	& 2	\\
	34	& 364	& 24	& 1	\\
	35	& 364	& 24	& 1	\\
	36	& 364	& 24	& 2	\\
	37	& 364	& 24	& 1	\\
	38	& 364	& 24	& 1	\\
	39	& 364	& 24	& 1	\\
	40	& 364	& 24	& 2	\\
	41	& 364	& 24	& 2	\\
	42	& 364	& 24	& 1	\\
	43	& 364	& 24	& 1	\\
	44	& 364	& 24	& 1	\\
	45	& 364	& 24	& 1	\\
	46	& 364	& 24	& 1	\\
	47	& 364	& 24	& 1	\\
	48	& 364	& 24	& 1	\\
	49	& 364	& 24	& 1	\\
	50	& 364	& 24	& 1	\\
	51	& 364	& 24	& 1	\\
	52	& 364	& 24	& 1	\\
	53	& 364	& 24	& 1	\\
	54	& 364	& 24	& 1	\\
	55	& 364	& 24	& 1	\\
	56	& 364	& 24	& 1	\\
	57	& 364	& 24	& 1	\\
	58	& 364	& 24	& 1	\\
	59	& 364	& 24	& 1	\\
	60	& 364	& 24	& 1	\\
	61	& 364	& 24	& 1	\\
	62	& 364	& 24	& 1	\\
	63	& 364	& 24	& 1	\\
	64	& 364	& 24	& 1	\\
	65	& 364	& 24	& 2	\\
	66	& 364	& 24	& 1	\\
	67	& 364	& 24	& 1	\\
	68	& 364	& 24	& 1	\\
	69	& 364	& 24	& 1	\\
	70	& 364	& 24	& 1	\\
	71	& 364	& 24	& 1	\\
	72	& 364	& 24	& 1	\\
	73	& 364	& 24	& 1	\\
	74	& 364	& 24	& 1	\\
	75	& 364	& 24	& 1	\\
	76	& 364	& 24	& 1	\\
	77	& 364	& 24	& 1	\\
	78	& 364	& 24	& 1	\\
	79	& 364	& 24	& 1	\\
	80	& 364	& 24	& 1	\\
	81	& 364	& 24	& 1	\\
	82	& 364	& 24	& 1	\\
	83	& 364	& 24	& 1	\\
	84	& 364	& 24	& 1	\\
	85	& 364	& 24	& 1	\\
	86	& 364	& 24	& 1	\\
	87	& 364	& 24	& 1	\\
	88	& 364	& 24	& 1	\\
	89	& 364	& 24	& 1	\\
	90	& 364	& 24	& 1	\\
	91	& 364	& 24	& 1	\\
	92	& 364	& 24	& 1	\\
	93	& 364	& 24	& 1	\\
	94	& 364	& 24	& 1	\\
	95	& 364	& 24	& 1	\\
	96	& 364	& 24	& 1	\\
	97	& 364	& 24	& 1	\\
	98	& 364	& 24	& 1	\\
	99	& 364	& 24	& 1	\\
	100	& 364	& 24	& 1	\\
\end{longtable}


\begin{longtable}[!]{c|ccc}
	\caption{Hasil pengujian kondisi \textit{delay} pada data lapangan \textit{instance} d06 dengan menggunakan sistem usulan (algoritma MDVRP berbasis CoEAs dan mekanisme Publish/Subscribe)}
	\label{tbl:test_result_d06_tw}\\
	\toprule
	Pengujian & \MyHead{3.1cm}{Total waktu pencacahan dari seluruh pencacah (hari)} & \MyHead{3.1cm}{Rata-rata waktu pencacahan dari setiap pencacah (hari)} & \MyHead{3.1cm}{Standar deviasi waktu pencacahan dari seluruh pencacah (hari)} \\ 
	\midrule
	\endfirsthead
	\toprule
	\textit{Pengujian} & \MyHead{3.1cm}{Total waktu pencacahan dari seluruh pencacah (hari)} & \MyHead{3.1cm}{Rata-rata waktu pencacahan dari setiap pencacah (hari)} & \MyHead{3.1cm}{Standar deviasi waktu pencacahan dari seluruh pencacah (hari)} \\ 
	\midrule
	\endhead
	\bottomrule
	\endfoot
	1	& 364	& 24	& 3	\\
	2	& 364	& 24	& 4	\\
	3	& 364	& 24	& 4	\\
	4	& 364	& 24	& 2	\\
	5	& 364	& 24	& 1	\\
	6	& 364	& 24	& 1	\\
	7	& 364	& 24	& 1	\\
	8	& 364	& 24	& 1	\\
	9	& 364	& 24	& 1	\\
	10	& 364	& 24	& 1	\\
	11	& 364	& 24	& 2	\\
	12	& 364	& 24	& 2	\\
	13	& 364	& 24	& 2	\\
	14	& 364	& 24	& 2	\\
	15	& 364	& 24	& 1	\\
	16	& 364	& 24	& 1	\\
	17	& 364	& 24	& 1	\\
	18	& 364	& 24	& 2	\\
	19	& 364	& 24	& 2	\\
	20	& 364	& 24	& 1	\\
	21	& 364	& 24	& 2	\\
	22	& 364	& 24	& 2	\\
	23	& 364	& 24	& 2	\\
	24	& 364	& 24	& 1	\\
	25	& 364	& 24	& 2	\\
	26	& 364	& 24	& 1	\\
	27	& 364	& 24	& 1	\\
	28	& 364	& 24	& 1	\\
	29	& 364	& 24	& 1	\\
	30	& 364	& 24	& 1	\\
	31	& 364	& 24	& 1	\\
	32	& 364	& 24	& 2	\\
	33	& 364	& 24	& 2	\\
	34	& 364	& 24	& 2	\\
	35	& 364	& 24	& 2	\\
	36	& 364	& 24	& 2	\\
	37	& 364	& 24	& 1	\\
	38	& 364	& 24	& 1	\\
	39	& 364	& 24	& 2	\\
	40	& 364	& 24	& 1	\\
	41	& 364	& 24	& 2	\\
	42	& 364	& 24	& 2	\\
	43	& 364	& 24	& 1	\\
	44	& 364	& 24	& 1	\\
	45	& 364	& 24	& 1	\\
	46	& 364	& 24	& 1	\\
	47	& 364	& 24	& 1	\\
	48	& 364	& 24	& 1	\\
	49	& 364	& 24	& 1	\\
	50	& 364	& 24	& 1	\\
	51	& 364	& 24	& 1	\\
	52	& 364	& 24	& 1	\\
	53	& 364	& 24	& 2	\\
	54	& 364	& 24	& 2	\\
	55	& 364	& 24	& 1	\\
	56	& 364	& 24	& 1	\\
	57	& 364	& 24	& 2	\\
	58	& 364	& 24	& 1	\\
	59	& 364	& 24	& 1	\\
	60	& 364	& 24	& 1	\\
	61	& 364	& 24	& 2	\\
	62	& 364	& 24	& 2	\\
	63	& 364	& 24	& 2	\\
	64	& 364	& 24	& 2	\\
	65	& 364	& 24	& 1	\\
	66	& 364	& 24	& 1	\\
	67	& 364	& 24	& 1	\\
	68	& 364	& 24	& 1	\\
	69	& 364	& 24	& 2	\\
	70	& 364	& 24	& 1	\\
	71	& 364	& 24	& 1	\\
	72	& 364	& 24	& 2	\\
	73	& 364	& 24	& 2	\\
	74	& 364	& 24	& 1	\\
	75	& 364	& 24	& 1	\\
	76	& 364	& 24	& 2	\\
	77	& 364	& 24	& 1	\\
	78	& 364	& 24	& 1	\\
	79	& 364	& 24	& 2	\\
	80	& 364	& 24	& 1	\\
	81	& 364	& 24	& 1	\\
	82	& 364	& 24	& 1	\\
	83	& 364	& 24	& 2	\\
	84	& 364	& 24	& 2	\\
	85	& 364	& 24	& 1	\\
	86	& 364	& 24	& 2	\\
	87	& 364	& 24	& 1	\\
	88	& 364	& 24	& 2	\\
	89	& 364	& 24	& 1	\\
	90	& 364	& 24	& 1	\\
	91	& 364	& 24	& 1	\\
	92	& 364	& 24	& 1	\\
	93	& 364	& 24	& 1	\\
	94	& 364	& 24	& 1	\\
	95	& 364	& 24	& 1	\\
	96	& 364	& 24	& 1	\\
	97	& 364	& 24	& 2	\\
	98	& 364	& 24	& 1	\\
	99	& 364	& 24	& 3	\\
	100	& 364	& 24	& 2	\\
\end{longtable}


\begin{longtable}[!]{c|ccc}
	\caption{Hasil pengujian kondisi \textit{delay} pada data lapangan \textit{instance} d07 dengan menggunakan sistem usulan (algoritma MDVRP berbasis CoEAs dan mekanisme Publish/Subscribe)}
	\label{tbl:test_result_d07_tw}\\
	\toprule
	Pengujian & \MyHead{3.1cm}{Total waktu pencacahan dari seluruh pencacah (hari)} & \MyHead{3.1cm}{Rata-rata waktu pencacahan dari setiap pencacah (hari)} & \MyHead{3.1cm}{Standar deviasi waktu pencacahan dari seluruh pencacah (hari)} \\ 
	\midrule
	\endfirsthead
	\toprule
	\textit{Pengujian} & \MyHead{3.1cm}{Total waktu pencacahan dari seluruh pencacah (hari)} & \MyHead{3.1cm}{Rata-rata waktu pencacahan dari setiap pencacah (hari)} & \MyHead{3.1cm}{Standar deviasi waktu pencacahan dari seluruh pencacah (hari)} \\ 
	\midrule
	\endhead
	\bottomrule
	\endfoot
	1	& 364	& 24	& 3	\\
	2	& 364	& 24	& 4	\\
	3	& 364	& 24	& 4	\\
	4	& 364	& 24	& 1	\\
	5	& 364	& 24	& 1	\\
	6	& 364	& 24	& 1	\\
	7	& 364	& 24	& 1	\\
	8	& 364	& 24	& 1	\\
	9	& 364	& 24	& 1	\\
	10	& 364	& 24	& 1	\\
	11	& 364	& 24	& 1	\\
	12	& 364	& 24	& 1	\\
	13	& 364	& 24	& 2	\\
	14	& 364	& 24	& 1	\\
	15	& 364	& 24	& 1	\\
	16	& 364	& 24	& 1	\\
	17	& 364	& 24	& 2	\\
	18	& 364	& 24	& 1	\\
	19	& 364	& 24	& 1	\\
	20	& 364	& 24	& 2	\\
	21	& 364	& 24	& 1	\\
	22	& 364	& 24	& 1	\\
	23	& 364	& 24	& 1	\\
	24	& 364	& 24	& 1	\\
	25	& 364	& 24	& 1	\\
	26	& 364	& 24	& 1	\\
	27	& 364	& 24	& 2	\\
	28	& 364	& 24	& 1	\\
	29	& 364	& 24	& 1	\\
	30	& 364	& 24	& 1	\\
	31	& 364	& 24	& 1	\\
	32	& 364	& 24	& 1	\\
	33	& 364	& 24	& 1	\\
	34	& 364	& 24	& 1	\\
	35	& 364	& 24	& 1	\\
	36	& 364	& 24	& 1	\\
	37	& 364	& 24	& 1	\\
	38	& 364	& 24	& 1	\\
	39	& 364	& 24	& 1	\\
	40	& 364	& 24	& 1	\\
	41	& 364	& 24	& 2	\\
	42	& 364	& 24	& 2	\\
	43	& 364	& 24	& 1	\\
	44	& 364	& 24	& 1	\\
	45	& 364	& 24	& 1	\\
	46	& 364	& 24	& 2	\\
	47	& 364	& 24	& 1	\\
	48	& 364	& 24	& 2	\\
	49	& 364	& 24	& 2	\\
	50	& 364	& 24	& 1	\\
	51	& 364	& 24	& 1	\\
	52	& 364	& 24	& 1	\\
	53	& 364	& 24	& 2	\\
	54	& 364	& 24	& 1	\\
	55	& 364	& 24	& 1	\\
	56	& 364	& 24	& 1	\\
	57	& 364	& 24	& 1	\\
	58	& 364	& 24	& 1	\\
	59	& 364	& 24	& 1	\\
	60	& 364	& 24	& 1	\\
	61	& 364	& 24	& 1	\\
	62	& 364	& 24	& 1	\\
	63	& 364	& 24	& 1	\\
	64	& 364	& 24	& 1	\\
	65	& 364	& 24	& 1	\\
	66	& 364	& 24	& 1	\\
	67	& 364	& 24	& 1	\\
	68	& 364	& 24	& 1	\\
	69	& 364	& 24	& 1	\\
	70	& 364	& 24	& 2	\\
	71	& 364	& 24	& 1	\\
	72	& 364	& 24	& 1	\\
	73	& 364	& 24	& 1	\\
	74	& 364	& 24	& 1	\\
	75	& 364	& 24	& 1	\\
	76	& 364	& 24	& 1	\\
	77	& 364	& 24	& 1	\\
	78	& 364	& 24	& 1	\\
	79	& 364	& 24	& 1	\\
	80	& 364	& 24	& 1	\\
	81	& 364	& 24	& 1	\\
	82	& 364	& 24	& 1	\\
	83	& 364	& 24	& 1	\\
	84	& 364	& 24	& 1	\\
	85	& 364	& 24	& 1	\\
	86	& 364	& 24	& 1	\\
	87	& 364	& 24	& 1	\\
	88	& 364	& 24	& 1	\\
	89	& 364	& 24	& 1	\\
	90	& 364	& 24	& 1	\\
	91	& 364	& 24	& 1	\\
	92	& 364	& 24	& 1	\\
	93	& 364	& 24	& 1	\\
	94	& 364	& 24	& 1	\\
	95	& 364	& 24	& 1	\\
	96	& 364	& 24	& 2	\\
	97	& 364	& 24	& 1	\\
	98	& 364	& 24	& 1	\\
	99	& 364	& 24	& 1	\\
	100	& 364	& 24	& 1	\\
\end{longtable}


\begin{appendices}
	\include{markLampiran}
	
	%-----------------------------------------------------------------------------%
\addChapter{LAMPIRAN X}
\chapter*{Lampiran X}
\label{ch:lampiran_hasil_pengujian_lapangan_kondisi_normal}
%-----------------------------------------------------------------------------%


\begin{longtable}[!]{c|ccc}
	\caption{Hasil pengujian kondisi normal pada data lapangan \textit{instance} tw01 dengan menggunakan sistem usulan (algoritma MDVRP berbasis CoEAs dan mekanisme Publish/Subscribe)}
	\label{tbl:test_result_field_tw01}\\
	\toprule
	Pengujian & \MyHead{3.1cm}{Total waktu pencacahan dari seluruh pencacah (hari)} & \MyHead{3.1cm}{Rata-rata waktu pencacahan dari setiap pencacah (hari)} & \MyHead{3.1cm}{Standar deviasi waktu pencacahan dari seluruh pencacah (hari)} \\ 
	\midrule
	\endfirsthead
	\toprule
	\textit{Pengujian} & \MyHead{3.1cm}{Total waktu pencacahan dari seluruh pencacah (hari)} & \MyHead{3.1cm}{Rata-rata waktu pencacahan dari setiap pencacah (hari)} & \MyHead{3.1cm}{Standar deviasi waktu pencacahan dari seluruh pencacah (hari)} \\ 
	\midrule
	\endhead
	\bottomrule
	\endfoot
	1	& 197	& 13	& 1	\\
	2	& 197	& 13	& 2	\\
	3	& 197	& 13	& 1	\\
	4	& 197	& 13	& 1	\\
	5	& 197	& 13	& 2	\\
	6	& 197	& 13	& 2	\\
	7	& 197	& 13	& 1	\\
	8	& 197	& 13	& 1	\\
	9	& 197	& 13	& 1	\\
	10	& 197	& 13	& 2	\\
	11	& 197	& 13	& 2	\\
	12	& 197	& 13	& 1	\\
	13	& 197	& 13	& 1	\\
	14	& 197	& 13	& 1	\\
	15	& 197	& 13	& 2	\\
	16	& 197	& 13	& 2	\\
	17	& 197	& 13	& 2	\\
	18	& 197	& 13	& 1	\\
	19	& 197	& 13	& 2	\\
	20	& 197	& 13	& 2	\\
	21	& 197	& 13	& 2	\\
	22	& 197	& 13	& 2	\\
	23	& 197	& 13	& 2	\\
	24	& 197	& 13	& 1	\\
	25	& 197	& 13	& 1	\\
	26	& 197	& 13	& 2	\\
	27	& 197	& 13	& 1	\\
	28	& 197	& 13	& 2	\\
	29	& 197	& 13	& 2	\\
	30	& 197	& 13	& 1	\\
	31	& 197	& 13	& 2	\\
	32	& 197	& 13	& 2	\\
	33	& 197	& 13	& 3	\\
	34	& 197	& 13	& 2	\\
	35	& 197	& 13	& 1	\\
	36	& 197	& 13	& 2	\\
	37	& 197	& 13	& 2	\\
	38	& 197	& 13	& 1	\\
	39	& 197	& 13	& 2	\\
	40	& 197	& 13	& 1	\\
	41	& 197	& 13	& 2	\\
	42	& 197	& 13	& 1	\\
	43	& 197	& 13	& 2	\\
	44	& 197	& 13	& 2	\\
	45	& 197	& 13	& 1	\\
	46	& 197	& 13	& 1	\\
	47	& 197	& 13	& 2	\\
	48	& 197	& 13	& 1	\\
	49	& 197	& 13	& 1	\\
	50	& 197	& 13	& 2	\\
	51	& 197	& 13	& 1	\\
	52	& 197	& 13	& 1	\\
	53	& 197	& 13	& 1	\\
	54	& 197	& 13	& 2	\\
	55	& 197	& 13	& 1	\\
	56	& 197	& 13	& 1	\\
	57	& 197	& 13	& 2	\\
	58	& 197	& 13	& 2	\\
	59	& 197	& 13	& 1	\\
	60	& 197	& 13	& 2	\\
	61	& 197	& 13	& 1	\\
	62	& 197	& 13	& 1	\\
	63	& 197	& 13	& 2	\\
	64	& 197	& 13	& 2	\\
	65	& 197	& 13	& 1	\\
	66	& 197	& 13	& 1	\\
	67	& 197	& 13	& 1	\\
	68	& 197	& 13	& 2	\\
	69	& 197	& 13	& 1	\\
	70	& 197	& 13	& 2	\\
	71	& 197	& 13	& 2	\\
	72	& 197	& 13	& 3	\\
	73	& 197	& 13	& 1	\\
	74	& 197	& 13	& 1	\\
	75	& 197	& 13	& 2	\\
	76	& 197	& 13	& 2	\\
	77	& 197	& 13	& 1	\\
	78	& 197	& 13	& 2	\\
	79	& 197	& 13	& 1	\\
	80	& 197	& 13	& 2	\\
	81	& 197	& 13	& 1	\\
	82	& 197	& 13	& 3	\\
	83	& 197	& 13	& 2	\\
	84	& 197	& 13	& 2	\\
	85	& 197	& 13	& 1	\\
	86	& 197	& 13	& 2	\\
	87	& 197	& 13	& 1	\\
	88	& 197	& 13	& 2	\\
	89	& 197	& 13	& 2	\\
	90	& 197	& 13	& 2	\\
	91	& 197	& 13	& 1	\\
	92	& 197	& 13	& 2	\\
	93	& 197	& 13	& 2	\\
	94	& 197	& 13	& 2	\\
	95	& 197	& 13	& 1	\\
	96	& 197	& 13	& 1	\\
	97	& 197	& 13	& 1	\\
	98	& 197	& 13	& 1	\\
	99	& 197	& 13	& 1	\\
	100	& 197	& 13	& 2	\\
\end{longtable}


\begin{longtable}[!]{c|ccc}
	\caption{Hasil pengujian kondisi normal pada data lapangan \textit{instance} tw02 dengan menggunakan sistem usulan (algoritma MDVRP berbasis CoEAs dan mekanisme Publish/Subscribe)}
	\label{tbl:test_result_field_tw02}\\
	\toprule
	Pengujian & \MyHead{3.1cm}{Total waktu pencacahan dari seluruh pencacah (hari)} & \MyHead{3.1cm}{Rata-rata waktu pencacahan dari setiap pencacah (hari)} & \MyHead{3.1cm}{Standar deviasi waktu pencacahan dari seluruh pencacah (hari)} \\ 
	\midrule
	\endfirsthead
	\toprule
	\textit{Pengujian} & \MyHead{3.1cm}{Total waktu pencacahan dari seluruh pencacah (hari)} & \MyHead{3.1cm}{Rata-rata waktu pencacahan dari setiap pencacah (hari)} & \MyHead{3.1cm}{Standar deviasi waktu pencacahan dari seluruh pencacah (hari)} \\ 
	\midrule
	\endhead
	\bottomrule
	\endfoot
	1	& 197	& 13	& 2	\\
	2	& 197	& 13	& 1	\\
	3	& 197	& 13	& 1	\\
	4	& 197	& 13	& 0	\\
	5	& 197	& 13	& 0	\\
	6	& 197	& 13	& 0	\\
	7	& 197	& 13	& 0	\\
	8	& 197	& 13	& 0	\\
	9	& 197	& 13	& 0	\\
	10	& 197	& 13	& 0	\\
	11	& 197	& 13	& 0	\\
	12	& 197	& 13	& 0	\\
	13	& 197	& 13	& 0	\\
	14	& 197	& 13	& 0	\\
	15	& 197	& 13	& 0	\\
	16	& 197	& 13	& 0	\\
	17	& 197	& 13	& 0	\\
	18	& 197	& 13	& 0	\\
	19	& 197	& 13	& 0	\\
	20	& 197	& 13	& 0	\\
	21	& 197	& 13	& 0	\\
	22	& 197	& 13	& 0	\\
	23	& 197	& 13	& 1	\\
	24	& 197	& 13	& 0	\\
	25	& 197	& 13	& 0	\\
	26	& 197	& 13	& 1	\\
	27	& 197	& 13	& 0	\\
	28	& 197	& 13	& 0	\\
	29	& 197	& 13	& 0	\\
	30	& 197	& 13	& 0	\\
	31	& 197	& 13	& 0	\\
	32	& 197	& 13	& 0	\\
	33	& 197	& 13	& 0	\\
	34	& 197	& 13	& 0	\\
	35	& 197	& 13	& 1	\\
	36	& 197	& 13	& 0	\\
	37	& 197	& 13	& 0	\\
	38	& 197	& 13	& 0	\\
	39	& 197	& 13	& 0	\\
	40	& 197	& 13	& 0	\\
	41	& 197	& 13	& 0	\\
	42	& 197	& 13	& 0	\\
	43	& 197	& 13	& 0	\\
	44	& 197	& 13	& 0	\\
	45	& 197	& 13	& 0	\\
	46	& 197	& 13	& 0	\\
	47	& 197	& 13	& 0	\\
	48	& 197	& 13	& 0	\\
	49	& 197	& 13	& 0	\\
	50	& 197	& 13	& 0	\\
	51	& 197	& 13	& 0	\\
	52	& 197	& 13	& 0	\\
	53	& 197	& 13	& 0	\\
	54	& 197	& 13	& 0	\\
	55	& 197	& 13	& 0	\\
	56	& 197	& 13	& 0	\\
	57	& 197	& 13	& 0	\\
	58	& 197	& 13	& 0	\\
	59	& 197	& 13	& 0	\\
	60	& 197	& 13	& 0	\\
	61	& 197	& 13	& 0	\\
	62	& 197	& 13	& 0	\\
	63	& 197	& 13	& 1	\\
	64	& 197	& 13	& 0	\\
	65	& 197	& 13	& 0	\\
	66	& 197	& 13	& 0	\\
	67	& 197	& 13	& 0	\\
	68	& 197	& 13	& 0	\\
	69	& 197	& 13	& 0	\\
	70	& 197	& 13	& 0	\\
	71	& 197	& 13	& 0	\\
	72	& 197	& 13	& 0	\\
	73	& 197	& 13	& 0	\\
	74	& 197	& 13	& 0	\\
	75	& 197	& 13	& 0	\\
	76	& 197	& 13	& 0	\\
	77	& 197	& 13	& 0	\\
	78	& 197	& 13	& 0	\\
	79	& 197	& 13	& 0	\\
	80	& 197	& 13	& 0	\\
	81	& 197	& 13	& 0	\\
	82	& 197	& 13	& 0	\\
	83	& 197	& 13	& 0	\\
	84	& 197	& 13	& 0	\\
	85	& 197	& 13	& 0	\\
	86	& 197	& 13	& 0	\\
	87	& 197	& 13	& 0	\\
	88	& 197	& 13	& 0	\\
	89	& 197	& 13	& 0	\\
	90	& 197	& 13	& 0	\\
	91	& 197	& 13	& 0	\\
	92	& 197	& 13	& 0	\\
	93	& 197	& 13	& 0	\\
	94	& 197	& 13	& 0	\\
	95	& 197	& 13	& 0	\\
	96	& 197	& 13	& 0	\\
	97	& 197	& 13	& 0	\\
	98	& 197	& 13	& 0	\\
	99	& 197	& 13	& 0	\\
	100	& 197	& 13	& 0	\\
\end{longtable}


\begin{longtable}[!]{c|ccc}
	\caption{Hasil pengujian kondisi normal pada data lapangan \textit{instance} tw03 dengan menggunakan sistem usulan (algoritma MDVRP berbasis CoEAs dan mekanisme Publish/Subscribe)}
	\label{tbl:test_result_field_tw03}\\
	\toprule
	Pengujian & \MyHead{3.1cm}{Total waktu pencacahan dari seluruh pencacah (hari)} & \MyHead{3.1cm}{Rata-rata waktu pencacahan dari setiap pencacah (hari)} & \MyHead{3.1cm}{Standar deviasi waktu pencacahan dari seluruh pencacah (hari)} \\ 
	\midrule
	\endfirsthead
	\toprule
	\textit{Pengujian} & \MyHead{3.1cm}{Total waktu pencacahan dari seluruh pencacah (hari)} & \MyHead{3.1cm}{Rata-rata waktu pencacahan dari setiap pencacah (hari)} & \MyHead{3.1cm}{Standar deviasi waktu pencacahan dari seluruh pencacah (hari)} \\ 
	\midrule
	\endhead
	\bottomrule
	\endfoot
	1	& 197	& 13	& 1	\\
	2	& 197	& 13	& 1	\\
	3	& 197	& 13	& 1	\\
	4	& 197	& 13	& 1	\\
	5	& 197	& 13	& 0	\\
	6	& 197	& 13	& 0	\\
	7	& 197	& 13	& 1	\\
	8	& 197	& 13	& 0	\\
	9	& 197	& 13	& 1	\\
	10	& 197	& 13	& 1	\\
	11	& 197	& 13	& 1	\\
	12	& 197	& 13	& 0	\\
	13	& 197	& 13	& 1	\\
	14	& 197	& 13	& 0	\\
	15	& 197	& 13	& 1	\\
	16	& 197	& 13	& 1	\\
	17	& 197	& 13	& 1	\\
	18	& 197	& 13	& 1	\\
	19	& 197	& 13	& 1	\\
	20	& 197	& 13	& 1	\\
	21	& 197	& 13	& 1	\\
	22	& 197	& 13	& 1	\\
	23	& 197	& 13	& 1	\\
	24	& 197	& 13	& 1	\\
	25	& 197	& 13	& 1	\\
	26	& 197	& 13	& 0	\\
	27	& 197	& 13	& 1	\\
	28	& 197	& 13	& 0	\\
	29	& 197	& 13	& 1	\\
	30	& 197	& 13	& 1	\\
	31	& 197	& 13	& 1	\\
	32	& 197	& 13	& 1	\\
	33	& 197	& 13	& 1	\\
	34	& 197	& 13	& 1	\\
	35	& 197	& 13	& 1	\\
	36	& 197	& 13	& 1	\\
	37	& 197	& 13	& 1	\\
	38	& 197	& 13	& 1	\\
	39	& 197	& 13	& 1	\\
	40	& 197	& 13	& 1	\\
	41	& 197	& 13	& 1	\\
	42	& 197	& 13	& 1	\\
	43	& 197	& 13	& 1	\\
	44	& 197	& 13	& 1	\\
	45	& 197	& 13	& 1	\\
	46	& 197	& 13	& 1	\\
	47	& 197	& 13	& 0	\\
	48	& 197	& 13	& 1	\\
	49	& 197	& 13	& 1	\\
	50	& 197	& 13	& 1	\\
	51	& 197	& 13	& 1	\\
	52	& 197	& 13	& 1	\\
	53	& 197	& 13	& 0	\\
	54	& 197	& 13	& 1	\\
	55	& 197	& 13	& 0	\\
	56	& 197	& 13	& 1	\\
	57	& 197	& 13	& 1	\\
	58	& 197	& 13	& 1	\\
	59	& 197	& 13	& 2	\\
	60	& 197	& 13	& 0	\\
	61	& 197	& 13	& 1	\\
	62	& 197	& 13	& 0	\\
	63	& 197	& 13	& 1	\\
	64	& 197	& 13	& 1	\\
	65	& 197	& 13	& 1	\\
	66	& 197	& 13	& 1	\\
	67	& 197	& 13	& 0	\\
	68	& 197	& 13	& 1	\\
	69	& 197	& 13	& 1	\\
	70	& 197	& 13	& 0	\\
	71	& 197	& 13	& 1	\\
	72	& 197	& 13	& 1	\\
	73	& 197	& 13	& 1	\\
	74	& 197	& 13	& 0	\\
	75	& 197	& 13	& 1	\\
	76	& 197	& 13	& 1	\\
	77	& 197	& 13	& 1	\\
	78	& 197	& 13	& 0	\\
	79	& 197	& 13	& 1	\\
	80	& 197	& 13	& 1	\\
	81	& 197	& 13	& 1	\\
	82	& 197	& 13	& 1	\\
	83	& 197	& 13	& 1	\\
	84	& 197	& 13	& 1	\\
	85	& 197	& 13	& 0	\\
	86	& 197	& 13	& 1	\\
	87	& 197	& 13	& 1	\\
	88	& 197	& 13	& 1	\\
	89	& 197	& 13	& 1	\\
	90	& 197	& 13	& 1	\\
	91	& 197	& 13	& 1	\\
	92	& 197	& 13	& 1	\\
	93	& 197	& 13	& 1	\\
	94	& 197	& 13	& 1	\\
	95	& 197	& 13	& 1	\\
	96	& 197	& 13	& 1	\\
	97	& 197	& 13	& 1	\\
	98	& 197	& 13	& 0	\\
	99	& 197	& 13	& 1	\\
	100	& 197	& 13	& 1	\\
\end{longtable}


\begin{longtable}[!]{c|ccc}
	\caption{Hasil pengujian kondisi normal pada data lapangan \textit{instance} tw04 dengan menggunakan sistem usulan (algoritma MDVRP berbasis CoEAs dan mekanisme Publish/Subscribe)}
	\label{tbl:test_result_field_tw04}\\
	\toprule
	Pengujian & \MyHead{3.1cm}{Total waktu pencacahan dari seluruh pencacah (hari)} & \MyHead{3.1cm}{Rata-rata waktu pencacahan dari setiap pencacah (hari)} & \MyHead{3.1cm}{Standar deviasi waktu pencacahan dari seluruh pencacah (hari)} \\ 
	\midrule
	\endfirsthead
	\toprule
	\textit{Pengujian} & \MyHead{3.1cm}{Total waktu pencacahan dari seluruh pencacah (hari)} & \MyHead{3.1cm}{Rata-rata waktu pencacahan dari setiap pencacah (hari)} & \MyHead{3.1cm}{Standar deviasi waktu pencacahan dari seluruh pencacah (hari)} \\ 
	\midrule
	\endhead
	\bottomrule
	\endfoot
	1	& 197	& 13	& 0	\\
	2	& 197	& 13	& 0	\\
	3	& 197	& 13	& 1	\\
	4	& 197	& 13	& 0	\\
	5	& 197	& 13	& 0	\\
	6	& 197	& 13	& 0	\\
	7	& 197	& 13	& 0	\\
	8	& 197	& 13	& 0	\\
	9	& 197	& 13	& 1	\\
	10	& 197	& 13	& 0	\\
	11	& 197	& 13	& 0	\\
	12	& 197	& 13	& 0	\\
	13	& 197	& 13	& 0	\\
	14	& 197	& 13	& 0	\\
	15	& 197	& 13	& 0	\\
	16	& 197	& 13	& 0	\\
	17	& 197	& 13	& 0	\\
	18	& 197	& 13	& 0	\\
	19	& 197	& 13	& 0	\\
	20	& 197	& 13	& 0	\\
	21	& 197	& 13	& 0	\\
	22	& 197	& 13	& 1	\\
	23	& 197	& 13	& 0	\\
	24	& 197	& 13	& 0	\\
	25	& 197	& 13	& 1	\\
	26	& 197	& 13	& 0	\\
	27	& 197	& 13	& 0	\\
	28	& 197	& 13	& 0	\\
	29	& 197	& 13	& 0	\\
	30	& 197	& 13	& 0	\\
	31	& 197	& 13	& 0	\\
	32	& 197	& 13	& 0	\\
	33	& 197	& 13	& 0	\\
	34	& 197	& 13	& 0	\\
	35	& 197	& 13	& 0	\\
	36	& 197	& 13	& 0	\\
	37	& 197	& 13	& 0	\\
	38	& 197	& 13	& 0	\\
	39	& 197	& 13	& 0	\\
	40	& 197	& 13	& 0	\\
	41	& 197	& 13	& 0	\\
	42	& 197	& 13	& 0	\\
	43	& 197	& 13	& 0	\\
	44	& 197	& 13	& 0	\\
	45	& 197	& 13	& 0	\\
	46	& 197	& 13	& 0	\\
	47	& 197	& 13	& 0	\\
	48	& 197	& 13	& 0	\\
	49	& 197	& 13	& 0	\\
	50	& 197	& 13	& 0	\\
	51	& 197	& 13	& 0	\\
	52	& 197	& 13	& 0	\\
	53	& 197	& 13	& 0	\\
	54	& 197	& 13	& 0	\\
	55	& 197	& 13	& 0	\\
	56	& 197	& 13	& 0	\\
	57	& 197	& 13	& 0	\\
	58	& 197	& 13	& 0	\\
	59	& 197	& 13	& 0	\\
	60	& 197	& 13	& 0	\\
	61	& 197	& 13	& 0	\\
	62	& 197	& 13	& 0	\\
	63	& 197	& 13	& 0	\\
	64	& 197	& 13	& 0	\\
	65	& 197	& 13	& 0	\\
	66	& 197	& 13	& 0	\\
	67	& 197	& 13	& 0	\\
	68	& 197	& 13	& 0	\\
	69	& 197	& 13	& 0	\\
	70	& 197	& 13	& 0	\\
	71	& 197	& 13	& 0	\\
	72	& 197	& 13	& 0	\\
	73	& 197	& 13	& 0	\\
	74	& 197	& 13	& 0	\\
	75	& 197	& 13	& 0	\\
	76	& 197	& 13	& 0	\\
	77	& 197	& 13	& 0	\\
	78	& 197	& 13	& 0	\\
	79	& 197	& 13	& 0	\\
	80	& 197	& 13	& 0	\\
	81	& 197	& 13	& 0	\\
	82	& 197	& 13	& 0	\\
	83	& 197	& 13	& 0	\\
	84	& 197	& 13	& 0	\\
	85	& 197	& 13	& 0	\\
	86	& 197	& 13	& 0	\\
	87	& 197	& 13	& 0	\\
	88	& 197	& 13	& 0	\\
	89	& 197	& 13	& 0	\\
	90	& 197	& 13	& 0	\\
	91	& 197	& 13	& 0	\\
	92	& 197	& 13	& 0	\\
	93	& 197	& 13	& 0	\\
	94	& 197	& 13	& 0	\\
	95	& 197	& 13	& 0	\\
	96	& 197	& 13	& 0	\\
	97	& 197	& 13	& 0	\\
	98	& 197	& 13	& 0	\\
	99	& 197	& 13	& 0	\\
	100	& 197	& 13	& 0	\\
\end{longtable}
	%-----------------------------------------------------------------------------%
\addChapter{LAMPIRAN X}
\chapter*{Lampiran X}
\label{ch:lampiran_hasil_pengujian_kondisi_delay}
%-----------------------------------------------------------------------------%


\begin{longtable}[!]{c|ccc}
	\caption{Hasil pengujian kondisi \textit{delay} pada data lapangan \textit{instance} d01 dengan menggunakan sistem usulan (algoritma MDVRP berbasis CoEAs dan mekanisme Publish/Subscribe)}
	\label{tbl:test_result_d01_tw}\\
	\toprule
	Pengujian & \MyHead{3.1cm}{Total waktu pencacahan dari seluruh pencacah (hari)} & \MyHead{3.1cm}{Rata-rata waktu pencacahan dari setiap pencacah (hari)} & \MyHead{3.1cm}{Standar deviasi waktu pencacahan dari seluruh pencacah (hari)} \\ 
	\midrule
	\endfirsthead
	\toprule
	\textit{Pengujian} & \MyHead{3.1cm}{Total waktu pencacahan dari seluruh pencacah (hari)} & \MyHead{3.1cm}{Rata-rata waktu pencacahan dari setiap pencacah (hari)} & \MyHead{3.1cm}{Standar deviasi waktu pencacahan dari seluruh pencacah (hari)} \\ 
	\midrule
	\endhead
	\bottomrule
	\endfoot
	1	& 223	& 14	& 3	\\
	2	& 221	& 14	& 2	\\
	3	& 219	& 14	& 3	\\
	4	& 223	& 14	& 1	\\
	5	& 230	& 15	& 1	\\
	6	& 223	& 14	& 1	\\
	7	& 222	& 14	& 1	\\
	8	& 227	& 15	& 1	\\
	9	& 229	& 15	& 1	\\
	10	& 221	& 14	& 1	\\
	11	& 224	& 14	& 2	\\
	12	& 215	& 14	& 1	\\
	13	& 234	& 15	& 1	\\
	14	& 220	& 14	& 1	\\
	15	& 217	& 14	& 1	\\
	16	& 229	& 15	& 1	\\
	17	& 219	& 14	& 1	\\
	18	& 226	& 15	& 1	\\
	19	& 233	& 15	& 1	\\
	20	& 219	& 14	& 1	\\
	21	& 218	& 14	& 1	\\
	22	& 220	& 14	& 1	\\
	23	& 215	& 14	& 1	\\
	24	& 220	& 14	& 1	\\
	25	& 220	& 14	& 1	\\
	26	& 218	& 14	& 1	\\
	27	& 221	& 14	& 1	\\
	28	& 224	& 14	& 1	\\
	29	& 217	& 14	& 1	\\
	30	& 227	& 15	& 1	\\
	31	& 222	& 14	& 1	\\
	32	& 226	& 15	& 1	\\
	33	& 234	& 15	& 2	\\
	34	& 219	& 14	& 1	\\
	35	& 224	& 14	& 1	\\
	36	& 229	& 15	& 1	\\
	37	& 224	& 14	& 1	\\
	38	& 222	& 14	& 1	\\
	39	& 224	& 14	& 2	\\
	40	& 218	& 14	& 1	\\
	41	& 224	& 14	& 1	\\
	42	& 221	& 14	& 1	\\
	43	& 235	& 15	& 1	\\
	44	& 223	& 14	& 1	\\
	45	& 218	& 14	& 1	\\
	46	& 234	& 15	& 1	\\
	47	& 217	& 14	& 1	\\
	48	& 224	& 14	& 1	\\
	49	& 224	& 14	& 1	\\
	50	& 219	& 14	& 1	\\
	51	& 223	& 14	& 1	\\
	52	& 221	& 14	& 1	\\
	53	& 227	& 15	& 1	\\
	54	& 221	& 14	& 1	\\
	55	& 222	& 14	& 1	\\
	56	& 226	& 15	& 1	\\
	57	& 224	& 14	& 1	\\
	58	& 223	& 14	& 1	\\
	59	& 231	& 15	& 1	\\
	60	& 220	& 14	& 1	\\
	61	& 221	& 14	& 1	\\
	62	& 228	& 15	& 1	\\
	63	& 217	& 14	& 1	\\
	64	& 226	& 15	& 1	\\
	65	& 227	& 15	& 1	\\
	66	& 225	& 15	& 1	\\
	67	& 224	& 14	& 1	\\
	68	& 225	& 15	& 2	\\
	69	& 227	& 15	& 1	\\
	70	& 223	& 14	& 1	\\
	71	& 234	& 15	& 1	\\
	72	& 227	& 15	& 1	\\
	73	& 227	& 15	& 1	\\
	74	& 221	& 14	& 2	\\
	75	& 226	& 15	& 1	\\
	76	& 225	& 15	& 1	\\
	77	& 221	& 14	& 1	\\
	78	& 229	& 15	& 1	\\
	79	& 226	& 15	& 1	\\
	80	& 231	& 15	& 1	\\
	81	& 233	& 15	& 1	\\
	82	& 222	& 14	& 1	\\
	83	& 231	& 15	& 1	\\
	84	& 225	& 15	& 1	\\
	85	& 221	& 14	& 1	\\
	86	& 227	& 15	& 1	\\
	87	& 221	& 14	& 1	\\
	88	& 224	& 14	& 2	\\
	89	& 218	& 14	& 1	\\
	90	& 225	& 15	& 1	\\
	91	& 222	& 14	& 1	\\
	92	& 228	& 15	& 1	\\
	93	& 220	& 14	& 1	\\
	94	& 223	& 14	& 1	\\
	95	& 213	& 14	& 0	\\
	96	& 219	& 14	& 1	\\
	97	& 218	& 14	& 2	\\
	98	& 230	& 15	& 1	\\
	99	& 227	& 15	& 1	\\
	100	& 225	& 15	& 1	\\
\end{longtable}


\begin{longtable}[!]{c|ccc}
	\caption{Hasil pengujian kondisi \textit{delay} pada data lapangan \textit{instance} d02 dengan menggunakan sistem usulan (algoritma MDVRP berbasis CoEAs dan mekanisme Publish/Subscribe)}
	\label{tbl:test_result_d02_tw}\\
	\toprule
	Pengujian & \MyHead{3.1cm}{Total waktu pencacahan dari seluruh pencacah (hari)} & \MyHead{3.1cm}{Rata-rata waktu pencacahan dari setiap pencacah (hari)} & \MyHead{3.1cm}{Standar deviasi waktu pencacahan dari seluruh pencacah (hari)} \\ 
	\midrule
	\endfirsthead
	\toprule
	\textit{Pengujian} & \MyHead{3.1cm}{Total waktu pencacahan dari seluruh pencacah (hari)} & \MyHead{3.1cm}{Rata-rata waktu pencacahan dari setiap pencacah (hari)} & \MyHead{3.1cm}{Standar deviasi waktu pencacahan dari seluruh pencacah (hari)} \\ 
	\midrule
	\endhead
	\bottomrule
	\endfoot
	1	& 219	& 14	& 3	\\
	2	& 226	& 15	& 3	\\
	3	& 220	& 14	& 4	\\
	4	& 222	& 14	& 1	\\
	5	& 216	& 14	& 1	\\
	6	& 219	& 14	& 1	\\
	7	& 223	& 14	& 1	\\
	8	& 225	& 15	& 1	\\
	9	& 218	& 14	& 1	\\
	10	& 217	& 14	& 1	\\
	11	& 223	& 14	& 1	\\
	12	& 218	& 14	& 1	\\
	13	& 221	& 14	& 1	\\
	14	& 219	& 14	& 1	\\
	15	& 227	& 15	& 1	\\
	16	& 226	& 15	& 1	\\
	17	& 225	& 15	& 1	\\
	18	& 226	& 15	& 1	\\
	19	& 228	& 15	& 1	\\
	20	& 220	& 14	& 2	\\
	21	& 225	& 15	& 1	\\
	22	& 220	& 14	& 1	\\
	23	& 221	& 14	& 1	\\
	24	& 223	& 14	& 1	\\
	25	& 228	& 15	& 1	\\
	26	& 229	& 15	& 1	\\
	27	& 230	& 15	& 1	\\
	28	& 228	& 15	& 2	\\
	29	& 224	& 14	& 1	\\
	30	& 225	& 15	& 1	\\
	31	& 223	& 14	& 1	\\
	32	& 218	& 14	& 1	\\
	33	& 221	& 14	& 1	\\
	34	& 220	& 14	& 2	\\
	35	& 224	& 14	& 1	\\
	36	& 220	& 14	& 1	\\
	37	& 225	& 15	& 1	\\
	38	& 223	& 14	& 1	\\
	39	& 226	& 15	& 1	\\
	40	& 226	& 15	& 1	\\
	41	& 226	& 15	& 1	\\
	42	& 218	& 14	& 1	\\
	43	& 223	& 14	& 1	\\
	44	& 232	& 15	& 2	\\
	45	& 222	& 14	& 1	\\
	46	& 224	& 14	& 1	\\
	47	& 217	& 14	& 1	\\
	48	& 226	& 15	& 1	\\
	49	& 228	& 15	& 1	\\
	50	& 220	& 14	& 1	\\
	51	& 234	& 15	& 1	\\
	52	& 218	& 14	& 1	\\
	53	& 225	& 15	& 2	\\
	54	& 227	& 15	& 2	\\
	55	& 225	& 15	& 2	\\
	56	& 219	& 14	& 3	\\
	57	& 221	& 14	& 3	\\
	58	& 228	& 15	& 3	\\
	59	& 229	& 15	& 3	\\
	60	& 221	& 14	& 3	\\
	61	& 219	& 14	& 1	\\
	62	& 223	& 14	& 1	\\
	63	& 222	& 14	& 1	\\
	64	& 223	& 14	& 1	\\
	65	& 214	& 14	& 1	\\
	66	& 217	& 14	& 1	\\
	67	& 222	& 14	& 1	\\
	68	& 225	& 15	& 1	\\
	69	& 226	& 15	& 0	\\
	70	& 228	& 15	& 1	\\
	71	& 220	& 14	& 1	\\
	72	& 223	& 14	& 1	\\
	73	& 224	& 14	& 1	\\
	74	& 226	& 15	& 1	\\
	75	& 216	& 14	& 1	\\
	76	& 217	& 14	& 1	\\
	77	& 231	& 15	& 1	\\
	78	& 224	& 14	& 1	\\
	79	& 226	& 15	& 1	\\
	80	& 219	& 14	& 1	\\
	81	& 227	& 15	& 1	\\
	82	& 217	& 14	& 0	\\
	83	& 213	& 14	& 1	\\
	84	& 226	& 15	& 1	\\
	85	& 223	& 14	& 2	\\
	86	& 224	& 14	& 2	\\
	87	& 224	& 14	& 1	\\
	88	& 222	& 14	& 1	\\
	89	& 219	& 14	& 1	\\
	90	& 227	& 15	& 1	\\
	91	& 232	& 15	& 1	\\
	92	& 220	& 14	& 1	\\
	93	& 214	& 14	& 1	\\
	94	& 225	& 15	& 1	\\
	95	& 227	& 15	& 1	\\
	96	& 225	& 15	& 1	\\
	97	& 219	& 14	& 1	\\
	98	& 229	& 15	& 2	\\
	99	& 220	& 14	& 1	\\
	100	& 225	& 15	& 1	\\
\end{longtable}


\begin{longtable}[!]{c|ccc}
	\caption{Hasil pengujian kondisi \textit{delay} pada data lapangan \textit{instance} d03 dengan menggunakan sistem usulan (algoritma MDVRP berbasis CoEAs dan mekanisme Publish/Subscribe)}
	\label{tbl:test_result_d03_tw}\\
	\toprule
	Pengujian & \MyHead{3.1cm}{Total waktu pencacahan dari seluruh pencacah (hari)} & \MyHead{3.1cm}{Rata-rata waktu pencacahan dari setiap pencacah (hari)} & \MyHead{3.1cm}{Standar deviasi waktu pencacahan dari seluruh pencacah (hari)} \\ 
	\midrule
	\endfirsthead
	\toprule
	\textit{Pengujian} & \MyHead{3.1cm}{Total waktu pencacahan dari seluruh pencacah (hari)} & \MyHead{3.1cm}{Rata-rata waktu pencacahan dari setiap pencacah (hari)} & \MyHead{3.1cm}{Standar deviasi waktu pencacahan dari seluruh pencacah (hari)} \\ 
	\midrule
	\endhead
	\bottomrule
	\endfoot
	1	& 364	& 24	& 4	\\
	2	& 364	& 24	& 3	\\
	3	& 364	& 24	& 5	\\
	4	& 364	& 24	& 1	\\
	5	& 364	& 24	& 1	\\
	6	& 364	& 24	& 1	\\
	7	& 364	& 24	& 1	\\
	8	& 364	& 24	& 1	\\
	9	& 364	& 24	& 1	\\
	10	& 364	& 24	& 1	\\
	11	& 364	& 24	& 2	\\
	12	& 364	& 24	& 1	\\
	13	& 364	& 24	& 1	\\
	14	& 364	& 24	& 1	\\
	15	& 364	& 24	& 1	\\
	16	& 364	& 24	& 1	\\
	17	& 364	& 24	& 1	\\
	18	& 364	& 24	& 1	\\
	19	& 364	& 24	& 1	\\
	20	& 364	& 24	& 1	\\
	21	& 364	& 24	& 1	\\
	22	& 364	& 24	& 1	\\
	23	& 364	& 24	& 1	\\
	24	& 364	& 24	& 2	\\
	25	& 364	& 24	& 1	\\
	26	& 364	& 24	& 1	\\
	27	& 364	& 24	& 1	\\
	28	& 364	& 24	& 1	\\
	29	& 364	& 24	& 1	\\
	30	& 364	& 24	& 1	\\
	31	& 364	& 24	& 1	\\
	32	& 364	& 24	& 1	\\
	33	& 364	& 24	& 1	\\
	34	& 364	& 24	& 1	\\
	35	& 364	& 24	& 1	\\
	36	& 364	& 24	& 1	\\
	37	& 364	& 24	& 1	\\
	38	& 364	& 24	& 1	\\
	39	& 364	& 24	& 1	\\
	40	& 364	& 24	& 1	\\
	41	& 364	& 24	& 1	\\
	42	& 364	& 24	& 2	\\
	43	& 364	& 24	& 1	\\
	44	& 364	& 24	& 1	\\
	45	& 364	& 24	& 1	\\
	46	& 364	& 24	& 1	\\
	47	& 364	& 24	& 2	\\
	48	& 364	& 24	& 1	\\
	49	& 364	& 24	& 1	\\
	50	& 364	& 24	& 2	\\
	51	& 364	& 24	& 1	\\
	52	& 364	& 24	& 1	\\
	53	& 364	& 24	& 1	\\
	54	& 364	& 24	& 1	\\
	55	& 364	& 24	& 1	\\
	56	& 364	& 24	& 1	\\
	57	& 364	& 24	& 1	\\
	58	& 364	& 24	& 1	\\
	59	& 364	& 24	& 2	\\
	60	& 364	& 24	& 1	\\
	61	& 364	& 24	& 2	\\
	62	& 364	& 24	& 1	\\
	63	& 364	& 24	& 1	\\
	64	& 364	& 24	& 1	\\
	65	& 364	& 24	& 1	\\
	66	& 364	& 24	& 1	\\
	67	& 364	& 24	& 1	\\
	68	& 364	& 24	& 1	\\
	69	& 364	& 24	& 1	\\
	70	& 364	& 24	& 1	\\
	71	& 364	& 24	& 1	\\
	72	& 364	& 24	& 1	\\
	73	& 364	& 24	& 1	\\
	74	& 364	& 24	& 1	\\
	75	& 364	& 24	& 1	\\
	76	& 364	& 24	& 1	\\
	77	& 364	& 24	& 1	\\
	78	& 364	& 24	& 1	\\
	79	& 364	& 24	& 1	\\
	80	& 364	& 24	& 1	\\
	81	& 364	& 24	& 1	\\
	82	& 364	& 24	& 1	\\
	83	& 364	& 24	& 1	\\
	84	& 364	& 24	& 0	\\
	85	& 364	& 24	& 1	\\
	86	& 364	& 24	& 1	\\
	87	& 364	& 24	& 1	\\
	88	& 364	& 24	& 1	\\
	89	& 364	& 24	& 1	\\
	90	& 364	& 24	& 1	\\
	91	& 364	& 24	& 1	\\
	92	& 364	& 24	& 1	\\
	93	& 364	& 24	& 1	\\
	94	& 364	& 24	& 1	\\
	95	& 364	& 24	& 1	\\
	96	& 364	& 24	& 1	\\
	97	& 364	& 24	& 1	\\
	98	& 364	& 24	& 1	\\
	99	& 364	& 24	& 1	\\
	100	& 364	& 24	& 1	\\
\end{longtable}


\begin{longtable}[!]{c|ccc}
	\caption{Hasil pengujian kondisi \textit{delay} pada data lapangan \textit{instance} d04 dengan menggunakan sistem usulan (algoritma MDVRP berbasis CoEAs dan mekanisme Publish/Subscribe)}
	\label{tbl:test_result_d04_tw}\\
	\toprule
	Pengujian & \MyHead{3.1cm}{Total waktu pencacahan dari seluruh pencacah (hari)} & \MyHead{3.1cm}{Rata-rata waktu pencacahan dari setiap pencacah (hari)} & \MyHead{3.1cm}{Standar deviasi waktu pencacahan dari seluruh pencacah (hari)} \\ 
	\midrule
	\endfirsthead
	\toprule
	\textit{Pengujian} & \MyHead{3.1cm}{Total waktu pencacahan dari seluruh pencacah (hari)} & \MyHead{3.1cm}{Rata-rata waktu pencacahan dari setiap pencacah (hari)} & \MyHead{3.1cm}{Standar deviasi waktu pencacahan dari seluruh pencacah (hari)} \\ 
	\midrule
	\endhead
	\bottomrule
	\endfoot
	1	& 224	& 14	& 2	\\
	2	& 227	& 15	& 2	\\
	3	& 220	& 14	& 2	\\
	4	& 228	& 15	& 1	\\
	5	& 230	& 15	& 3	\\
	6	& 224	& 14	& 1	\\
	7	& 224	& 14	& 1	\\
	8	& 220	& 14	& 1	\\
	9	& 226	& 15	& 1	\\
	10	& 217	& 14	& 1	\\
	11	& 230	& 15	& 1	\\
	12	& 214	& 14	& 1	\\
	13	& 224	& 14	& 1	\\
	14	& 227	& 15	& 1	\\
	15	& 216	& 14	& 1	\\
	16	& 223	& 14	& 2	\\
	17	& 226	& 15	& 1	\\
	18	& 217	& 14	& 1	\\
	19	& 229	& 15	& 2	\\
	20	& 218	& 14	& 1	\\
	21	& 213	& 14	& 1	\\
	22	& 221	& 14	& 1	\\
	23	& 213	& 14	& 1	\\
	24	& 222	& 14	& 1	\\
	25	& 229	& 15	& 1	\\
	26	& 232	& 15	& 1	\\
	27	& 225	& 15	& 1	\\
	28	& 221	& 14	& 1	\\
	29	& 226	& 15	& 1	\\
	30	& 217	& 14	& 1	\\
	31	& 221	& 14	& 1	\\
	32	& 223	& 14	& 1	\\
	33	& 234	& 15	& 1	\\
	34	& 221	& 14	& 1	\\
	35	& 230	& 15	& 1	\\
	36	& 222	& 14	& 1	\\
	37	& 223	& 14	& 1	\\
	38	& 226	& 15	& 1	\\
	39	& 218	& 14	& 1	\\
	40	& 220	& 14	& 2	\\
	41	& 229	& 15	& 1	\\
	42	& 224	& 14	& 1	\\
	43	& 221	& 14	& 1	\\
	44	& 228	& 15	& 1	\\
	45	& 222	& 14	& 1	\\
	46	& 219	& 14	& 1	\\
	47	& 227	& 15	& 1	\\
	48	& 226	& 15	& 1	\\
	49	& 216	& 14	& 1	\\
	50	& 220	& 14	& 2	\\
	51	& 220	& 14	& 1	\\
	52	& 223	& 14	& 1	\\
	53	& 229	& 15	& 1	\\
	54	& 238	& 15	& 1	\\
	55	& 220	& 14	& 1	\\
	56	& 229	& 15	& 1	\\
	57	& 220	& 14	& 1	\\
	58	& 230	& 15	& 1	\\
	59	& 224	& 14	& 1	\\
	60	& 222	& 14	& 1	\\
	61	& 227	& 15	& 1	\\
	62	& 224	& 14	& 1	\\
	63	& 229	& 15	& 2	\\
	64	& 225	& 15	& 1	\\
	65	& 218	& 14	& 1	\\
	66	& 218	& 14	& 1	\\
	67	& 222	& 14	& 1	\\
	68	& 224	& 14	& 1	\\
	69	& 231	& 15	& 1	\\
	70	& 225	& 15	& 1	\\
	71	& 222	& 14	& 1	\\
	72	& 216	& 14	& 1	\\
	73	& 226	& 15	& 1	\\
	74	& 227	& 15	& 2	\\
	75	& 232	& 15	& 2	\\
	76	& 232	& 15	& 1	\\
	77	& 221	& 14	& 1	\\
	78	& 222	& 14	& 1	\\
	79	& 233	& 15	& 1	\\
	80	& 221	& 14	& 1	\\
	81	& 222	& 14	& 1	\\
	82	& 216	& 14	& 1	\\
	83	& 218	& 14	& 1	\\
	84	& 218	& 14	& 1	\\
	85	& 211	& 14	& 1	\\
	86	& 221	& 14	& 1	\\
	87	& 219	& 14	& 1	\\
	88	& 225	& 15	& 1	\\
	89	& 231	& 15	& 1	\\
	90	& 225	& 15	& 1	\\
	91	& 219	& 14	& 1	\\
	92	& 225	& 15	& 1	\\
	93	& 215	& 14	& 1	\\
	94	& 221	& 14	& 1	\\
	95	& 222	& 14	& 2	\\
	96	& 222	& 14	& 1	\\
	97	& 236	& 15	& 1	\\
	98	& 231	& 15	& 2	\\
	99	& 220	& 14	& 1	\\
	100	& 209	& 13	& 1	\\
\end{longtable}


\begin{longtable}[!]{c|ccc}
	\caption{Hasil pengujian kondisi \textit{delay} pada data lapangan \textit{instance} d05 dengan menggunakan sistem usulan (algoritma MDVRP berbasis CoEAs dan mekanisme Publish/Subscribe)}
	\label{tbl:test_result_d05_tw}\\
	\toprule
	Pengujian & \MyHead{3.1cm}{Total waktu pencacahan dari seluruh pencacah (hari)} & \MyHead{3.1cm}{Rata-rata waktu pencacahan dari setiap pencacah (hari)} & \MyHead{3.1cm}{Standar deviasi waktu pencacahan dari seluruh pencacah (hari)} \\ 
	\midrule
	\endfirsthead
	\toprule
	\textit{Pengujian} & \MyHead{3.1cm}{Total waktu pencacahan dari seluruh pencacah (hari)} & \MyHead{3.1cm}{Rata-rata waktu pencacahan dari setiap pencacah (hari)} & \MyHead{3.1cm}{Standar deviasi waktu pencacahan dari seluruh pencacah (hari)} \\ 
	\midrule
	\endhead
	\bottomrule
	\endfoot
	1	& 364	& 24	& 4	\\
	2	& 364	& 24	& 3	\\
	3	& 364	& 24	& 1	\\
	4	& 364	& 24	& 1	\\
	5	& 364	& 24	& 1	\\
	6	& 364	& 24	& 1	\\
	7	& 364	& 24	& 1	\\
	8	& 364	& 24	& 1	\\
	9	& 364	& 24	& 1	\\
	10	& 364	& 24	& 1	\\
	11	& 364	& 24	& 2	\\
	12	& 364	& 24	& 1	\\
	13	& 364	& 24	& 1	\\
	14	& 364	& 24	& 1	\\
	15	& 364	& 24	& 1	\\
	16	& 364	& 24	& 1	\\
	17	& 364	& 24	& 1	\\
	18	& 364	& 24	& 1	\\
	19	& 364	& 24	& 1	\\
	20	& 364	& 24	& 1	\\
	21	& 364	& 24	& 1	\\
	22	& 364	& 24	& 2	\\
	23	& 364	& 24	& 3	\\
	24	& 364	& 24	& 1	\\
	25	& 364	& 24	& 2	\\
	26	& 364	& 24	& 1	\\
	27	& 364	& 24	& 1	\\
	28	& 364	& 24	& 1	\\
	29	& 364	& 24	& 0	\\
	30	& 364	& 24	& 1	\\
	31	& 364	& 24	& 1	\\
	32	& 364	& 24	& 1	\\
	33	& 364	& 24	& 2	\\
	34	& 364	& 24	& 1	\\
	35	& 364	& 24	& 1	\\
	36	& 364	& 24	& 2	\\
	37	& 364	& 24	& 1	\\
	38	& 364	& 24	& 1	\\
	39	& 364	& 24	& 1	\\
	40	& 364	& 24	& 2	\\
	41	& 364	& 24	& 2	\\
	42	& 364	& 24	& 1	\\
	43	& 364	& 24	& 1	\\
	44	& 364	& 24	& 1	\\
	45	& 364	& 24	& 1	\\
	46	& 364	& 24	& 1	\\
	47	& 364	& 24	& 1	\\
	48	& 364	& 24	& 1	\\
	49	& 364	& 24	& 1	\\
	50	& 364	& 24	& 1	\\
	51	& 364	& 24	& 1	\\
	52	& 364	& 24	& 1	\\
	53	& 364	& 24	& 1	\\
	54	& 364	& 24	& 1	\\
	55	& 364	& 24	& 1	\\
	56	& 364	& 24	& 1	\\
	57	& 364	& 24	& 1	\\
	58	& 364	& 24	& 1	\\
	59	& 364	& 24	& 1	\\
	60	& 364	& 24	& 1	\\
	61	& 364	& 24	& 1	\\
	62	& 364	& 24	& 1	\\
	63	& 364	& 24	& 1	\\
	64	& 364	& 24	& 1	\\
	65	& 364	& 24	& 2	\\
	66	& 364	& 24	& 1	\\
	67	& 364	& 24	& 1	\\
	68	& 364	& 24	& 1	\\
	69	& 364	& 24	& 1	\\
	70	& 364	& 24	& 1	\\
	71	& 364	& 24	& 1	\\
	72	& 364	& 24	& 1	\\
	73	& 364	& 24	& 1	\\
	74	& 364	& 24	& 1	\\
	75	& 364	& 24	& 1	\\
	76	& 364	& 24	& 1	\\
	77	& 364	& 24	& 1	\\
	78	& 364	& 24	& 1	\\
	79	& 364	& 24	& 1	\\
	80	& 364	& 24	& 1	\\
	81	& 364	& 24	& 1	\\
	82	& 364	& 24	& 1	\\
	83	& 364	& 24	& 1	\\
	84	& 364	& 24	& 1	\\
	85	& 364	& 24	& 1	\\
	86	& 364	& 24	& 1	\\
	87	& 364	& 24	& 1	\\
	88	& 364	& 24	& 1	\\
	89	& 364	& 24	& 1	\\
	90	& 364	& 24	& 1	\\
	91	& 364	& 24	& 1	\\
	92	& 364	& 24	& 1	\\
	93	& 364	& 24	& 1	\\
	94	& 364	& 24	& 1	\\
	95	& 364	& 24	& 1	\\
	96	& 364	& 24	& 1	\\
	97	& 364	& 24	& 1	\\
	98	& 364	& 24	& 1	\\
	99	& 364	& 24	& 1	\\
	100	& 364	& 24	& 1	\\
\end{longtable}


\begin{longtable}[!]{c|ccc}
	\caption{Hasil pengujian kondisi \textit{delay} pada data lapangan \textit{instance} d06 dengan menggunakan sistem usulan (algoritma MDVRP berbasis CoEAs dan mekanisme Publish/Subscribe)}
	\label{tbl:test_result_d06_tw}\\
	\toprule
	Pengujian & \MyHead{3.1cm}{Total waktu pencacahan dari seluruh pencacah (hari)} & \MyHead{3.1cm}{Rata-rata waktu pencacahan dari setiap pencacah (hari)} & \MyHead{3.1cm}{Standar deviasi waktu pencacahan dari seluruh pencacah (hari)} \\ 
	\midrule
	\endfirsthead
	\toprule
	\textit{Pengujian} & \MyHead{3.1cm}{Total waktu pencacahan dari seluruh pencacah (hari)} & \MyHead{3.1cm}{Rata-rata waktu pencacahan dari setiap pencacah (hari)} & \MyHead{3.1cm}{Standar deviasi waktu pencacahan dari seluruh pencacah (hari)} \\ 
	\midrule
	\endhead
	\bottomrule
	\endfoot
	1	& 364	& 24	& 3	\\
	2	& 364	& 24	& 4	\\
	3	& 364	& 24	& 4	\\
	4	& 364	& 24	& 2	\\
	5	& 364	& 24	& 1	\\
	6	& 364	& 24	& 1	\\
	7	& 364	& 24	& 1	\\
	8	& 364	& 24	& 1	\\
	9	& 364	& 24	& 1	\\
	10	& 364	& 24	& 1	\\
	11	& 364	& 24	& 2	\\
	12	& 364	& 24	& 2	\\
	13	& 364	& 24	& 2	\\
	14	& 364	& 24	& 2	\\
	15	& 364	& 24	& 1	\\
	16	& 364	& 24	& 1	\\
	17	& 364	& 24	& 1	\\
	18	& 364	& 24	& 2	\\
	19	& 364	& 24	& 2	\\
	20	& 364	& 24	& 1	\\
	21	& 364	& 24	& 2	\\
	22	& 364	& 24	& 2	\\
	23	& 364	& 24	& 2	\\
	24	& 364	& 24	& 1	\\
	25	& 364	& 24	& 2	\\
	26	& 364	& 24	& 1	\\
	27	& 364	& 24	& 1	\\
	28	& 364	& 24	& 1	\\
	29	& 364	& 24	& 1	\\
	30	& 364	& 24	& 1	\\
	31	& 364	& 24	& 1	\\
	32	& 364	& 24	& 2	\\
	33	& 364	& 24	& 2	\\
	34	& 364	& 24	& 2	\\
	35	& 364	& 24	& 2	\\
	36	& 364	& 24	& 2	\\
	37	& 364	& 24	& 1	\\
	38	& 364	& 24	& 1	\\
	39	& 364	& 24	& 2	\\
	40	& 364	& 24	& 1	\\
	41	& 364	& 24	& 2	\\
	42	& 364	& 24	& 2	\\
	43	& 364	& 24	& 1	\\
	44	& 364	& 24	& 1	\\
	45	& 364	& 24	& 1	\\
	46	& 364	& 24	& 1	\\
	47	& 364	& 24	& 1	\\
	48	& 364	& 24	& 1	\\
	49	& 364	& 24	& 1	\\
	50	& 364	& 24	& 1	\\
	51	& 364	& 24	& 1	\\
	52	& 364	& 24	& 1	\\
	53	& 364	& 24	& 2	\\
	54	& 364	& 24	& 2	\\
	55	& 364	& 24	& 1	\\
	56	& 364	& 24	& 1	\\
	57	& 364	& 24	& 2	\\
	58	& 364	& 24	& 1	\\
	59	& 364	& 24	& 1	\\
	60	& 364	& 24	& 1	\\
	61	& 364	& 24	& 2	\\
	62	& 364	& 24	& 2	\\
	63	& 364	& 24	& 2	\\
	64	& 364	& 24	& 2	\\
	65	& 364	& 24	& 1	\\
	66	& 364	& 24	& 1	\\
	67	& 364	& 24	& 1	\\
	68	& 364	& 24	& 1	\\
	69	& 364	& 24	& 2	\\
	70	& 364	& 24	& 1	\\
	71	& 364	& 24	& 1	\\
	72	& 364	& 24	& 2	\\
	73	& 364	& 24	& 2	\\
	74	& 364	& 24	& 1	\\
	75	& 364	& 24	& 1	\\
	76	& 364	& 24	& 2	\\
	77	& 364	& 24	& 1	\\
	78	& 364	& 24	& 1	\\
	79	& 364	& 24	& 2	\\
	80	& 364	& 24	& 1	\\
	81	& 364	& 24	& 1	\\
	82	& 364	& 24	& 1	\\
	83	& 364	& 24	& 2	\\
	84	& 364	& 24	& 2	\\
	85	& 364	& 24	& 1	\\
	86	& 364	& 24	& 2	\\
	87	& 364	& 24	& 1	\\
	88	& 364	& 24	& 2	\\
	89	& 364	& 24	& 1	\\
	90	& 364	& 24	& 1	\\
	91	& 364	& 24	& 1	\\
	92	& 364	& 24	& 1	\\
	93	& 364	& 24	& 1	\\
	94	& 364	& 24	& 1	\\
	95	& 364	& 24	& 1	\\
	96	& 364	& 24	& 1	\\
	97	& 364	& 24	& 2	\\
	98	& 364	& 24	& 1	\\
	99	& 364	& 24	& 3	\\
	100	& 364	& 24	& 2	\\
\end{longtable}


\begin{longtable}[!]{c|ccc}
	\caption{Hasil pengujian kondisi \textit{delay} pada data lapangan \textit{instance} d07 dengan menggunakan sistem usulan (algoritma MDVRP berbasis CoEAs dan mekanisme Publish/Subscribe)}
	\label{tbl:test_result_d07_tw}\\
	\toprule
	Pengujian & \MyHead{3.1cm}{Total waktu pencacahan dari seluruh pencacah (hari)} & \MyHead{3.1cm}{Rata-rata waktu pencacahan dari setiap pencacah (hari)} & \MyHead{3.1cm}{Standar deviasi waktu pencacahan dari seluruh pencacah (hari)} \\ 
	\midrule
	\endfirsthead
	\toprule
	\textit{Pengujian} & \MyHead{3.1cm}{Total waktu pencacahan dari seluruh pencacah (hari)} & \MyHead{3.1cm}{Rata-rata waktu pencacahan dari setiap pencacah (hari)} & \MyHead{3.1cm}{Standar deviasi waktu pencacahan dari seluruh pencacah (hari)} \\ 
	\midrule
	\endhead
	\bottomrule
	\endfoot
	1	& 364	& 24	& 3	\\
	2	& 364	& 24	& 4	\\
	3	& 364	& 24	& 4	\\
	4	& 364	& 24	& 1	\\
	5	& 364	& 24	& 1	\\
	6	& 364	& 24	& 1	\\
	7	& 364	& 24	& 1	\\
	8	& 364	& 24	& 1	\\
	9	& 364	& 24	& 1	\\
	10	& 364	& 24	& 1	\\
	11	& 364	& 24	& 1	\\
	12	& 364	& 24	& 1	\\
	13	& 364	& 24	& 2	\\
	14	& 364	& 24	& 1	\\
	15	& 364	& 24	& 1	\\
	16	& 364	& 24	& 1	\\
	17	& 364	& 24	& 2	\\
	18	& 364	& 24	& 1	\\
	19	& 364	& 24	& 1	\\
	20	& 364	& 24	& 2	\\
	21	& 364	& 24	& 1	\\
	22	& 364	& 24	& 1	\\
	23	& 364	& 24	& 1	\\
	24	& 364	& 24	& 1	\\
	25	& 364	& 24	& 1	\\
	26	& 364	& 24	& 1	\\
	27	& 364	& 24	& 2	\\
	28	& 364	& 24	& 1	\\
	29	& 364	& 24	& 1	\\
	30	& 364	& 24	& 1	\\
	31	& 364	& 24	& 1	\\
	32	& 364	& 24	& 1	\\
	33	& 364	& 24	& 1	\\
	34	& 364	& 24	& 1	\\
	35	& 364	& 24	& 1	\\
	36	& 364	& 24	& 1	\\
	37	& 364	& 24	& 1	\\
	38	& 364	& 24	& 1	\\
	39	& 364	& 24	& 1	\\
	40	& 364	& 24	& 1	\\
	41	& 364	& 24	& 2	\\
	42	& 364	& 24	& 2	\\
	43	& 364	& 24	& 1	\\
	44	& 364	& 24	& 1	\\
	45	& 364	& 24	& 1	\\
	46	& 364	& 24	& 2	\\
	47	& 364	& 24	& 1	\\
	48	& 364	& 24	& 2	\\
	49	& 364	& 24	& 2	\\
	50	& 364	& 24	& 1	\\
	51	& 364	& 24	& 1	\\
	52	& 364	& 24	& 1	\\
	53	& 364	& 24	& 2	\\
	54	& 364	& 24	& 1	\\
	55	& 364	& 24	& 1	\\
	56	& 364	& 24	& 1	\\
	57	& 364	& 24	& 1	\\
	58	& 364	& 24	& 1	\\
	59	& 364	& 24	& 1	\\
	60	& 364	& 24	& 1	\\
	61	& 364	& 24	& 1	\\
	62	& 364	& 24	& 1	\\
	63	& 364	& 24	& 1	\\
	64	& 364	& 24	& 1	\\
	65	& 364	& 24	& 1	\\
	66	& 364	& 24	& 1	\\
	67	& 364	& 24	& 1	\\
	68	& 364	& 24	& 1	\\
	69	& 364	& 24	& 1	\\
	70	& 364	& 24	& 2	\\
	71	& 364	& 24	& 1	\\
	72	& 364	& 24	& 1	\\
	73	& 364	& 24	& 1	\\
	74	& 364	& 24	& 1	\\
	75	& 364	& 24	& 1	\\
	76	& 364	& 24	& 1	\\
	77	& 364	& 24	& 1	\\
	78	& 364	& 24	& 1	\\
	79	& 364	& 24	& 1	\\
	80	& 364	& 24	& 1	\\
	81	& 364	& 24	& 1	\\
	82	& 364	& 24	& 1	\\
	83	& 364	& 24	& 1	\\
	84	& 364	& 24	& 1	\\
	85	& 364	& 24	& 1	\\
	86	& 364	& 24	& 1	\\
	87	& 364	& 24	& 1	\\
	88	& 364	& 24	& 1	\\
	89	& 364	& 24	& 1	\\
	90	& 364	& 24	& 1	\\
	91	& 364	& 24	& 1	\\
	92	& 364	& 24	& 1	\\
	93	& 364	& 24	& 1	\\
	94	& 364	& 24	& 1	\\
	95	& 364	& 24	& 1	\\
	96	& 364	& 24	& 2	\\
	97	& 364	& 24	& 1	\\
	98	& 364	& 24	& 1	\\
	99	& 364	& 24	& 1	\\
	100	& 364	& 24	& 1	\\
\end{longtable}
\end{appendices}

\end{document}