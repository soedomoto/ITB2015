%---------------------------------------------------------------
\chapter{\kesimpulan}
%---------------------------------------------------------------
%\todo{Tambahkan kesimpulan dan saran terkait dengan perkerjaan 
%	yang dilakukan.}


%---------------------------------------------------------------
\section{Kesimpulan}
%---------------------------------------------------------------
Berdasarkan hasil pengujian, diperoleh kesimpulan sebagai berikut:

\begin{enumerate}
	\item Pada permasalahan MDVRP yang tidak mempertimbangkan \textit{service time}, dari pengujian dengan data Cordeau diperoleh hasil bahwa dengan menggunakan sistem usulan 8 dari 10 \textit{instance} menghasilkan \textbf{total waktu} yang lebih lama, atau 80 persen lebih buruk dibandingkan hasil pengujian dengan menggunakan program pembanding dengan rata-rata perbedaan 35 persen. Sementara dari segi \textbf{standar deviasi} diperoleh hasil bahwa sistem usulan menghasilkan standar deviasi yang lebih rendah pada 8 dari 10 \textit{instance}, atau 80 persen lebih baik dibandingkan program pembanding dengan rata-rata perbedaan 53 persen. Dengan demikian dapat disimpulkan algoritma MDVRP berbasis CoEAs dengan mekanisme Publish/Subscribe memberikan hasil yang \textbf{lebih merata} antar rute yang dihasilkan dengan selisih \textit{total waktu} yang rendah.
	
	\item Pada permasalahan MDVRP yang mempertimbangkan \textit{service time}, dari pengujian dengan data Cordeau diperoleh hasil bahwa dengan menggunakan sistem usulan 9 dari 10 \textit{instance} menghasilkan \textbf{total waktu} yang lebih lama, atau 90 persen lebih buruk dibandingkan hasil pengujian dengan menggunakan program pembanding dengan rata-rata perbedaan sebesar 0,023 persen. Sementara dari segi \textbf{standar deviasi} diperoleh hasil bahwa sistem usulan menghasilkan standar deviasi yang lebih rendah pada seluruh \textit{instance}, atau 100 persen lebih baik dibandingkan program pembanding, dengan rata-rata perbedaan lebih dari 11 kali lipat. Sementara pada pengujian dengan data lapangan diperoleh hasil bahwa keseluruhan \textit{instance} memberikan hasil yang lebih buruk dari segi \textbf{total waktu} dengan rata-rata perbedaan sebesar 5,87 persen, tetapi lebih baik dari segi \textbf{standar deviasi} pada seluruh \textit{instance} dengan rata-rata perbedaan lebih dari 5 kali lipat. Dengan demikian dapat disimpulkan algoritma MDVRP berbasis CoEAs dengan mekanisme Publish/Subscribe memberikan hasil yang \textbf{lebih merata} antar rute yang dihasilkan dengan selisih \textit{total waktu} yang rendah.
	
	\item Pada permasalahan MDVRP yang mempertimbangkan \textit{service time} dan \textit{delay}, dari pengujian dengan data lapangan diperoleh hasil bahwa dengan menggunakan sistem usulan pada keseluruhan \textit{instance} menghasilkan \textbf{total waktu} yang lebih lama, atau 100 persen lebih buruk dibandingkan hasil pengujian dengan menggunakan program pembanding, dengan rata-rata perbedaan 16,19 persen. Akan tetapi dari segi \textbf{standar deviasi} diperoleh hasil bahwa sistem usulan menghasilkan standar deviasi yang lebih rendah pada seluruh \textit{instance}, atau 100 persen lebih baik dibandingkan program pembanding, dengan rata-rata perbedaan lebih dari 5 kali lipat. Dengan demikian dapat disimpulkan algoritma MDVRP berbasis CoEAs dengan mekanisme Publish/Subscribe memberikan hasil yang \textbf{lebih merata} antar rute yang dihasilkan dengan selisih \textit{total waktu} yang rendah.
	
	\item Dari seluruh pengujian dengan data dan skenario yang bervariasi, dapat disimpulkan bahwa algoritma MDVRP berbasis CoEAs dengan mekanisme Publish/Subscribe memberikan hasil yang \textbf{lebih merata} antar rute yang dihasilkan dengan selisih \textit{total waktu} yang rendah.
\end{enumerate}


%---------------------------------------------------------------
\section{Saran}
%---------------------------------------------------------------
Dari penelitian ini, saran yang diberikan untuk penelitian selanjutnya sebagai berikut:

\begin{enumerate}
	\item Mengkaji mekanisme komunikasi yang digunakan, misalnya dengan membandingkan dengan mekanisme Request/Reply dan Push/Pull untuk mendapatkan mekanisme komunikasi yang paling tepat.
	\item Mengkaji parameter yang dapat digunakan dalam VRP Solver yaitu lama waktu eksekusi dan lama waktu dimana tidak ada perubahan solusi terbaik, sehingga diperoleh waktu eksekusi yang paling optimal.
	\item Mengkasi penggunaan algoritma MDVRP yang lain, misalnya Tabu Search, Particle Swarm Optimization, dan lain lain, sehingga diperoleh algoritma MDVRP yang paling sesuai.
\end{enumerate}
