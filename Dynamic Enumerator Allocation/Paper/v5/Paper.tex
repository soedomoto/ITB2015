
%% bare_conf.tex
%% V1.4b
%% 2015/08/26
%% by Michael Shell
%% See:
%% http://www.michaelshell.org/
%% for current contact information.
%%
%% This is a skeleton file demonstrating the use of IEEEtran.cls
%% (requires IEEEtran.cls version 1.8b or later) with an IEEE
%% conference paper.
%%
%% Support sites:
%% http://www.michaelshell.org/tex/ieeetran/
%% http://www.ctan.org/pkg/ieeetran
%% and
%% http://www.ieee.org/

%%*************************************************************************
%% Legal Notice:
%% This code is offered as-is without any warranty either expressed or
%% implied; without even the implied warranty of MERCHANTABILITY or
%% FITNESS FOR A PARTICULAR PURPOSE! 
%% User assumes all risk.
%% In no event shall the IEEE or any contributor to this code be liable for
%% any damages or losses, including, but not limited to, incidental,
%% consequential, or any other damages, resulting from the use or misuse
%% of any information contained here.
%%
%% All comments are the opinions of their respective authors and are not
%% necessarily endorsed by the IEEE.
%%
%% This work is distributed under the LaTeX Project Public License (LPPL)
%% ( http://www.latex-project.org/ ) version 1.3, and may be freely used,
%% distributed and modified. A copy of the LPPL, version 1.3, is included
%% in the base LaTeX documentation of all distributions of LaTeX released
%% 2003/12/01 or later.
%% Retain all contribution notices and credits.
%% ** Modified files should be clearly indicated as such, including  **
%% ** renaming them and changing author support contact information. **
%%*************************************************************************


% *** Authors should verify (and, if needed, correct) their LaTeX system  ***
% *** with the testflow diagnostic prior to trusting their LaTeX platform ***
% *** with production work. The IEEE's font choices and paper sizes can   ***
% *** trigger bugs that do not appear when using other class files.       ***                          ***
% The testflow support page is at:
% http://www.michaelshell.org/tex/testflow/



\documentclass[conference]{IEEEtran}
% Some Computer Society conferences also require the compsoc mode option,
% but others use the standard conference format.
%
% If IEEEtran.cls has not been installed into the LaTeX system files,
% manually specify the path to it like:
% \documentclass[conference]{../sty/IEEEtran}





% Some very useful LaTeX packages include:
% (uncomment the ones you want to load)


% *** MISC UTILITY PACKAGES ***
%
%\usepackage{ifpdf}
% Heiko Oberdiek's ifpdf.sty is very useful if you need conditional
% compilation based on whether the output is pdf or dvi.
% usage:
% \ifpdf
%   % pdf code
% \else
%   % dvi code
% \fi
% The latest version of ifpdf.sty can be obtained from:
% http://www.ctan.org/pkg/ifpdf
% Also, note that IEEEtran.cls V1.7 and later provides a builtin
% \ifCLASSINFOpdf conditional that works the same way.
% When switching from latex to pdflatex and vice-versa, the compiler may
% have to be run twice to clear warning/error messages.






% *** CITATION PACKAGES ***
%
\usepackage{cite}
% cite.sty was written by Donald Arseneau
% V1.6 and later of IEEEtran pre-defines the format of the cite.sty package
% \cite{} output to follow that of the IEEE. Loading the cite package will
% result in citation numbers being automatically sorted and properly
% "compressed/ranged". e.g., [1], \cite{rubinsztejn_framework_2005}, \cite{gamma_design_1995}, \cite{holder_system_1999}, \cite{pratikakis_transparent_2004}, \cite{_fusion_????} without using
% cite.sty will become [1], \cite{gamma_design_1995}, \cite{pratikakis_transparent_2004}--\cite{holder_system_1999}, \cite{rubinsztejn_framework_2005} using cite.sty. cite.sty's
% \cite will automatically add leading space, if needed. Use cite.sty's
% noadjust option (cite.sty V3.8 and later) if you want to turn this off
% such as if a citation ever needs to be enclosed in parenthesis.
% cite.sty is already installed on most LaTeX systems. Be sure and use
% version 5.0 (2009-03-20) and later if using hyperref.sty.
% The latest version can be obtained at:
% http://www.ctan.org/pkg/cite
% The documentation is contained in the cite.sty file itself.






% *** GRAPHICS RELATED PACKAGES ***
%
\ifCLASSINFOpdf
  \usepackage[pdftex]{graphicx}
  % declare the path(s) where your graphic files are
  % \graphicspath{{../pdf/}{../jpeg/}}
  % and their extensions so you won't have to specify these with
  % every instance of \includegraphics
  \DeclareGraphicsExtensions{.pdf,.jpeg,.png}
\else
  % or other class option (dvipsone, dvipdf, if not using dvips). graphicx
  % will default to the driver specified in the system graphics.cfg if no
  % driver is specified.
  \usepackage[dvips]{graphicx}
  % declare the path(s) where your graphic files are
  % \graphicspath{{../eps/}}
  % and their extensions so you won't have to specify these with
  % every instance of \includegraphics
  \DeclareGraphicsExtensions{.eps}
\fi
% graphicx was written by David Carlisle and Sebastian Rahtz. It is
% required if you want graphics, photos, etc. graphicx.sty is already
% installed on most LaTeX systems. The latest version and documentation
% can be obtained at: 
% http://www.ctan.org/pkg/graphicx
% Another good source of documentation is "Using Imported Graphics in
% LaTeX2e" by Keith Reckdahl which can be found at:
% http://www.ctan.org/pkg/epslatex
%
% latex, and pdflatex in dvi mode, support graphics in encapsulated
% postscript (.eps) format. pdflatex in pdf mode supports graphics
% in .pdf, .jpeg, .png and .mps (metapost) formats. Users should ensure
% that all non-photo figures use a vector format (.eps, .pdf, .mps) and
% not a bitmapped formats (.jpeg, .png). The IEEE frowns on bitmapped formats
% which can result in "jaggedy"/blurry rendering of lines and letters as
% well as large increases in file sizes.
%
% You can find documentation about the pdfTeX application at:
% http://www.tug.org/applications/pdftex





% *** MATH PACKAGES ***
%
\usepackage[fleqn]{amsmath}
% A popular package from the American Mathematical Society that provides
% many useful and powerful commands for dealing with mathematics.
%
% Note that the amsmath package sets \interdisplaylinepenalty to 10000
% thus preventing page breaks from occurring within multiline equations. Use:
%\interdisplaylinepenalty=2500
% after loading amsmath to restore such page breaks as IEEEtran.cls normally
% does. amsmath.sty is already installed on most LaTeX systems. The latest
% version and documentation can be obtained at:
% http://www.ctan.org/pkg/amsmath




% *** MATH PACKAGES ***
%
\usepackage{amssymb}





% *** SPECIALIZED LIST PACKAGES ***
%
\usepackage{algorithmic}
% algorithmic.sty was written by Peter Williams and Rogerio Brito.
% This package provides an algorithmic environment fo describing algorithms.
% You can use the algorithmic environment in-text or within a figure
% environment to provide for a floating algorithm. Do NOT use the algorithm
% floating environment provided by algorithm.sty (by the same authors) or
% algorithm2e.sty (by Christophe Fiorio) as the IEEE does not use dedicated
% algorithm float types and packages that provide these will not provide
% correct IEEE style captions. The latest version and documentation of
% algorithmic.sty can be obtained at:
% http://www.ctan.org/pkg/algorithms
% Also of interest may be the (relatively newer and more customizable)
% algorithmicx.sty package by Szasz Janos:
% http://www.ctan.org/pkg/algorithmicx




% *** ALIGNMENT PACKAGES ***
%
\usepackage{array}
% Frank Mittelbach's and David Carlisle's array.sty patches and improves
% the standard LaTeX2e array and tabular environments to provide better
% appearance and additional user controls. As the default LaTeX2e table
% generation code is lacking to the point of almost being broken with
% respect to the quality of the end results, all users are strongly
% advised to use an enhanced (at the very least that provided by array.sty)
% set of table tools. array.sty is already installed on most systems. The
% latest version and documentation can be obtained at:
% http://www.ctan.org/pkg/array


% IEEEtran contains the IEEEeqnarray family of commands that can be used to
% generate multiline equations as well as matrices, tables, etc., of high
% quality.




% *** SUBFIGURE PACKAGES ***
%\ifCLASSOPTIONcompsoc
%  \usepackage[caption=false,font=normalsize,labelfont=sf,textfont=sf]{subfig}
%\else
%  \usepackage[caption=false,font=footnotesize]{subfig}
%\fi
% subfig.sty, written by Steven Douglas Cochran, is the modern replacement
% for subfigure.sty, the latter of which is no longer maintained and is
% incompatible with some LaTeX packages including fixltx2e. However,
% subfig.sty requires and automatically loads Axel Sommerfeldt's caption.sty
% which will override IEEEtran.cls' handling of captions and this will result
% in non-IEEE style figure/table captions. To prevent this problem, be sure
% and invoke subfig.sty's "caption=false" package option (available since
% subfig.sty version 1.3, 2005/06/28) as this is will preserve IEEEtran.cls
% handling of captions.
% Note that the Computer Society format requires a larger sans serif font
% than the serif footnote size font used in traditional IEEE formatting
% and thus the need to invoke different subfig.sty package options depending
% on whether compsoc mode has been enabled.
%
% The latest version and documentation of subfig.sty can be obtained at:
% http://www.ctan.org/pkg/subfig
\usepackage{subcaption}




% *** FLOAT PACKAGES ***
%
\usepackage{fixltx2e}
% fixltx2e, the successor to the earlier fix2col.sty, was written by
% Frank Mittelbach and David Carlisle. This package corrects a few problems
% in the LaTeX2e kernel, the most notable of which is that in current
% LaTeX2e releases, the ordering of single and double column floats is not
% guaranteed to be preserved. Thus, an unpatched LaTeX2e can allow a
% single column figure to be placed prior to an earlier double column
% figure.
% Be aware that LaTeX2e kernels dated 2015 and later have fixltx2e.sty's
% corrections already built into the system in which case a warning will
% be issued if an attempt is made to load fixltx2e.sty as it is no longer
% needed.
% The latest version and documentation can be found at:
% http://www.ctan.org/pkg/fixltx2e


\usepackage{stfloats}
% stfloats.sty was written by Sigitas Tolusis. This package gives LaTeX2e
% the ability to do double column floats at the bottom of the page as well
% as the top. (e.g., "\begin{figure*}[!b]" is not normally possible in
% LaTeX2e). It also provides a command:
%\fnbelowfloat
% to enable the placement of footnotes below bottom floats (the standard
% LaTeX2e kernel puts them above bottom floats). This is an invasive package
% which rewrites many portions of the LaTeX2e float routines. It may not work
% with other packages that modify the LaTeX2e float routines. The latest
% version and documentation can be obtained at:
% http://www.ctan.org/pkg/stfloats
% Do not use the stfloats baselinefloat ability as the IEEE does not allow
% \baselineskip to stretch. Authors submitting work to the IEEE should note
% that the IEEE rarely uses double column equations and that authors should try
% to avoid such use. Do not be tempted to use the cuted.sty or midfloat.sty
% packages (also by Sigitas Tolusis) as the IEEE does not format its papers in
% such ways.
% Do not attempt to use stfloats with fixltx2e as they are incompatible.
% Instead, use Morten Hogholm'a dblfloatfix which combines the features
% of both fixltx2e and stfloats:
%
% \usepackage{dblfloatfix}
% The latest version can be found at:
% http://www.ctan.org/pkg/dblfloatfix




% *** PDF, URL AND HYPERLINK PACKAGES ***
%
\usepackage{url}
% url.sty was written by Donald Arseneau. It provides better support for
% handling and breaking URLs. url.sty is already installed on most LaTeX
% systems. The latest version and documentation can be obtained at:
% http://www.ctan.org/pkg/url
% Basically, \url{my_url_here}.




%
%
\usepackage[bookmarks=false]{hyperref}
\newcommand{\algorithmautorefname}{Algorithm}




%
%
\usepackage{minted}
\usepackage{tcolorbox}
\usepackage{etoolbox}
\BeforeBeginEnvironment{minted}{\begin{tcolorbox}}%
\AfterEndEnvironment{minted}{\end{tcolorbox}}%
\AtBeginEnvironment{minted}{\fontsize{8}{8}}




%
% Digunakan untuk membuat inline list dan numbering
\usepackage{paralist}




%
% 
\usepackage{algorithm}
\usepackage{algorithmic}




%
% Digunakan untuk membuat tabel yang panjang, lebih dari 1 halaman
%
\usepackage{longtable, supertabular, booktabs}
  \newcommand{\ra}[1]{\renewcommand{\arraystretch}{#1}}




% *** Do not adjust lengths that control margins, column widths, etc. ***
% *** Do not use packages that alter fonts (such as pslatex).         ***
% There should be no need to do such things with IEEEtran.cls V1.6 and later.
% (Unless specifically asked to do so by the journal or conference you plan
% to submit to, of course. )


% correct bad hyphenation here
\hyphenation{op-tical net-works semi-conduc-tor}


\begin{document}
%
% paper title
% Titles are generally capitalized except for words such as a, an, and, as,
% at, but, by, for, in, nor, of, on, or, the, to and up, which are usually
% not capitalized unless they are the first or last word of the title.
% Linebreaks \\ can be used within to get better formatting as desired.
% Do not put math or special symbols in the title.
\title{Real-Time Location Recommendation System for Field Data Collection}


% author names and affiliations
% use a multiple column layout for up to three different
% affiliations
\author{
\IEEEauthorblockN{Aris Prawisudatama}
\IEEEauthorblockA{School of Electrical Engineering and Informatics\\
Institut Teknologi Bandung\\
Bandung, Indonesia\\
Email: soedomoto@gmail.com}
\and
\IEEEauthorblockN{I Gusti Bagus Baskara Nugraha}
\IEEEauthorblockA{School of Electrical Engineering and Informatics\\
Institut Teknologi Bandung\\
Bandung, Indonesia\\
Email: baskara@stei.itb.ac.id}
}

% conference papers do not typically use \thanks and this command
% is locked out in conference mode. If really needed, such as for
% the acknowledgment of grants, issue a \IEEEoverridecommandlockouts
% after \documentclass

% for over three affiliations, or if they all won't fit within the width
% of the page, use this alternative format:
% 
%\author{\IEEEauthorblockN{Michael Shell\IEEEauthorrefmark{1},
%Homer Simpson\IEEEauthorrefmark{2},
%James Kirk\IEEEauthorrefmark{3}, 
%Montgomery Scott\IEEEauthorrefmark{3} and
%Eldon Tyrell\IEEEauthorrefmark{4}}
%\IEEEauthorblockA{\IEEEauthorrefmark{1}School of Electrical and Computer Engineering\\
%Georgia Institute of Technology,
%Atlanta, Georgia 30332--0250\\ Email: see http://www.michaelshell.org/contact.html}
%\IEEEauthorblockA{\IEEEauthorrefmark{2}Twentieth Century Fox, Springfield, USA\\
%Email: homer@thesimpsons.com}
%\IEEEauthorblockA{\IEEEauthorrefmark{3}Starfleet Academy, San Francisco, California 96678-2391\\
%Telephone: (800) 555--1212, Fax: (888) 555--1212}
%\IEEEauthorblockA{\IEEEauthorrefmark{4}Tyrell Inc., 123 Replicant Street, Los Angeles, California 90210--4321}}




% use for special paper notices
%\IEEEspecialpapernotice{(Invited Paper)}




% make the title area
\maketitle

% As a general rule, do not put math, special symbols or citations
% in the abstract
\begin{abstract}
Field data collection is one of the main activities performed by national statistical agencies in every country. Data collection activities have a similar workflow with Multi-Depot Vehicle Routing Problem (MDVRP). The use of MDVRP to generate pre-calculated routes resulted in total route costs with high standard deviation. The real-time mechanism by utilizing the publish/subscribe paradigm combined with MDVRP based on Cooperative Coevolution Algorithms (CoEAs) is proposed to reduce the inequality (large variation) of the completion time. The test results show that routes produced by the combination of publish/subscribe paradigm and CoEAs are more prevalent in enumerator's total route times compared with the pre-calculated routes produced by MDVRP based on CoEAs only.
\end{abstract}

% no keywords




% For peer review papers, you can put extra information on the cover
% page as needed:
% \ifCLASSOPTIONpeerreview
% \begin{center} \bfseries EDICS Category: 3-BBND \end{center}
% \fi
%
% For peerreview papers, this IEEEtran command inserts a page break and
% creates the second title. It will be ignored for other modes.
\IEEEpeerreviewmaketitle




%-----------------------------------------------------------------------------%
\section{Introduction}
\label{sec:introduction}
%-----------------------------------------------------------------------------%
Data collection is one of the main duties of a statistical agency in every country. There are two data collection methods that are generally used: primary and secondary data collection. The primary data collection requires a direct interview with the respondents, while the secondary data collection only compiles the data collected by other agencies. Related to this, the primary data collection involves an activity of allocating the enumerators to several census/survey areas. The allocation is usually conducted by assigning an equal number of locations to each enumerator.

The workflow of the data collection performs as follows: 
\begin{enumerate}
	\item Each enumerator gets a list of census/survey locations to visit. 
	\item The enumeration starts on one particular point. It could be the statistical office or the enumerator's house. 
	\item The enumerator moves to the first census/survey location and visits all respondents in that location. 
	\item After finishing the first census/survey area, the enumerator moves to the second census/survey location and continue doing so until all locations have been visited successfully.
\end{enumerate}
This workflow is significantly similar to the Vehicle Routing Problem (VRP), specifically, the Multi-Depot Vehicle Routing Problem (MDVRP). The enumerators are analogous to vehicles, census/survey locations are comparable with customers, and the enumeration's starting point is corresponding with the depot. 

Although sharing similar workflow with MDVRP, the rough implementation of the MDVRP algorithm is not feasible for solving the allocation problems. Solutions or routes obtained using the MDVRP algorithm often result in uneven time of completion. This is because the MDVRP algorithm when used in the search solution uses the assumption that service time at each location is evenly distributed, whereas the facts vary greatly.

The inclusion of service time variables that match the field conditions in the search for a solution is not possible, because variable service time will not be available until after the enumerator finishes visiting the census/survey location. Therefore the MDVRP algorithm needs to be integrated with a communication mechanism to create a real-time location recommendation system for field data collection. This study aims to propose a system that integrates the MDVRP algorithm based on CoEAs and publish/subscribe mechanism. 


%-----------------------------------------------------------------------------%
\section{Related Works}
\label{sec:related-works}
%-----------------------------------------------------------------------------%

%-----------------------------------------------------------------------------%
\subsection{MDVRP based on Coevolution Algorithms (CoEAs)}
\label{ssec:evolution-algorithms}
%-----------------------------------------------------------------------------%
VRP is an optimization problem that focused on the query: \textit{`given a set of vehicles and a starting point (depot), find the best route to visit all customers'}. In cases there is more than one depot, VRP is known as MDVRP \cite{montoya-torres_literature_2015}. \autoref{fig:mdvrp-illustration} shows an example of an MDVRP solution that uses two depots and two routes associated with each depot. Basically, a solution for MDVRP is a set of routes such that: (i) each route starts and ends at the same depot, (ii) each customer is only served once by one vehicle, (iii) the total demand on each route does not exceed vehicle's capacity, (iv) the maximum route time is satisfied, and (v) the total cost is minimized.

\begin{figure}[!]
	\centering
	\includegraphics[width=8cm]{Resources/Images/mdvrp-illustration}
	\caption{An illustration for \textit{Multi Depot} VRP}
	\label{fig:mdvrp-illustration}
\end{figure}

Coevolutionary algorithms (CoEAs) is one of many algorithms \cite{cordeau_tabu_1997, pisinger_general_2007, lau_application_2010, cordeau_parallel_2012, subramanian_hybrid_2013, vidal_implicit_2014, escobar_hybrid_2014, de_oliveira_cooperative_2016} that can be used to solve MDVRP. CoEAs realize that in natural evolution the physical environment is influenced by other independently-acting biological populations \cite{engelbrecht_coevolution_2007}. Based on the interaction of each species, CoEAs can be distinguished into two categories: competitive and cooperative \cite{engelbrecht_coevolution_2007}. In the competitive coevolution, each individual competes with other individuals in the same group. Meanwhile, the cooperative coevolution has its species mutually interacted or, at least, not harming each other.

The use of CoEAs for solving MDVRP has been studied by several researchers, one of which is de Oliveira et al. \cite{de_oliveira_cooperative_2016}. De Oliveira's algorithm consists of 2 (two) stages: initiation and evolution. In initiation stage, the population is built using Nearest Insertion Heuristic (NIH) and NIH-based semi-greedy. Then, in the evolution stage, this population will be evolved using a parallel module, which consists of 3 submodules: Population Evolve (PE), Complete Solution Evaluation (CSE), and Elite Group (EG) submodules. This stage is controlled by a monitor module to ensure that all processes run well. \autoref{fig:coes_paralel_modules} shows the parallel architecture proposed by de Oliveira.


\begin{figure}[!]
	\centering
	\includegraphics[width=6cm]{Resources/Images/coes_paralel_modules}
	\caption{Architecture of Parallel Modules \cite{de_oliveira_cooperative_2016}}
	\label{fig:coes_paralel_modules}
\end{figure}


%-----------------------------------------------------------------------------%
\subsection{Publish/Subscribe Paradigm}
\label{ssec:pub-sub}
%-----------------------------------------------------------------------------%
The publish/subscribe interaction is a communication pattern between publisher (server) and subscribers (clients). The subscribers are the parties who have interests in a certain event/topic and will get a notification about the event/topic they are interested in \cite{eugster_many_2003}. One of the benefits of using publish/subscribe mechanism is the \textit{loose coupling} characteristic \cite{eugster_many_2003} between the publisher and the subscriber.

The basic model of the publish/subscribe system relies on the event notification service that provides storage and management of subscription. This service acts as a mediator between the publisher (the event's producer) and the subscriber (the event's consumer).


%-----------------------------------------------------------------------------%
\section{Proposed Solution}
\label{sec:proposed-solution}
%-----------------------------------------------------------------------------%
Using pure MDVRP algorithm for location recommendation comes with one major drawback: the absence of service time data that match the field conditions which is very important for computing the recommendation. Thus, MDVRP algorithm needs to be integrated with a communication mechanism such as Web Service, Remote Procedure Call (RPC), message passing, and publish/subscribe mechanism \cite{eugster_many_2003}.

In this research, the publish/subscribe mechanism is chosen because, in addition to its loose coupling characteristic \cite{eugster_many_2003}, it is also suitable for an information-driven system \cite{muhl_large-scale_2002}. \textit{Loose coupling} allows the publish/subscribe mechanism to work asynchronously, meaning the request and the reply do not have to be processed in sequence. Publishing and subscribing activities can still be done even if one party is offline. 


\begin{figure}[!]
	\centering
	\includegraphics[width=8.7cm]{Resources/Images/system-overview}
	\caption{The outline of the proposed system}
	\label{fig:system-overview}
\end{figure}


\autoref{fig:system-overview} shows the outline for the proposed system that consists of 3 (three) main components: the publisher, the subscriber, and the message broker. The communication between publisher and subscriber is built upon their similarity in either event or topic. In this research, the subscriber's current location is chosen as the communication topic. 


%-----------------------------------------------------------------------------%
\subsection{The Recommendation Publisher}
\label{ssec:recommendation-publisher}
%-----------------------------------------------------------------------------%
Communication between publishers and subscribers uses a topic as a basis. This communication is done indirectly through a message broker. Unfortunately, the publish/subscribe mechanism makes it impossible for brokers to notify when new topics are subscribed. Therefore, it is necessary to check regularly by using a topic watcher to find out new topics that are subscribed. \autoref{alg:topic-watcher} shows the referred algorithm.

\begin{algorithm}[!]
	\caption{Topic Watcher}
	\label{alg:topic-watcher}
	\begin{algorithmic}[1]
		\renewcommand{\algorithmicrequire}{\textbf{Input:}}
		\renewcommand{\algorithmicensure}{\textbf{Output:}}
		\REQUIRE $None$
		\ENSURE  $None$
		\\ $TP$ = Threadpool
		\\ $N$ = Number of locations
		\\ $M$ = Number of enumerators
		\WHILE {true}
		\STATE $C \leftarrow readAvailableTopicFromBroker()$	// channel
		\FOR {$m = 1$ to $len(C)$}
		\FOR {$n = 1$ to $N$}
		\IF {($C_m == L_n$)}
		\STATE $T_n = Thread(C_m, E_1...E_M, (unassigned) L_1...L_N)$		// E = enumerator
		\STATE submitThreadToThreadpool($T_n$, $TP$)
		\ENDIF
		\ENDFOR
		\ENDFOR
		\ENDWHILE
	\end{algorithmic}
\end{algorithm}

Every time a new topic (subscriber’s current location) emerges, a new thread will be created using that new topic as an ID. Accordingly, a thread pool is provided to accommodate all threads in an arrival order. By the time all locations has been assigned to each enumerator, the thread pool size will be equal to the number of topics (census/survey locations). 

Each thread will be running consecutively, one session for each thread. The reason why only one thread is executed is to avoid the recommended location conflict as illustrated in \autoref{fig:conflict-illustration}. The current running thread calls a global VRP solver procedure that acts as a solution/route finder. VRP solver procedure takes into account all $M$ enumerators and $(unassigned) N$ locations. This aims to prevent a local best solution, where 'next location' is the best location from one vehicle perspective only.

\begin{figure}[!]
	\centering
	\includegraphics[width=8cm]{Resources/Images/conflict-illustration}
	\caption{The illustration of recommendation conflict}
	\label{fig:conflict-illustration}
\end{figure} 

The VRP solver procedure in each thread will produce at least $1$ route and at most $M$ routes. Every generated route $R_i$ will be published and coupled with a topic $C_i$. The message broker will inform the publisher the number of subscribers receiving the message. If a route $R_i$ has at least one subscriber, the thread with ID $C_{R_i}$ will be canceled. This annulment is aimed to prevent a solution/route that has been received by a subscriber from being recomputed. The above steps are listed on \autoref{alg:vrp-worker}.

\begin{algorithm}[!]
	\caption{Recommendation Publisher}
	\label{alg:vrp-worker}
	\begin{algorithmic}[1]
		\renewcommand{\algorithmicrequire}{\textbf{Input:}}
		\renewcommand{\algorithmicensure}{\textbf{Output:}}
		\REQUIRE $TP$		// threadpool
		\ENSURE  $None$
		
		\WHILE {true}
		\STATE $T$ = popFirstThreadOrWaitNewThreadFromThreadpool()
		\STATE $R$ = VRPSolver($T$)
		\FOR {$j = 1$ to $len(R)$}
		\STATE $r$ = publish($C_{R_j}$, $R_j$)
		\IF {($r > 0$)}
		\STATE cancelSolver($T_j$)
		\ELSIF {($C_{T_i} \notin C_{R_j}$)}
		\STATE $T_i = Thread(C_{T_i}, V_m, (unassigned) E_1...E_N)$
		\ENDIF
		\ENDFOR
		\ENDWHILE	
	\end{algorithmic}
\end{algorithm}

It could happen that the running VRP solver with ID $C_i$ fails to produce the solution for topic $C_i$. In this situation, the topic $C_i$ will be re-queued in the thread pool by involving only the subscriber of that particular topic. This mechanism guarantees that every topic $C_i$ will have a solution/route. 

Despite the flexibility, the loose coupling characteristic also brings a shortcoming to publish/subscribe mechanism. The publisher's unawareness towards subscribers' identities caused the subscribers' location cannot be identified. Hence, a technique that can support information exchange between publisher and subscribers is needed. The idea is to use a shared memory contains subscribers' current location data that can be accessed by the publisher. 

The whole processes described above, from detecting a new topic to obtaining the solution, will keep being repeated until all customers are already assigned to each vehicle. In terms of data collection, the entire iterative processes will stop once all census/survey locations have been assigned to each enumerator. 

% \autoref{fig:publisher-algorithm} illustrates the workflow of the algorithm used in recommendation publisher.


%\begin{figure}[!]
%	\centering
%	\includegraphics[width=8cm]{Resources/Images/publisher-algorithm}
%	\caption{The Publisher Workflow}
%	\label{fig:publisher-algorithm}
%\end{figure} 


%-----------------------------------------------------------------------------%
\subsection{VRP Solver}
\label{ssec:vrp-solver}
%-----------------------------------------------------------------------------%
VRP solver is a module used for finding the best route/solution to visit all customers (census/survey locations). VRP solver is implemented on the publisher's side and called in each thread in the thread pool. In this research, VRP solver utilizes CoEAs which generates a competitive \textit{mean solution values} with relatively low CPU time compared to other algorithms \cite{de_oliveira_cooperative_2016}. 

The VRP solver works as follows:
\begin{enumerate}
	\item Problem creation \\
	Create a MDVRP instance that consists of $M$ vehicles (enumerators) and $N$ customers (locations).
	\item Problem decomposition \\
	Decompose the MDVRP into several subproblems using Nearest Insertion Heuristic (NIH) and Semi-greedy algorithm, which are parts of CoEAs. 
	\item Individual evolution. \\
	Before each evolution, each individual in every subproblem is evaluated to find the current best individual (CBI) of all subproblems. Then, each individual is evolved to generate new individuals. These new individuals will be evaluated to find the new best individual (NBI). If NBI is better than CBI, update CBI with NBI value. Otherwise, leave the CBI as it is. Repeat the evolution process until it reaches a particular time limit of 60 seconds or 40 seconds if there is no change happens to the CBI.    
\end{enumerate}

%-----------------------------------------------------------------------------%
\subsection{Message Broker}
\label{ssec:message-broker}
%-----------------------------------------------------------------------------%
A message broker is a component responsible for routing the message from publisher to subscriber based on the subscribed topic \cite{banavar_efficient_1999}. A publish/subscribe system can have a single broker or multi-broker. In single broker architecture, all subscribers and publishers are connected to one single broker, while in multi-broker architecture, every subscriber or publisher can connect to any nearest broker. This multi-broker architecture is also called distributed publish/subscribe system \cite{muhl_large-scale_2002} as depicted in \autoref{fig:pub_sub_distributed_ilustration}. In this research, the proposed design will implement distributed architecture as the census/survey locations are geographically scattered. 


\begin{figure}[!]
	\centering
	\includegraphics[width=7cm]{Resources/Images/pub_sub_distributed_ilustration}
	\caption{The illustration of distributed publish-subscribe architecture}
	\label{fig:pub_sub_distributed_ilustration}
\end{figure}


%-----------------------------------------------------------------------------%
\section{Result}
\label{sec:testing}
%-----------------------------------------------------------------------------%

%-----------------------------------------------------------------------------%
\subsection{Experimental Setup}
\label{ssec:experimental-setup}
%-----------------------------------------------------------------------------%

\textbf{Implementation.} Each component in this proposed system is implemented differently. Recommendation publisher is implemented in \verb!Python 2.7!. On the other hand, VRP Solver with the CoEAs algorithm is coded using \verb!C++! and compiled with \verb!C++ compiler 5.4.0 20160609!. Furthermore, the message broker is developed using \verb|Redis Cluster 3.2.6| with 6 nodes: 3 nodes for masters and the other 3 nodes for the slaves.

\textbf{Environment.} The experiment is conducted on Elementary OS Loki 64bit operating system and Quad-Core Intel® Core™ i3-4030U CPU @ 1.90GHz with 4 GB DDR3 RAM. Each Redis node is executed as a \verb|Docker| container on Debian Jessie 64bit operating system and Quad-Core Intel® Core™ i3-4030U CPU @ 1.90GHz with 4 GB RAM.

\textbf{Dataset.} This experiment is conducted using real administrative data, which consists of 182 locations and 15 enumerators. The distances and the time needed to travel between locations are measured using Google Maps Direction API \cite{google_google_2016}. While the service time for these two data is randomly generated based on Sudman \cite{sudman_time_1965} by following the normal distribution.

\textbf{Scenario.} The test will be run using 2 (two) scenarios: scenario of normal condition and delay condition. The dataset for testing in normal condition is generated by 4 (four) instances, where each instance has a different service time. While the dataset for testing in delay conditions is generated by 7 (seven) instances, where each instance has a different service time and also has varying delay.

\textbf{Metric.} The output from the proposed program (MDVRP algorithm based on CoEAs with publish/pubscribe mechanism) is compared against the output from the benchmark program that uses MDVRP algorithm based on CoEAs only. Both programs result in the routes for each vehicle which can be illustrated as follows:

\begin{itemize}
\item \textit{Enumerator} A = Loc1 $\rightarrow$ Loc5 $\rightarrow$ Loc15 $\rightarrow$ Loc6
\item \textit{Enumerator} B = Loc6 $\rightarrow$ Loc2 $\rightarrow$ Loc16 $\rightarrow$ Loc3
\item \textit{Enumerator} C = Loc4 $\rightarrow$ Loc8 $\rightarrow$ Loc14 $\rightarrow$ Loc 7
\item \textit{Enumerator} D = Loc9 $\rightarrow$ Loc10 $\rightarrow$ Loc11 $\rightarrow$ Loc12
\end{itemize}
where Loc is the location to visit. 

The total days spent for each enumerator route is calculated by summing up the service time and the transport time of all locations in that route, which every day is limited by 8 (eight) work hours. After that, the total days and standard deviation from all enumerators can be calculated.

Standard deviation is chosen as a metric because it represents the actual condition, where the low standard deviation means the total number of days spent more evenly for all enumerators. Therefore, a better program will have a smaller standard deviation.

%-----------------------------------------------------------------------------%
\subsection{Experimental Result}
\label{ssec:experimental-result}
%-----------------------------------------------------------------------------%

\textbf{Normal scenario.} In the testing process, each dataset instance is executed 100 times to prove that the results obtained during the test have a low standard error. Based on the test result using field data and using the assumption of normal condition, where there are no obstacles in enumeration, all tested instances resulted that the average of enumeration day between the proposed system and the comparison program are the same, that is 13 days for each enumerator, as depicted in  \autoref{fig:test_result_real_tw_mean_of_total_time}. In terms of standard deviation, however, it shows that the proposed system yields a lower standard deviation on all instances, as depicted in \autoref{fig:test_result_real_tw_stdev_of_total_time}.

In line with the average number of enumeration days and standard deviation, using the proposed system, the amount of time spent waiting until all enumerators complete their enumeration is lower than the comparison program, as figured in \autoref{fig:test_result_real_tw_mean_stdev_of_total_time}. So it can be concluded the proposed system is more efficient than the comparison system.

Based on the testing of normal conditions, it can be concluded that the proposed system can provide better recommendations. It means, the variation of the day of enumeration using the proposed system is lower, so the time required to wait for all enumerators to complete the enumeration is lower than the proposed program.

\begin{figure}[!]
	\centering
	\includegraphics[width=8.7cm]{Resources/Images/test_result_real_tw_mean_of_total_time_en}
	\caption{The average days spent by each enumerator from 100 tests normal condition on field dataset}
	\label{fig:test_result_real_tw_mean_of_total_time}
\end{figure}

\begin{figure}[!]
	\centering
	\includegraphics[width=8.7cm]{Resources/Images/test_result_real_tw_stdev_of_total_time_en}
	\caption{The standard deviation days spent by each enumerator from 100 tests normal condition on field dataset}
	\label{fig:test_result_real_tw_stdev_of_total_time}
\end{figure}

\begin{figure}[!]
	\centering
	\includegraphics[width=8.7cm]{Resources/Images/test_result_real_tw_mean_stdev_of_total_time_en}
	\caption{The maximum days spent by each enumerator from 100 tests normal condition on field dataset}
	\label{fig:test_result_real_tw_mean_stdev_of_total_time}
\end{figure}

\textbf{Delay scenario.} Similar to testing in a normal scenario, testing on a delay scenario is also executed 100 times each instance . Contrast with the results of the tests in the normal scenario, the results obtained in the test with the delay scenario show that the average day spent by an enumerator in the proposed system is longer than the benchmark program in all instances, as figured in \autoref{fig:test_result_delay_real_tw_mean_of_total_time}. However, the standard deviation of the proposed system is lower than that of the benchmark program, as depicted in \autoref{fig:test_result_delay_real_tw_stdev_of_total_time}, so the proposed system is more evenly distributed in term of day of the enumeration.

Although the average number of enumeration days using the proposed system takes longer time, the proposed system is still more efficient when applied to relatively low delay. \autoref{fig:test_result_delay_real_tw_mean_stdev_of_total_time} shows that using the proposed system, the total number of days required to wait until all enumerators completed their enumeration is lower in the instance with the relatively low delay, which are d01, d02, and d04. While in the instance with high delay, which are d03, d05, and d06, the benchmark program gives more efficient results.

\begin{figure}[!]
	\centering
	\includegraphics[width=8.7cm]{Resources/Images/test_result_delay_real_tw_mean_of_total_time_en}
	\caption{The average days spent by each enumerator from 100 tests delay condition on field dataset}
	\label{fig:test_result_delay_real_tw_mean_of_total_time}
\end{figure}

\begin{figure}[!]
	\centering
	\includegraphics[width=8.7cm]{Resources/Images/test_result_delay_real_tw_stdev_of_total_time_en}
	\caption{The standard deviation days spent by each enumerator from 100 tests delay condition on field dataset}
	\label{fig:test_result_delay_real_tw_stdev_of_total_time}
\end{figure}

\begin{figure}[!]
	\centering
	\includegraphics[width=8.7cm]{Resources/Images/test_result_delay_real_tw_mean_stdev_of_total_time_en}
	\caption{The maximum days spent by each enumerator from 100 tests delay condition on field dataset}
	\label{fig:test_result_delay_real_tw_mean_stdev_of_total_time}
\end{figure}

%-----------------------------------------------------------------------------%
\section{Conclusion and Future Works}
\label{sec:conclusion-future-works}
%-----------------------------------------------------------------------------%
This paper is aimed to solve the problem found in MDVRP based on CoEAS when used for generating a recommendation for enumeration locations. The proposed solution is to combine MDVRP based on CoEAS with publish/subscribe mechanism. The proposed system consists of 2 main components: the publisher and the message broker. The publisher contains a VRP solver which utilizes CoEAs algorithm.

Based on the results of the test using field data, the results show that the proposed system provides better results than the benchmark program on a scenario with no delay and a low delay scenario. While in scenarios with high delay, benchmark programs provide better results than proposed systems.

Publish/subscribe mechanism is not the only option that can be used to develop a real-time system. Further studies using other mechanisms such as Push/Pull and Request/Reply mechanism is required for comparison. A research on analyzing the performance of each mechanism in varying network conditions are also necessary to create a simulation of real field conditions.


% An example of a floating figure using the graphicx package.
% Note that \label must occur AFTER (or within) \caption.
% For figures, \caption should occur after the \includegraphics.
% Note that IEEEtran v1.7 and later has special internal code that
% is designed to preserve the operation of \label within \caption
% even when the captionsoff option is in effect. However, because
% of issues like this, it may be the safest practice to put all your
% \label just after \caption rather than within \caption{}.
%
% Reminder: the "draftcls" or "draftclsnofoot", not "draft", class
% option should be used if it is desired that the figures are to be
% displayed while in draft mode.
%
%\begin{figure}[!t]
%\centering
%\includegraphics[width=2.5in]{myfigure}
% where an .eps filename suffix will be assumed under latex, 
% and a .pdf suffix will be assumed for pdflatex; or what has been declared
% via \DeclareGraphicsExtensions.
%\caption{Simulation results for the network.}
%\label{fig_sim}
%\end{figure}

% Note that the IEEE typically puts floats only at the top, even when this
% results in a large percentage of a column being occupied by floats.


% An example of a double column floating figure using two subfigures.
% (The subfig.sty package must be loaded for this to work.)
% The subfigure \label commands are set within each subfloat command,
% and the \label for the overall figure must come after \caption.
% \hfil is used as a separator to get equal spacing.
% Watch out that the combined width of all the subfigures on a 
% line do not exceed the text width or a line break will occur.
%
%\begin{figure*}[!t]
%\centering
%\subfloat[Case I]{\includegraphics[width=2.5in]{box}%
%\label{fig_first_case}}
%\hfil
%\subfloat[Case II]{\includegraphics[width=2.5in]{box}%
%\label{fig_second_case}}
%\caption{Simulation results for the network.}
%\label{fig_sim}
%\end{figure*}
%
% Note that often IEEE papers with subfigures do not employ subfigure
% captions (using the optional argument to \subfloat[]), but instead will
% reference/describe all of them (a), (b), etc., within the main caption.
% Be aware that for subfig.sty to generate the (a), (b), etc., subfigure
% labels, the optional argument to \subfloat must be present. If a
% subcaption is not desired, just leave its contents blank,
% e.g., \subfloat[].


% An example of a floating table. Note that, for IEEE style tables, the
% \caption command should come BEFORE the table and, given that table
% captions serve much like titles, are usually capitalized except for words
% such as a, an, and, as, at, but, by, for, in, nor, of, on, or, the, to
% and up, which are usually not capitalized unless they are the first or
% last word of the caption. Table text will default to \footnotesize as
% the IEEE normally uses this smaller font for tables.
% The \label must come after \caption as always.
%
%\begin{table}[!t]
%% increase table row spacing, adjust to taste
%\renewcommand{\arraystretch}{1.3}
% if using array.sty, it might be a good idea to tweak the value of
% \extrarowheight as needed to properly center the text within the cells
%\caption{An Example of a Table}
%\label{table_example}
%\centering
%% Some packages, such as MDW tools, offer better commands for making tables
%% than the plain LaTeX2e tabular which is used here.
%\begin{tabular}{|c||c|}
%\hline
%One & Two\\
%\hline
%Three & Four\\
%\hline
%\end{tabular}
%\end{table}


% Note that the IEEE does not put floats in the very first column
% - or typically anywhere on the first page for that matter. Also,
% in-text middle ("here") positioning is typically not used, but it
% is allowed and encouraged for Computer Society conferences (but
% not Computer Society journals). Most IEEE journals/conferences use
% top floats exclusively. 
% Note that, LaTeX2e, unlike IEEE journals/conferences, places
% footnotes above bottom floats. This can be corrected via the
% \fnbelowfloat command of the stfloats package.


% conference papers do not normally have an appendix


% use section* for acknowledgment
% \section*{Acknowledgment}
% The authors would like to thank...





% trigger a \newpage just before the given reference
% number - used to balance the columns on the last page
% adjust value as needed - may need to be readjusted if
% the document is modified later
%\IEEEtriggeratref{8}
% The "triggered" command can be changed if desired:
%\IEEEtriggercmd{\enlargethispage{-5in}}

% references section

% can use a bibliography generated by BibTeX as a .bbl file
% BibTeX documentation can be easily obtained at:
% http://mirror.ctan.org/biblio/bibtex/contrib/doc/
% The IEEEtran BibTeX style support page is at:
% http://www.michaelshell.org/tex/ieeetran/bibtex/
% bibliographystyle{IEEEtran}
% argument is your BibTeX string definitions and bibliography database(s)
% bibliography{IEEEabrv,Paper.bib}
%
% <OR> manually copy in the resultant .bbl file
% set second argument of \begin to the number of references
% (used to reserve space for the reference number labels box)
% \begin{thebibliography}{1}

% \bibitem{IEEEhowto:kopka}
% H.~Kopka and P.~W. Daly, \emph{A Guide to \LaTeX}, 3rd~ed.\hskip 1em plus
%   0.5em minus 0.4em\relax Harlow, England: Addison-Wesley, 1999.

% \end{thebibliography}

\bibliographystyle{IEEEtran}
\bibliography{Resources/Bib/Reference}


% that's all folks
\end{document}


