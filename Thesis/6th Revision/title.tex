% !TeX root = draft-thesis.tex
\usepackage{setspace}
\usepackage{titling}
\usepackage{lipsum}
\usepackage{array}
\usepackage{graphicx}
\usepackage{ifthen}
\graphicspath{{images/}}
\DeclareGraphicsExtensions{.pdf,.jpeg,.png}

\usepackage{float}
\usepackage{titlesec}
\usepackage{fancyhdr}
\usepackage{tocloft}
%\usepackage{fontspec}
%\usepackage{polyglossia,etoolbox}
\usepackage[indonesian]{babel}
\usepackage[justification=centering]{caption}
\usepackage[top=3cm,right=3cm,bottom=3cm,left=4cm]{geometry}
\usepackage{etoolbox}% http://ctan.org/pkg/etoolbox

%\setmainfont[Ligatures=TeX]{Times New Roman}
 %\setdefaultlanguage{bahasai}
% Bibliografi to Daftar Pustaka
\addto\captionsindonesian{%
 \renewcommand{\bibname}%
   {Daftar Pustaka}%
 \renewcommand{\refname}{Daftar Pustaka}%
}

\usepackage{mathptmx}
\usepackage[bookmarks=true,hidelinks]{hyperref}
\usepackage{bookmark}
\usepackage{bibspacing}
\setlength{\bibspacing}{1.1\baselineskip}


% \gappto\captionsbahasai{\renewcommand{\bibname}{Daftar Pustaka}}

%%titlesec
\makeatletter
\def\ttl@mkchap@i#1#2#3#4#5#6#7{%
  \ttl@assign\@tempskipa#3\relax\beforetitleunit
  %\vspace*{\@tempskipa}% NEW
  \global\@afterindenttrue
  \ifcase#5 \global\@afterindentfalse\fi
  \ttl@assign\@tempskipb#4\relax\aftertitleunit
  \ttl@topmode{\@tempskipb}{%
    \ttl@select{#6}{#1}{#2}{#7}}%
  \ttl@finmarks  % Outside the box!
  \@ifundefined{ttlp@#6}{}{\ttlp@write{#6}}}
\makeatother

\renewcommand{\thechapter}{\Roman{chapter}}
\titleformat{\chapter}[display]
    {\large\bfseries\centering}
    {\large\chaptertitlename\ \thechapter}
    {0in}
    {}
\titlespacing*{\chapter}
    {0pt} % left
    {0pt} % before
    {2\baselineskip} % after
\titleformat{\chapter}[display]
    {\large\bfseries\centering}
    {\large\chaptertitlename\ \thechapter}
    {0in}
    {}
\titlespacing*{\section}
    {0pt} % left
    {\baselineskip} % before
    {0pt} % after

\titleformat{\section}[hang]
    {\bfseries}
    {\thesection\ }
    {0in}
    {}

\titlespacing*{\subsection}
    {0pt} % left
    {\baselineskip} % before
    {0pt} % after

\titleformat{\subsection}[hang]
    {\bfseries}
    {\thesubsection\ }
    {0in}
    {}
    
\titleclass{\subsubsubsection}{straight}[\subsection]

\newcounter{subsubsubsection}[subsubsection]
\renewcommand\thesubsubsubsection{\thesubsubsection.\arabic{subsubsubsection}}
 
\titlespacing*{\subsubsubsection}
  {0pt} % left
  {\baselineskip} % before
  {0pt} % after

\titleformat{\subsubsubsection}[hang]
 {\bfseries}
 {\thesubsubsubsection\ }
 {0in}
 {}
    
    \makeatletter
    \def\toclevel@subsubsubsection{4}
    \def\l@subsubsubsection{\@dottedtocline{4}{7em}{4em}}
    \setcounter{secnumdepth}{4}
    \setcounter{tocdepth}{4}
    \cftsetindents{subsubsection}{0.6in}{0.6in}
   \makeatother


%%%customtitle
\makeatletter
\newcommand{\thegelar}{Magister}
\newcommand{\theenglishtitle}{\thetitle}
\newcommand{\englishtitle}[1][\thetitle]{
    \renewcommand{\theenglishtitle}{#1}
}
\newcommand{\gelar}[1]{\renewcommand{\thegelar}{#1}}
\newcommand{\thedoctype}{Tesis}
\newcommand{\doctype}[1]{%
    \renewcommand{\thedoctype}{#1}
   }
\newcommand{\theprodi}{Magister Informatika}
\newcommand{\prodi}[1]{\renewcommand{\theprodi}{#1}}

\newcommand{\theenglishprodi}{\theprodi}
\newcommand{\englishprodi}[1][\theprodi]{\renewcommand{\theprodi}{#1}}

\newcommand{\theopsi}{Teknologi Informasi}
\newcommand{\opsi}[1]{\renewcommand{\theopsi}{#1}}
\newcommand{\thefak}{Sekolah Teknik Elektro dan Informatika}
\newcommand{\fakultas}[1]{\renewcommand{\thefak}{#1}}
\newcommand{\theuni}{Institut Teknologi Bandung}
\newcommand{\universitas}[1]{\renewcommand{\theuni}{#1}}
\newcommand{\thepembsatuname}{John Smith}
\newcommand{\thepembsatunip}{123123}
\newcommand{\thepembduaname}{}
\newcommand{\thepembduanip}{}
\newcommand{\pembimbingsatu}[2]{
    \renewcommand{\thepembsatuname}{#1}
    \renewcommand{\thepembsatunip}{#2}
    }
\newcommand{\pembimbingdua}[2]{
    \renewcommand{\thepembduaname}{#1}
    \renewcommand{\thepembduanip}{#2}
    }

\newcommand{\prodistring}{%
        Program Studi \theprodi\\
        Opsi \theopsi\\
        \thefak\\
        \theuni\\
}
\newcommand{\ifempty}[3]{%
    \sbox0{#1}%
    \ifdim\wd0=0pt %
    {#2}% if #1 is empty
    \else%
    \ifdim0pt=\dimexpr\ht0+\dp0\relax
    {#2}% if #1 is empty
    \else
    {#3}% if #1 is not empty
    \fi
    \fi%
    }
\makeatother

\makeatletter
\newcommand{\makecustomtitle}[1]{%
    \begin{titlepage}
        \begin{center}
            {\large\bfseries  %
                \expandafter \uppercase \expandafter{\@title}%
                \\}
            % ----------------------------------------------------------------
            % \vspace{2cm}
            %	{Thesis  submitted to} \\[5pt]
            %	\emph{{Your University}}\\[2cm]
            \vspace{2cm}
            \expandafter\uppercase\expandafter{\thedoctype}\\[1.5cm]
            Karya tulis sebagai salah satu syarat\\
            Untuk memperoleh gelar {\thegelar} dari\\
            \theuni\\[1.5cm]
            \textsc{\Large{{ }}} \\[5pt]
            { } \vspace{0.4cm} 
            % {By}\\[5pt] {\Large \sc {Me}}
            \vfill
            % ----------------------------------------------------------------
            
            \vspace{6ex}
            {\small\bfseries{oleh}\\}
            {\normalsize\bfseries
                \expandafter\uppercase\expandafter{\@author}%
                \\	
            }
            \vspace{10ex}
            % ----------------------------------------------------------------
            \includegraphics[height=3.5cm]{#1}\\[5pt]
            \vspace{2cm}
            {\bfseries \large \uppercase
                {\theuni}\\[5pt]
                {\@date}
            }
            \vfill
            
        \end{center}
    \end{titlepage}
}
\makeatother


\makeatletter
\newcommand{\makecustompengesahan}[1][Bandung]{%
\newpage\addcontentsline{toc}{chapter}{Lembar Pengesahan}
        \begin{center}
            {\large\bfseries  %
                \expandafter \uppercase \expandafter{\@title}%
                \\}
            \vspace{2cm}
            \expandafter\uppercase\expandafter{\thedoctype}\\[1.5cm]
            {\small\bfseries{oleh}\\}
            {\normalsize\bfseries
                \expandafter\uppercase\expandafter{\@author}%
                \\	
            }\vspace{1cm}
            \textbf{\prodistring}
            \vspace{2cm}
            \begin{center}
                Menyetujui\\
%                #1, \today\\
                \vspace{3em}
                \ifempty{\thepembduaname}{%
                    \begin{tabularx}{0.5\linewidth}{X}
                        \begin{center}
                            {Pembimbing I\\
                             \vspace{2cm}
                             \hrule\vspace{.5em}
                             \textbf{(\thepembsatuname)}\\
                             NIP. \thepembsatunip
                            }
                        \end{center}
                    \end{tabularx}
                }{%
                   \begin{tabularx}{\textwidth}{XX}
                       \centering{
                                Pembimbing I
                               \vspace{5em}
                               \hrule\vspace{.5em}
                               \textbf{(\thepembsatuname)}\\
                               NIP. \thepembsatunip}%
                        &%
\centering{Pembimbing II\vspace{5em}
                            \hrule\vspace{.5em}
                            \textbf{(\thepembduaname)}\\
                            NIP. \thepembduanip}
                \end{tabularx}
                }

            \end{center}

            \vfill
            
        \end{center}
    
}

\newenvironment{abstrak}[1][english]{%
    \newcommand{\abstraktitle}{\theenglishtitle}
    \newcommand{\abstrakprodi}{\theenglishprodi}
    \newcommand{\abstrakname}{Abstract}
    \newcommand{\abstrakoleh}{by}
    \ifthenelse{\equal{#1}{indo}}%
    {\renewcommand{\abstraktitle}{\thetitle}
        \renewcommand{\abstrakprodi}{\theprodi}
        \renewcommand{\abstrakname}{Abstrak}
        \renewcommand{\abstrakoleh}{oleh}}%
    {}
    \newpage\addcontentsline{toc}{chapter}{\abstrakname}
    \chapter*{\abstrakname}
    \begin{center}
        {\large\bfseries  %
            \expandafter \uppercase \expandafter{\abstraktitle}%
        }\\\vspace{1em}
        \abstrakoleh\\\vspace{1em}
        {\bfseries \theauthor\\
        (\abstrakprodi)}
    \end{center}
    }{\clearpage}

\newenvironment{custompage}[1]{
    \newpage\addcontentsline{toc}{chapter}{#1}
    \chapter*{#1}
    }{\clearpage}

\makeatletter
\patchcmd{\@chapter}{%
	\addtocontents{lof}{\protect\addvspace{10\p@}}%
	\addtocontents{lot}{\protect\addvspace{10\p@}}%
}{}{\message{Success to patch \string\@chapter}}%
{\message{Failed to patch \string\@chapter}}
\makeatother



%%fancyhdr
\setlength{\parskip}{2.5ex plus 1ex minus 1ex}
\setlength{\parindent}{0cm}
\pagestyle{fancyplain}
\fancyhf{}
\fancyfoot[reo]{\thepage}
  \renewcommand{\headrulewidth}{0pt} % remove lines as well
  \renewcommand{\footrulewidth}{0pt}
  
%% tocloft : toc lot lof modifier
\setcounter{tocdepth}{4}
\renewcommand{\cfttoctitlefont}{\bfseries\hfill}
\renewcommand{\cftaftertoctitle}{\hfill}
\setlength{\cftbeforetoctitleskip}{0pt}
\setlength{\cftsecindent}{7.5pt}
\setlength{\cftsubsecindent}{15pt}
\setlength{\cftbeforechapskip}{0.1ex}
\renewcommand{\cftchapfont}{}
\renewcommand{\cftchappagefont}{}
\renewcommand{\cftchapleader}{\cftdotfill{\cftdotsep}}
\setlength{\cftchapnumwidth}{4em}
\renewcommand{\cftchappresnum}{\chaptername\ }

\renewcommand{\cftloftitlefont}{\bfseries\hfill}
\renewcommand{\cftafterloftitle}{\hfill}
\setlength{\cftbeforeloftitleskip}{0pt}
\setlength{\cftfigindent}{0pt}
\setlength{\cftfignumwidth}{7.5em}
\renewcommand{\cftfigpresnum}{\figurename\ }

\renewcommand{\cftlottitlefont}{\bfseries\hfill}
\renewcommand{\cftafterlottitle}{\hfill}
\setlength{\cftbeforelottitleskip}{0pt}
\setlength{\cfttabindent}{0pt}
\setlength{\cfttabnumwidth}{7.5em}
\renewcommand{\cfttabpresnum}{\tablename\ }

%% no hyph
%\pretolerance=10000
%\tolerance=2000
%\emergencystretch=8pt
\usepackage{cite}
\onehalfspacing